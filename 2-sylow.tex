\chapter{Sylow Theory}
\section{Group Actions}
\begin{defn}
    Let\/ $G$ be a group and $X$ a set. An \textsl{action} of\/ $G$ on $X$ is a map
    \begin{align*}
        G\times X&\to X\\
        (\omega,x)&\mapsto \omega\cdot x,
    \end{align*}
    satisfying $1\cdot x=x$ and\/ $\nu\cdot(\omega\cdot x)=\nu\omega\cdot x$ for all\/ $x\in X$ and all\/ $\omega,\nu\in G$.

    In this case we will also say that\/ $G$ \textsl{acts on\/} $X$. There is also the notion of \textsl{right} action in which\/ $G$ acts from the right on $X$.
\end{defn}

\begin{ntn}
    If\/ $G$ acts on $X$, we will sometimes write $\omega x$ for $\omega\cdot x$.
    
    Given $\omega\in G$, we will consider the map
    \begin{align*}
        \sigma_\omega\colon X&\to X\\
        x&\mapsto \omega x
    \end{align*}
    which is a bijection with inverse $\sigma_{\omega^{-1}}$.
    
    Note that the map
    \begin{align*}
        \sigma\colon G&\to\Sym(X)\\
        \omega&\mapsto \sigma_\omega
    \end{align*}
    is a morphism of groups.
\end{ntn}

\begin{rem}\label{actions-are-group-morphisms}
    As we just pointed out, every action of\/ $G$ on\/ $X$ determines a morphism of groups\/ $G\to\Sym(X)$. The converse is also true. Every group morphism\/ $\sigma\colon G\to\Sym(X)$ determines the action defined by: $\omega\cdot x=\sigma_\omega(x)$. This is indeed an action:
    $$
        1\cdot x=\sigma_1(x)=\id_X(x)=x\quad\text{\rm and}\quad
        \omega\nu\cdot x=\sigma_{\omega\nu}(x)=\sigma_\omega(\sigma_\nu(x))=\omega\cdot(\nu\cdot x).
    $$
    Moreover, both constructions are reciprocal because the morphism induced by this action is no other than\/~$\sigma$.
\end{rem}

\begin{defn}
    If\/ $G$ acts on $X$, the \textsl{kernel} of the action $\ker(\sigma)$ is characterized as the normal subgroup of\/ $G$ consisting of all elements $\omega$ that satisfy
    $$
        \omega\cdot x=x\quad\textrm{\rm for all }x\in X.
    $$
    The action is \textsl{faithful} when the kernel is trivial.
\end{defn}

\needspace{2\baselineskip}
\begin{xmpls}${}$
\begin{enumerate}[\rm 1.]
    \item Take $X=G$ and the action of\/ $G$ on $G$ to be the group operation, which is called the \textsl{regular} action. Note that this action is faithful, and so\/ $G$ is isomorphic to a normal subgroup of\/ $\Sym(G)$.

    \item Now consider the action of\/ $G$ on itself defined by conjugation, i.e., 
    $$
        \omega\cdot x= x^\omega.
    $$
    which is actually a (left) action because
    $$
        \nu\omega\cdot x=x^{\nu\omega}=(x^\omega)^\nu=\nu\cdot(\omega\cdot x).
    $$
    The kernel of this action is the center $Z(G)$.

    \item Take $X=2^G$, the set of parts of $G$. We can define here two left actions $A\mapsto A^\omega$ and $A\mapsto\omega A$ and one right action $A\mapsto A\omega$. Note that these actions remain well-defined if we restrict $X$ to a certain subset of parts that are preserved by conjugation and left (resp.~right) multiplication. For instance, $X$ could be {\rm(a)}~the set of all subgroups of\/ $G$, {\rm(b)}~the set of parts of a given cardinality or {\rm(c)}~the set $\lco GH$ of left cosets of a given subgroup~$H$.
\end{enumerate}
\end{xmpls}

\begin{defn}
    Let $G$ be a group acting on a set $X$. The \textsl{stabilizer} of an element $x\in X$ is the subgroup
    $$
        G_x = \set{\omega\in G\mid \omega\cdot x=x}.
    $$
\end{defn}

\begin{rem}
    Every subgroup\/ $H$ of a group $G$ that acts on a set $X$ inherits an action on~$X$. In that case~$H_x=G_x\cap H$ for all $x\in X$.
\end{rem}

\needspace{2\baselineskip}
\begin{rem}\label{basic-stabilizers}
    Let $G$ be a group. Then
    \begin{enumerate}[\rm i)]
        \item \textrm{\rm regular action:} $G_\omega=\gen1$, $\omega\in G$.
        \item \textrm{\rm conjugation:} $G_\omega=C_G(\omega)$, $\omega\in G$.
        \item \textrm{\rm conjugation on $2^G$:} $G_X=N_G(X)$, $X\in 2^G$.
    \end{enumerate}
\end{rem}

\begin{lem}\label{stabilizer-conjugate}
    Let\/ $G$ be a group acting on $X$. Then
    $$
        y = \omega\cdot x\implies G_y = (G_x)^\omega.
    $$
    If the group acts from the right, then
    $$
        y = x\cdot\omega\implies G_x = (G_y)^\omega,
    $$
    In both cases the stabilizers are conjugate in\/ $G$.
\end{lem}

\begin{proof}
\begin{align*}
    \zeta\in(G_x)^\omega &\iff \zeta=\omega\nu\omega^{-1},\quad\textrm{\rm for some }\nu\in G_x\\
        &\implies \zeta\cdot y=\omega\nu\omega^{-1}\cdot y=\omega\nu\cdot x
            =\omega\cdot x=y\\
        &\iff \zeta\in G_y
\end{align*}
The other inclusion follows from this one applied to $\omega^{-1}$, which interchanges the roles of $x$ and $y$:
$$
    (G_y)^{\omega^{-1}} \subseteq G_x.
$$
The proof is similar when the action is from the right.  \end{proof}


\begin{lem}\label{core}
    Let\/ $G$ be a group and $H$ a subgroup. Consider the left action of\/~$G$ on the set\/ $\lco GH$ of left cosets of\/ $H$ given by $\omega\cdot \zeta H = \omega\zeta H$. Then
    $$
        G_{\omega H}= H^\omega.
    $$
\end{lem}

\begin{proof} This follows from the previous lemma because
$$
    G_H=\set{\omega\in G\mid \omega H=H} = H.
$$
\end{proof}

\begin{defn}\label{Core_G}
    Given a group\/ $G$ and a subgroup\/ $H$ the \textsl{core} of\/ $H$ in $G$ is the subgroup
    $$
        \Core_G(H) = \bigcap_{x\in G}H^x.
    $$
\end{defn}

\begin{rem}\label{G-to-Sym(X)}
    Consider the map
    \begin{align*}
        \sigma\colon G&\to\Sym(\lco GH)\\
        \omega&\mapsto\sigma_{\omega},
    \end{align*}
    where $\sigma_\omega\colon (\lco GH)\to(\lco GH)$ is given by $\sigma_\omega(xH)=\omega xH$. Since
    $$
        \sigma_{\omega\nu}(x) = \omega\nu xH = \sigma_\omega(\sigma_\nu(xH)),
    $$
    we see that $\sigma$ defines a morphism from $G$ to $\Sym(\lco GH)$.
\end{rem}


\begin{lem}\label{core=ker}
    $\Core_G(H)=\ker(\sigma)$.
\end{lem}

\begin{proof}     
    In fact, the kernel of $\sigma$ is the intersection of all stabilizer groups, which is precisely the core as shown in {\rm Lemma~\ref{core}}.  \end{proof}

\begin{prop}\label{core-cap-normal}
    Let\/ $H$ be a subgroup of\/ $G$. If\/ $N$ is normal, then
    $$
        \Core_G(H)\cap N= \Core_G(H\cap N).
    $$
\end{prop}

\begin{proof} Given $x\in G$, it is easy to see that $H^x\cap N=(H\cap N)^x$. Therefore,
$$
    \Core_G(H)\cap N = \bigcap_{x\in G}H^x\cap N = \bigcap_{x\in G}(H\cap N)^x
        =\Core_G(H\cap N),
$$
as wanted.  \end{proof}

\begin{thm}\label{largest-normal-subgroup}
    Let $G$ be a group and\/ $H$ a subgroup of\/ $G$. Then
    \begin{enumerate}[\rm a)]
        \item $\Core_G(H)$ is the largest normal subgroup contained in $H$.
        \item The quotient $G/\Core_G(H)$ is isomorphic to a subgroup of\/ $\Sym(\lco GH)$. In particular, if the index $\abs{G:H}=n$, then $G/\Core_G(H)$ is isomorphic to a subgroup of\/~$S_n$.
    \end{enumerate}
\end{thm}

\begin{proof}${}$

\begin{enumerate}[\rm a)]
    \item By {\rm Lemma~\ref{core=ker}} $\Core_G(H)$ is normal. It is included in $H$ because $1\in G$ and $H^1=H$.
    
    If $N\normal G$ and $N\subseteq H$, then $N=N^x\subseteq H^x$ for all $x\in G$ and therefore $N\subseteq\Core_G(H)$.

    \item According to {\rm Lemma~\ref{core=ker}} $G/\Core_G(H)$ is isomorphic to a subgroup of\/ $\Sym(\lco GH)$. Then the conclusion follows because $n=|G:H|$.
\end{enumerate}
\end{proof}

\begin{cor}\label{n!-theorem}{\rm[$n!$ theorem]}
    Let $G$ be a group and\/ $H \subseteq G$ a subgroup with $\abs{G:H} = n$. Then $H$ contains a normal subgroup\/ $N$ of\/ $G$ such that $\abs{G:N}$ divides $n!$.
\end{cor}

\begin{proof} Take $N=\Core_G(H)$.  \end{proof}

\begin{defn}
    A group is \textsl{simple} if it doesn't have any nontrivial normal subgroup.  {\rm[cf.~Exercise~\ref{exr:simple-group-mention}]}
\end{defn}

\begin{cor}\label{oreder-divide-n!}
    Let $G$ be simple and contain a subgroup of index $n > 1$. Then $\abs G$ divides $n!$
\end{cor}

\begin{proof} Let $H$ be a subgroup with $\abs{G:H}=n$. By the previous corollary, there is a normal group $N\subseteq H$ such that $\abs{G:N}$ divides $n!$. Since $n>1$, $H$ is proper and therefore $N=\gen1$. Thus, $\abs G=\abs{G:N}$ divides $n!$.
\end{proof}

\begin{defn}
    Let\/ $G$ be a group acting on a set $X$. Given an element $x\in X$, its \textsl{orbit} is the set
    $$
        {\cal O}_x = \set{\omega\cdot x\mid \omega\in G}. 
    $$
    Since two orbits are equal or disjoint, they induce a partition of\/ $X$.
\end{defn}

\needspace{2\baselineskip}
\begin{xmpls}${}$
\begin{enumerate}[\rm 1)]
    \item If\/ $H$ is a subgroup of\/ $G$, then $H\times G\to G$, defined by $(y,\omega)\mapsto y\cdot\omega$, is an action of\/ $H$ on\/ $G$. The orbits of this action are the right cosets $H\omega$.

    \item For the case\/ $X=G$ we have the conjugate map $(x,\omega)\mapsto x\omega x^{-1}$. The orbit of an element $\omega$ is known as the \textsl{conjugacy class} of\/ $\omega$ and denoted by~$[\omega]$, i.e.,
    $$
        [\omega] = {\cal O}_\omega = \set{x\omega x^{-1}\mid x\in G}.
    $$
    By\/ {\rm Proposition \ref{normal-closure}}, the \textsl{normal closure} of the subgroup $\gen\omega$ is
    $$
        \gen\omega^*=\gen{[\omega]}.
    $$
\end{enumerate}
\end{xmpls}

\begin{rem}
    Let\/ $G$ be a group acting on\/ $X$. If\/ $H$ is a subgroup and\/ ${\cal O}^H_x$ denotes the orbit of\/ $x$ under the action inherited by\/ $H$, then ${\cal O}^H_x\subseteq{\cal O}_x$, sometimes strictly.
\end{rem}

\begin{thm}{\rm[The Fundamental Counting Principle]}\label{fundamental-counting-principle}
    Let\/ $G$ act on $A$ and suppose that\/ $\cal O$ is one of the orbits. Let $a \in\cal O$ and write $G_a = \set{x \in G \mid x\cdot a = a}$ for the stabilizer of\/ $a$. Let\/ $\lco G{G_a}$ be the set of left cosets of\/ $G_a$ in~$G$. Then there is a bijection\/ $\theta:(\lco G{G_a})\to\cal O$ such that\/ $\theta(xG_a) = x\cdot a$. In particular, $\abs{\cal O} = \abs{G:G_a}$.
\end{thm}

\begin{proof}
\begin{align*}
\intertext{\quad$\theta$ is well defined:}
    xG_a =yG_a &\implies x=yz\textrm{ for some }z\in G_a\\
        &\implies x\cdot a= yz\cdot a= y\cdot a.
\intertext{\quad$\theta$ is injective:}
    x\cdot a=y\cdot a&\implies y^{-1}x\in G_a\\
    &\implies xG_a = y(y^{-1}xG_a) \subseteq yG_a\\[0.05 in]
    &\implies xG_a=yG_a.
\end{align*}
The surjectivity of\/ $\theta$ is trivial because ${\cal O}={\cal O}_a$.  \end{proof}

\begin{cor}
    Let $G$ be a finite group, $\omega \in G$ and\/ $[\omega]$ the conjugacy class of\/~$\omega\in G$. Then $\abs{[\omega]} = \abs{G:C_G(\omega)}$.
\end{cor}

\begin{proof} Consider the conjugate action. Then,
\begin{align*}
    \abs{[\omega]} &= \abs{{\cal O}_\omega}&&\textrm{; $[\omega]={\cal O}_\omega$}\\
        &= \abs{\set{xG_\omega\mid x\in G}}&&\textrm{; Thm.}\\
        &= \abs{\set{xC_G(\omega)\mid x\in G}}&&\textrm{; Rem.~\ref{basic-stabilizers}}\\
        &= \abs{G:C_G(\omega)}&&\textrm{; Rem.~\ref{left-index}}
\end{align*}
 \end{proof}

\begin{cor}\label{p-groups-have-center}
    Every nontrivial group whose order is a power of a prime has a nontrivial center. In symbols $\abs G=p^e\implies Z(G)\ne\gen1$.
    
    \textrm{\rm[See also Theorem~\ref{nontrivial-center}]}
\end{cor}

\begin{proof} Since $[\omega]$ is an orbit ${\cal O}_\omega$ and orbits partition $G$, there exist $\omega_1,\dots,\omega_n$ and $\zeta_1,\dots,\zeta_m$ such that $G$ can be decomposed as the disjoint union
\begin{align*}
    G &= \bigcup_{i=1}^n[\omega_i] \cup \bigcup_{j=1}^m[\zeta_j] \cup[1]\\
    \intertext{where}
    \abs{[\omega_i]}&>1,\quad\abs{[\zeta_j]}=1\quad\text{and}\quad \zeta_j\ne1.
\end{align*}
 As the second union is clearly $Z(G)\setminus\set1$, we can write
 $$
    \abs G = \sum_{i=1}^n|[\omega_i]| + |Z(G)\setminus\set1| + 1.
 $$
 Now, given that $[\omega]=G$ if, and only if, $\omega=1$, and in such a case $G$ would be trivial, the previous corollary implies that every $\abs{[\omega_i]}$ is a positive power of $p$. In particular, $\abs{[\omega_i]}\equiv0$ module $p$ and so
 $$
    |Z(G)\setminus\set1|\equiv-1\pmod p,
 $$
which implies that $Z(G)\ne\gen1$.  \end{proof}

\begin{cor}\label{p-squared-is-abelian}
    If\/ $|G|=p^2$ then $G$ is abelian. \textrm{\rm [cf.~Problem~\ref{problem-1.D.9}]}
\end{cor}

\begin{proof} Assume it is not. By the previous corollary, $|Z(G)|=p$. Then we can take $z\in Z(G)$ with $\ord(z)=p$. Thus, if $x\in G\setminus Z(G)$, then $G=\gen z\gen x$, which is clearly abelian.  \end{proof}
 
\begin{cor}\label{conjugate-count}
    Let\/ $G$ be a finite group and\/ $H \subseteq G$ a subgroup. Then the total number of distinct conjugates of\/ $H$ in\/ $G$ is\/ $\abs{G : N_G(H)}$.
\end{cor}

\begin{proof} Let $X$ be the set of subgroups of\/ $G$ and let $G$ act on $X$ by conjugation. The conjugates of\/ $H$ form the orbit\/ ${\cal O}_H$. According to the theorem
$$
    \abs{{\cal O}_H}=|G:G_H|=|G:N_G(H)|
$$
by Remark \ref{basic-stabilizers}.  \end{proof}

\textbf{Observation.} {[\rm Wording by chatGPT]} The corollary highlights the significance of understanding the actions of groups on sets in group theory. It demonstrates that by carefully selecting an action of a group on a specific set, such as the set of subgroups, one can deduce properties of the group that are not directly related to the set itself. This illustrates the utility of actions of groups on sets as a tool for understanding and analyzing the structure of groups.

\subsection{Problems A}

\begin{probl}\label{problem-1.A.1}
    Let $H$ be a subgroup of prime index $p$ in the finite group $G$, and suppose that no prime smaller than\/ $p$ divides $\abs G$. Prove that $H\normal G$.
\end{probl}

\begin{solution} Suppose that $H$ is not normal. Put $n=\abs G$. By the $n!$~theorem~\ref{n!-theorem}, there exists a normal subgroup $N$ such that $N\subseteq H$ and $r=\abs{G:N}$ divides $p!$. Since $N\subseteq H\varsubsetneq G$, $r>1$. Moreover, $r\mid n$. The hypothesis implies that $p$ is the only prime that may divide $r$. Thus $r=p^e$ and $r\mid p!$. It follows that $e=1$, i.e., $\abs{G: N}=p$. From Corollary~\ref{group-index-product} we get
$$
    p=\abs{G:N}=\abs{G:H}\abs{H:N}= p\abs{H:N}.
$$
Thus, $\abs{H:N}=1$, i.e., $H=N$.  \end{solution}

\begin{probl}\label{HwK}
    Given subgroups $H$, $K \subseteq G$ and an element $\omega \in G$, the set $H\omega K = \set{ x\omega y \mid x \in H,\; y \in K }$ is called an \textsl{$(H,K)$-double coset}. In the case where $H$ and\/ $K$ are finite, show that\/ $|H\omega^{-1} K|=\abs H\abs K/\abs{H^\omega\cap K}$.
\end{probl}

\begin{solution} The bijective function $z\mapsto \omega^{-1} z$ maps $H^\omega K$ onto $H\omega^{-1}K$. Therefore, by Lemma~\ref{HK-cardinality}, we get
$$
    |H\omega^{-1}K|=\abs{H^\omega K}=\abs{H^\omega}\abs K/\abs{H^\omega\cap K}
        = \abs H\abs K/\abs{H^\omega\cap K}
$$
 \end{solution}

\begin{probl}\label{problem-1.A.3} Suppose that $G$ is finite and that $H , K \subseteq G$ are subgroups. 
    \begin{enumerate}[\rm a)]
        \item Show that $\abs{H:H \cap K} \leq \abs{G:K}$, with equality if, and only if, $HK=G$.
        \item If\/ $\abs{G:H}$ and $\abs{G:K}$ are coprime, show that $HK=G$.
    \end{enumerate}
\end{probl}

\begin{solution} See Corollaries \ref{index-inequality} and \ref{coprime-indexes}.  \end{solution}

\begin{probl}\label{problem-1.A.4}
    Suppose that $G = H K$, where $H$ and $K$ are subgroups. Show that also $G = H^xK^y$ for all elements $x,y \in G$. Deduce that if\/ $G = HH^x$ for a subgroup $H$ and an element $x \in G$, then $H = G$.
\end{probl}

\begin{solution} For the first part take $x\in G$. Since $KH=G$, we can pick $h\in H$ and $k\in K$ such that $x=kh$. Then,
$$
    H^x=H^{kh}=(H^h)^k=H^k.
$$
This means that for the product $H^xK^y$ we may assume that $x\in K$ and $y\in H$. Put\/ $\omega=x^{-1}y$. Thus, given $z\in G$, since $x^{-1}zy\in G=HK$, we obtain $z\in xHKy^{-1}$. Now, using that $KH=HK$, we can pick $a\in H$ and $b\in K$ such that $\omega=ab$. Then, $H\omega K=HabK=HK$. In consequence,
$$
    z\in xHKy^{-1}=xH\omega Ky^{-1}=H^xK^y.
$$
For the second part assume that $G=HH^x$. Applying the first part to $H$ and $K=H^x$, we can conclude that $G=H^1K^{x^{-1}}=HH=H$.  \end{solution}

\begin{probl}
    An action of a group\/ $G$ on a set\/ $S$ is \textsl{transitive} if\/ $S$ consists of a single orbit. Equivalently, $G$ is transitive on\/ $S$ if for every choice of points $a,b \in S$, there exists an element $x \in G$ such that $x \cdot a = b$. Now assume that a group\/ $G$ acts transitively on each of two sets $S$ and\/ $T$. Prove that the natural induced action of\/ $G$ on the Cartesian product $S \times T$ is transitive if, and only if, $G=G_\alpha G_\beta$ for some choice of\/ $\alpha \in S$ and\/ $\beta \in T$.
    
    \textrm{\rm Hint. Show that if $G_\alpha G_\beta=G$ for some $\alpha \in S$ and $\beta \in T$, then in fact, this holds for all $a \in S$ and $b \in T$.}
\end{probl}
    
\begin{proof} 
\begin{description}
    
\item[\rm\textit{if\/ } part:] First observe that $G_a^\omega=G_{\omega^{-1}\cdot a}$. Therefore, by the previous problem, $G_{x\cdot\alpha}G_{y\cdot\beta}=G$ for all $x,y\in G$, which, in view of the hypothesis, implies that $G_sG_t=G$ for every $(s,t)\in S\times T$.

Let $(u,v)\in S\times T$ be another pair. We have to find $z\in G$ such that
\begin{align*}
    \begin{cases}
        z\cdot s=u,\\
        z\cdot t=v
    \end{cases}
\end{align*}
By hypothesis there exist $x,y\in G$ such that $x\cdot s=u$ and $y\cdot t=v$. Consider $xy^{-1}$. Since $G_uG_v=G$, we can write $xy^{-1}$ as a product $x_u^{-1}y_v$, where $x_u\in G_u$ and $y_v\in G_v$. Then $z=x_ux=y_vy$ satisfies
\begin{align*}
    \begin{cases}
        z\cdot s = x_ux\cdot s=x_u\cdot u=u\\
        z\cdot t = y_vy\cdot t= y_v\cdot v=v,        
    \end{cases}
\end{align*}
as wanted.

\item[\rm\textit{only if\/}:] Take $x\in G$ and a pair $(s,t)\in S\times T$. By transitivity there exists $y\in G$ such that
$$
    y\cdot(s,t)=(x\cdot s,t),
$$
i.e., $y\cdot s=x\cdot s$ and $y\cdot t=t$. Thus, $y^{-1}x\in G_s$ and\/ $y\in G_t$. Therefore, $x=y(y^{-1}x)\in G_tG_s$.  \end{description}
\end{proof}

\begin{probl}\label{permutation-character-equation}
    Let $G$ act on $S$, where both $G$ and $S$ are finite. For each element\/ $x\in G$, write
    $$
        \chi(x) = \abs{\set{s\in S \mid x\cdot s = s }}.
    $$
    The nonnegative-integer-valued function $\chi$ is called the \textsl{permutation character} associated with the action. Show that
    $$
        \sum_{x \in G}\chi(x) = \sum_{s\in S}\abs{G_s}= n|G|,
    $$
    where\/ $n$ is the number of orbits of\/ $G$ on\/ $S$.
\end{probl}

\begin{solution} For $x\in G$ put $S_x=\set{s\in S\mid x\cdot s=s}$ and define
\begin{align*}
    \tau\colon \bigcup_{x\in G}\set{x}\times S_x&\to\bigcup_{s\in S}\set{s}\times G_s\\
    (x,s) &\mapsto(s,x).
\end{align*}
Since $\tau$ is clearly a bijection and $\abs{S_x}=\chi(x)$, the first equality follows.

\medskip

For the second equality, by Lemma~\ref{stabilizer-conjugate} we have
$$
    t \in \mathcal O_s \implies |G_t|=|G_s|.
$$
Thus, after grouping the elements of $S$ by their orbits $\set{\mathcal O_{s_i}\mid 1\le i\le n}$, we obtain
$$
    \sum_{s\in S}|G_s|
        = \sum_{i=1}^n|\mathcal O_{s_i}||G_{s_i}|
        =\sum_{i=1}^n|G:G_{s_i}||G_{s_i}|
        = n|G|,
$$
where the second equality follows from the Fundamental Counting Principle Theorem~\ref{fundamental-counting-principle}.

\end{solution}

\begin{probl}\label{problem-1.A.7}
    Let $G$ be a finite group, and suppose that $H \varsubsetneq G$ is a proper subgroup. Show that the number of elements of\/ $G$ that do not lie in any conjugate of\/ $H$ is at least\/ $\abs H$.

    \textrm{\rm Hint. Let $\chi$ be the permutation character associated with the action of\/ $G$ on the left cosets of $H$. Then $\sum_{x \in G}\chi(x) = |G|$. Show that $\sum_{z \in H}\chi(z) > 2\abs H$. Use this information to get an estimate on the number of elements of\/ $G$ where~$\chi$ vanishes.}
\end{probl}

\begin{solution}
    The result is clear when $H\normal G$ because in that case $H$ is the only conjugate of $H$ and there are exactly $\abs G-\abs H$ elements that don't lie in $H$. Since $\abs G=\abs{G:H}\abs H$ and $H$ is proper, $\abs G\ge2\abs H$, i.e., $\abs G-\abs H\ge \abs H$.

    Let's now deal with the case where $H$ is not normal.
    
    [MSE, \citeyear{1522816}] To prove the hint we have to consider the action of $H$ on the very same set $S=\set{yH\mid y\in G}$ introduced above (i.e., same set, different group). Let
    $$
        {\cal O}^H_{yH} = \set{zyH\mid z\in H}
    $$
    denote the orbit of $yH$ under this action. Since $H\cap yH=\emptyset$ for $y\notin H$, it must be
    $$
        {\cal O}^H_{1H}\cap{\cal O}^H_{yH}=\emptyset.
    $$
    Therefore, the number $n$ of such orbits is at least $2$. Let $\chi^H\colon H\to\N_0$ be the permutation character associated to the action of $H$ on $S$. We have
    $$
        \chi^H(z) = |\set{yH\in S\mid zyH=yH}| = \chi(z),
    $$
    i.e., $\chi^H$ is nothing but the restriction $\chi|_H$ of $\chi$ to $H$. Therefore,
    $$
        \sum_{z\in H}\chi(z) =\sum_{z\in H}\chi^H(z) = n\abs H \ge 2\abs H,
    $$
    not quite the hint, but close. Then,
    \begin{align*}
        \abs G &=\sum_{z\in H}\chi(z) + \sum_{x\in\hat H\setminus H}\chi(x)\\
            &= n\abs H + |\hat H\setminus H|\\
            &=n\abs H + |\hat H| - \abs H    &&;\; H\subseteq\hat H,
    \end{align*}
    i.e., $\abs G- |\hat H| = (n-1)\abs H\ge \abs H$.
    
    Finally note that $H\normal G$ whenever equality is attained. Indeed, in such a case we would have $n=2$. Thus,  ${\cal O}^H_{xH}={\cal O}^H_{yH}$ for any pair $x,y\in G\setminus H$. This implies $xH=zyH$ for some $z\in H$, i.e., $z^{-1}x\in yH$, hence $Hx=yH$. In other words, every right coset of $H$ would be a left coset of $H$.
\end{solution}

\begin{probl}\label{cauchy}
    Let $G$ be a finite group, $n > 0$ an integer and\/ $\Z_n$ the additive group of integers modulo $n$. Let $S$ be the set of $n$-tuples $(x_1, x_2, ..., x_n)$ of elements of\/ $G$ such that $x_1x_2\cdots x_n = 1$.
    \begin{enumerate}[\rm a)]
        \item Show that $\Z_n$ acts on $S$ according to the formula
        $$
            k\cdot(x_1, x_2, \dots, x_n) = (x_{1+k}, x_{2+k}, \dots, x_{n+k}),
        $$
        where $k\in\Z_n$ and the subscripts are interpreted modulo $n$.

        \item Take $n = p$ where $p$ is a prime that divides $|G|$. Show that $p$ divides the number of\/ $\Z_p$-orbits of size $1$ on\/ $S$. Deduce that the number of elements of order\/~$p$ in\/~$G$ is congruent to $-1 \pmod{p}$.
    \end{enumerate}
    
    \textrm{\rm{\bf Note.} In particular, if a prime $p$ divides $\abs G$, then $G$ has at least one element of order $p$. This is a theorem of Cauchy, and the proof in this problem is due to J. H. McKay. Cauchy's theorem can also be derived as a corollary of Sylow's theorem. Alternatively, a proof of Sylow's theorem different from Wielandt's can be based on Cauchy's theorem. [cf.~Problem~\ref{sylow-e-2}]}
\end{probl}

\begin{solution}
\begin{enumerate}[\rm a)]
    \item Take $h,k\in\Z_n$. Then
    \begin{align*}
        (h+k)\cdot(x_1, x_2, \dots, x_n) &= (x_{1+h+k}, x_{2+h+k}, \dots, x_{n+h+k})\\
            &=h\cdot(x_{1+k}, x_{2+k}, \dots, x_{n+k})\\
            &=h\cdot k\cdot(x_1, x_2, \dots, x_n).
    \end{align*}

    \item Let's first recall that
    $$
        {\cal O}_{(x_1,\dots,x_p)} = \set{(x_{k+1},\dots,x_{k+p})\mid k\in\Z_p}.
    $$
    Since the additive group $\Z_p$ is cyclic generated by $1$, we have
    \begin{align*}
        |{\cal O}_{(x_1,\dots,x_p)}| = 1&\iff k\cdot(x_1,\dots,x_p) = (x_1,\dots,x_p)
                \textrm{ for }k\in\Z_p\\
            &\iff 1\cdot(x_1,\dots,x_p) = (x_1,\dots,x_p)\\
            &\iff(x_2, x_3,\dots,x_p,x_1) = (x_1,x_2,\dots,x_p)\\
            &\iff x_1=\cdots=x_p.
    \end{align*}
    Thus, the orbits of size $1$ are
    $$
        \set{{\cal O}_{(x,\dots,x)}\mid x\in G\textrm{ and }x^p=1}
        = \set{{\cal O}_{(x,\dots,x)}\mid x\in G,\;\ord(x)\mid p}.
    $$
    Hence, the number of orbits of size $1$ is
    $$
        d = |\set{x\in G\mid x^p=1}|.
    $$
    Now take an element $(x_1,\dots,x_p)$ and consider the stabilizer
    $$
        (\Z_p)_{(x_1,\dots,x_p)} = \set{k\in\Z_p\mid k\cdot(x_1,\dots,x_p)=(x_1,\dots,x_p)}.
    $$
    Then
    $$
        (\Z_p)_{(x_1,\dots,x_p)}=\begin{cases}
            \Z_p&{\rm if\ } x_1=\cdots=x_p,\\
            \set0&\rm otherwise.
        \end{cases} 
    $$
    To see this assume that $k\ne0$ is in the stabilizer. Take two indexes $i<j$ and let's show that $x_i=x_j$. Define $q\in\Z_p$ by $qk\equiv j-i\pmod p$. Given that~$k$ satisfies $x_\cdot= x_{\cdot+k}$, we have
    $$
        x_i=x_{i+k},\quad x_{i+k}=x_{i+2k},\quad\dots\quad,x_{i+(q-1)k}=x_{i+qk}=x_j.
    $$
    In consequence
    $$
        |(\Z_p)_{(x_1,\dots,x_p)}|=\begin{cases}
            p&{\rm if\ } x_1=\cdots=x_p=x,\; x^p=1,\\
            1&\rm otherwise.
        \end{cases} 
    $$
    By definition $|S|=\abs G^{p-1}$. Therefore
    $$
        \sum_{(x_1,\dots,x_p)\in S}|(\Z_p)_{(x_1,\dots,x_p)}| = dp + \abs G^{p-1}-d.
    $$
    Using Problem~\ref{permutation-character-equation} we get
    $$
        np=n\abs{\Z_p}=d(p-1) + \abs G^{p-1},
    $$
    which implies that $p\mid d$. Since
    $$
        d = |\set{x\in G\mid x^p=1}| = |\set1\cup\set{x\in G\mid \ord(x)=p}|,
    $$
    we deduce that $|\set{x\in G\mid \ord(x)=p}|=d-1\equiv-1\pmod p$.
    
\end{enumerate}
\end{solution}

\begin{probl}\label{problem-1.A.9}
    Suppose $|G| = pm$, where $p > m$ and $p$ is prime. Show that $G$ has a unique subgroup of order $p$.
\end{probl}

\begin{solution} By the previous problem, there exists at least an element $x\in G$ with $\ord(x)=p$. Thus, $\gen x$ is a subgroup of order $p$.

Let $H$ be another subgroup of order $p$. Then $\abs{G:H}=m$. By the $n!$~theorem~\ref{n!-theorem}, $H$ has a normal subgroup $N$ such that $\abs{G:N}\mid m!$. Then $\abs{G:N}\perp p$ and so the equation
$$
    pm=\abs N\abs{G:N}
$$
implies $p\mid\abs N$, i.e., $H=N$ is normal. In particular $\gen x$ is normal. By Corollary~\ref{product-by-normal} $H\gen x$ is a subgroup. Moreover,
$$
    \abs{H\gen x} = \frac{p^2}{\abs{H\cap\gen x}}.
$$
Since $p^2>\abs G$, it follows that $\abs{H\cap\gen x}=p$, i.e., $H=\gen x$.  \end{solution}

\begin{probl}\label{problem-1.A.10}
    Let $H\subseteq G$
    \begin{enumerate}[\rm a)]
        \item Show that $\abs{N_G(H):H}$ is equal to the number of left cosets of\/ $H$ in $G$ that are invariant under left multiplication by $H$.

        \item Suppose that $\abs H$ is a power of the prime $p$ and that $\abs{G:H}$ is divisible by~$p$. Show that $\abs{N_G(H):H}$ is divisible by\/ $p$.
    \end{enumerate}
\end{probl}

\begin{solution}
\begin{enumerate}[\rm a)]
    \item First recall that
    $$
        N_G(H) = \set{y\in G \mid H^y=H} = \set{y\mid yH=Hy}.
    $$
    Then,
    $$
        \set{yH\mid y\in N_G(H)} = \set{yH\mid yH=Hy} = \set{yH\mid HyH= yH}
    $$
    and the conclusion follows.

    \item We will prove something slightly stronger:
    $$
        \abs{G:H} \equiv \abs{N_G(H):H}\pmod p.
    $$
    Consider the action $H\times S\to S$, where $S=\lco GH$ is the set of left cosets of $H$ in $G$, defined by $z\cdot xH = zxH$. Given $x\in G$, we have
    \begin{align*}
        %S_z &= \set{xH \mid zxH=xH}\\
        %H_{xH} &= \set{z\in H\mid zxH= xH}\\
        {\cal O}_{xH} &= \set{zxH\mid z\in H}
    \end{align*}
    To compute the size of ${\cal O}_{xH}$ observe the following
    \begin{align*}
        \hat zxH = zxH &\iff \hat zx\in zxH\\
            &\iff z^{-1}\hat z\in H\cap H^x\\
            &\iff \hat z\in z\big(H\cap H^x\big).
    \end{align*}
    Then $\abs{{\cal O}_{xH}}=\abs H/|H\cap H^x|$. Put $N=N_G(H)$. Since $\abs H$ is a power of~$p$, the orders of its subgroups $H\cap H^x$ are also powers of $p$. More precisely,
    \begin{align*}
        \abs{{\cal O}_{xH}} &=
            \begin{cases}
                1   &x\in N,\\
                p^{k} &(k\ge1)\textrm{ otherwise}.
            \end{cases}\\[0.1in]
            &\equiv
           \begin{cases}
                1 &x\in N,\\
                0 &x\notin N.
            \end{cases}\pmod p
        \end{align*}
    Thus,
    \begin{align*}
        \abs{G:H} &= \abs S\\
                &= \sum_{xH\in S}\abs{{\cal O}_{xH}}\\
                &=\sum_{\set{yH\mid y\in N}}1
                    + \sum_{\set{xH\mid x\notin N}}\abs{{\cal O}_{xH}}\\
                &= \abs{N:H} + \sum_{\set{xH\mid x\notin N}}\abs{{\cal O}_{xH}}
    \end{align*}
    and the conclusion follows by taking congruence mod $p$.
\end{enumerate}
\end{solution}

\section{Automorphisms}

Recall that, given a group $G$, the conjugation epimorphism $\sigma\colon G\to\Inn(G)$ is defined by $\sigma(x)(\,\cdot\,)=(\,\cdot\,)^x$.

\begin{prop}
    With the preceding notations, we have
    $$
        \ker(\sigma)=Z(G)\quad\text{\rm and}\quad G/Z(G)\cong\Inn(G).
    $$
\end{prop}

\begin{proof} These are direct consequences of the definitions.  \end{proof}

\begin{cor}
    If $G/Z(G)$ is cyclic, then $G$ is abelian.
\end{cor}

\begin{proof} By the proposition there exists $x\in X$ such that $\Inn(G)=\gen{\sigma_x}$. Take $y\in G$. Then $\sigma_y=(\sigma_x)^n=\sigma_{x^n}$, i.e., $y=x^nz$ for some $z\in Z(G)$. Therefore, $x\leftrightarrow y$ because $x\leftrightarrow x^n$ and $x\leftrightarrow z$.  \end{proof}

\begin{rem} Let $G$ be a group and $H$ a subgroup. Then
\begin{align*}
    H\normal G &\iff \varsigma(H)=H\quad\text{\rm for all } \varsigma\in\Inn(G)\\
    H\ch G &\iff \theta(H)=H\quad\text{\rm for all } \theta\in\Aut(G).
\end{align*}
\end{rem}

\needspace{2\baselineskip}
\begin{prop}\label{Inn-normal-in-Aut}
    Let\/ $G$ be a group. The following hold true
    \begin{enumerate}[\rm a)]
        \item If\/ $\theta\in\Aut(G)$ and\/ $x\in G$, then\/ $\sigma_x^\theta=\sigma_{\theta(x)}$,
        \item $\Inn(G)\normal\Aut(G)$.
    \end{enumerate}
\end{prop}

\needspace{2\baselineskip}
\begin{proof}${}$
\begin{enumerate}[\rm a)]
    \item We have
    $$
        \sigma_x^\theta(\,\cdot\,) = \theta\circ\sigma_x(\theta^{-1}(\,\cdot\,))
            = \theta(\theta^{-1}(\,\cdot\,)^x) = (\,\cdot\,)^{\theta(x)}=\sigma_{\theta(x)}(\,\cdot\,),
    $$

    \item According to part a) $\sigma_x^\theta=\sigma_{\theta(x)}\in\Inn(G)$, as required.
\end{enumerate}
\end{proof}


\subsection{Exercises - Kurzweil \& Stellmacher - \S 1.3}

\begin{exr}
    Let\/ $N$ be characteristic in\/ $G$. The automorphisms\/ $\alpha$ of\/ $G$ satisfying\/ $\alpha|_N = 1$ form a normal subgroup of\/ $\Aut(G)$.
\end{exr}


\begin{solution} First of all note that $\alpha|_N$ can be seen, after coastriction, as the identity in $\Aut(N)$. More generally, the restriction/coastriction map
\begin{align*}
    \Aut(G)&\to\Aut(N)\\
    \sigma&\mapsto\sigma_N
\end{align*}
is well defined and its kernel is precisely the set of automorphisms $\alpha$ such that $\alpha|_N=1$.  \end{solution}

\begin{exr}
    The automorphisms\/ $\alpha$ of\/ $G$ satisfying\/ $\alpha(H) = H$ for all subgroups\/ $H$ of\/ $G$ form a normal subgroup of\/ $\Aut(G)$.
\end{exr}

\begin{solution} Such a set is a subgroup of $\Aut(G)$ because $\id_G$ is clearly one of them and the set is closed under composition. It is normal because, given $\sigma\in\Aut(G)$, we have
$$
    \alpha^\sigma(H)=\sigma\circ\alpha\circ\sigma^{-1}(H)
        =\sigma(\sigma^{-1}(H))=H.
$$
 \end{solution}

\begin{exr}
    Let\/ $\alpha \in \Aut(G)$ be such that\/ $|\set{x \in G \mid \alpha(x) = x}| > |G|/2$. Then\/ $\alpha = \id_G$.
\end{exr}

\begin{solution} Let $F$ be the set of elements that are fixed by $\alpha$. Since $\id_G\in F$ and $F$ is closed under the group operation, we see that $F\subgroup G$. It follows that $|G|=|G:F||F|>|G:F||G|/2$. Therefore, $|G:F|<2$.  \end{solution}

\begin{exr}
    The group\/ $G$ is abelian if, and only if, the mapping
    $$
        G \to G, \quad x \mapsto x^{-1} \quad (x \in G)
    $$
    is an automorphism of\/ $G$.
\end{exr}

\begin{solution} This is a direct consequence of the equation $(xy)^{-1}=y^{-1}x^{-1}$.  \end{solution}

\begin{exr}
    Let\/ $G$ be finite and\/ $\alpha \in \Aut(G)$ such that\/ $\alpha(x) \neq x = \alpha^2(x)$ for all\/ $x \in G^* = G\setminus\set1$. The following hold:
    \begin{enumerate}[\rm a)]
        \item For every\/ $x \in G$, there exists\/ $y \in G$ such that\/ $x = y^{-1}\alpha(y)$.
        \item $G$ is abelian of odd order.
    \end{enumerate}
\end{exr}

\begin{solution}
Define the relation `$\sim$' as $x_1\sim x_2$ iff $x_1=x_2$ or $\alpha(x_1)=x_2$. Then, `$\sim$' is an equivalence relation and the class of every $x$ in $G^*\!/{\sim}$ has exactly two elements, namely $x$ and $\alpha(x)$. It follows that $|G^*|=2|G^*\!/{\sim}|$, which is an even number, i.e., $|G|$ is odd.

\begin{enumerate}[\rm a)]
    \item Consider the function
    \begin{align*}
        \alpha^*\colon G^*&\to G^*\\
        y&\mapsto y^{-1}\alpha(y).
    \end{align*}
    Note that $\alpha^*$ in injective:
    \begin{align*}
        \alpha^*(y_1)=\alpha^*(y_2)
            &\implies y_1^{-1}\alpha(y_1)=y_2^{-1}\alpha(y_2)\\
            &\implies y_2y_1^{-1}=\alpha(y_2y_1^{-1})\\
            &\implies y_2y_1^{-1}=1.
    \end{align*}
    Since $G^*$ is finite, $\alpha^*$ is surjective, as desired.

    \item Take $x\in G$ and pick $y$ such that $x=y^{-1}\alpha(y)$. Then
    $$
        \alpha(x) = \alpha(y)^{-1}y=(y^{-1}\alpha(y))^{-1}=x^{-1},
    $$
    and $G$ is abelian by the previous exercise.
\end{enumerate}
\end{solution}

\begin{exr}
    Let\/ $N\normal G$ and\/ $L\subgroup G$ such that\/ $G = NL$. Then there exists an inclusion-preserving bijection from the set of subgroups\/ $X$ satisfying\/ $L\subgroup X\subgroup G$ to the set of\/ $L$-invariant subgroups\/ $Y$ satisfying\/ $L\cap N\subgroup Y\subgroup N$.
\end{exr}

\begin{solution} For every $L\subgroup X\subgroup G$, define $\theta(X)=X\cap N$. If $x\in L$, then
$$
    \theta(X)^x= (X\cap N)^x = X^x\cap N = X\cap N=\theta(X)
$$
because $x\in L\subseteq X$. It follows that $\theta(X)$ is $L$-invariant and $\theta$ is well-defined. Clearly, $\theta$ is inclusion-preserving.

To see that $\theta$ is injective it is enough to show that $L(X\cap N)=X$, i.e., that $X\subseteq L(X\cap N)$. Take $\xi\in X$. We can write it down as $\xi=xy$ with $x\in L$ and $y\in N$. Then $y=x^{-1}\xi\in LX=X$ and we see that $y\in X\cap N$. Therefore, $\xi\in L(X\cap N)$, as wanted.

For the surjectivity, take an $L$-invariant subgroup $Y$ satisfying $L\cap N\subgroup Y\subgroup N$ and define $X=YL$. We need to show that $X$ is a group and that $X\cap N=Y$. Take $\xi=yx\in YL$, with $y\in Y$ and $x\in L$. Then $\xi=x(x^{-1}yx)\in LY$, which shows that $X$ is a group. It remains to be seen that $LY\cap N\subseteq Y$ (the other inclusion is trivial). Take $\xi=xy\in N$ with $x\in L$ and $y\in Y$. Then $x=\xi y^{-1}\in NY=N$, hence $x\in L\cap N\subseteq Y$ and $\xi=xy\in Y$.  \end{solution}

\begin{exr}
    Let\/ $G$ be finite with\/ $Z(G) = \gen1$, and set\/ $\Gamma = \Aut(G)$ and\/ $\Lambda := \text{\rm Inn}(G)$.
    \begin{enumerate}[\rm a)]
        \item $C_\Gamma(\Lambda) = \gen{\id_G}$.
        \item Suppose that\/ $\Lambda$ is characteristic in\/ $\Gamma$, i.e., $\alpha(\Lambda)=\Lambda$ for all\/ $\alpha \in \Aut(\Gamma)$. Then\/ $\Aut(\Gamma) = \Inn(\Gamma)$.
        \item Suppose that\/ $G$ is simple. Then\/ $\Aut(\Gamma) = \Inn(\Gamma)$.
    \end{enumerate}
\end{exr}

\begin{solution}
\begin{enumerate}[\rm a)]
    \item By Proposition~\ref{Inn-normal-in-Aut}, given $x\in G$ and $\theta\in\Gamma$, we have 
    $$
        \sigma_x^\theta=\sigma_{\theta(x)}.
    $$
    If, in addition, $\theta\in C_\Gamma(\Lambda)$, then $\theta\leftrightarrow\sigma_x$ and $\sigma_{\theta(x)}=\sigma_x^\theta=\sigma_x$. Since $\sigma$ is mono because $\ker\sigma=Z(G)=\gen1$, it follows that $\theta(x)=x$, i.e., $\theta=\id_G$.

    \item From part a) applied to $\Gamma$, we know that
    $$
        Z(\Gamma)\subseteq C_\Gamma(\Lambda) = \gen{\id_G}.
    $$
    Therefore,
    \begin{equation}\label{eq-7.1}
        C_{\Aut(\Gamma)}(\Inn(\Gamma))=\gen{\id_\Gamma}.
    \end{equation}
    
    Fix $\alpha\in\Aut(\Gamma)$. Given $x\in G$, there exists $y\in G$ such that $\alpha(\sigma_x)=\sigma_y$. Since such an $y$ is determined by $x$, which in turn is determined by $\sigma_x$, this defines a map $\hat\alpha\colon x\mapsto y$. Moreover $\hat\alpha(1)=1$, because 
    $$
        \sigma_{\hat\alpha(1)}=\alpha(\sigma_1)=\alpha(\id_G)=\id_G=\sigma_1,
    $$
    and
    \begin{align*}
            \sigma_{\hat\alpha(x_1)\hat\alpha(x_2)} &= \sigma_{\hat\alpha(x_1)}\sigma_{\hat\alpha(x_2)}\\
            &= \alpha(\sigma_{x_1})\alpha(\sigma_{x_2})\\
            &= \alpha(\sigma_{x_1}\sigma_{x_2})\\
            & =\alpha(\sigma_{x_1x_2})\\
            &= \sigma_{\hat\alpha(x_1x_2)}.
    \end{align*}
    It follows that $\hat\alpha\in\Gamma$. Furthermore
    $$
        \alpha(\sigma_x)=\sigma_{\hat\alpha(x)}=\sigma_x^{\hat\alpha} = \mbf\sigma_{\hat\alpha}(\sigma_x),
    $$
    where $\mbf\sigma\colon\Gamma\to\Aut(\Gamma)$ is the conjugation morphism. Thus, $\alpha|_\Lambda=\mbf\sigma_{\hat\alpha}|_\Lambda$.

   \textbf{Claim 1:} \textit{$\eta\mapsto\mbf\sigma_\eta|_\Lambda$ is mono.}

    {\small\textsc{proof:} $\mbf\sigma_\eta|_\Lambda=\id_\Lambda\implies
        \eta\in C_\Gamma(\Lambda)=\gen{\id_G}$ by part~a)}
   
    \textbf{Claim 2:} \textit{The map\/ $\jmath\colon\Aut(\Gamma)\to\Gamma$ given by $\alpha\mapsto\hat\alpha$ is a morphism.}

    {\small\textsc{proof:} First note that $\id_{\Aut(\Gamma)}|_\Lambda=\mbf\sigma_{\id_G}|_\Lambda$. Hence, $\jmath(\id_{\Aut(\Gamma)})=\id_G$ by Claim~1. Second, $\mbf\sigma_{\hat\alpha_1}|_\Lambda\circ\mbf\sigma_{\hat\alpha_2}|_\Lambda=\mbf\sigma_{\hat\alpha_1\circ\hat\alpha_2}|_\Lambda$. Again, according to Claim~1, we get $\jmath(\alpha_1\circ\alpha_2)=\hat\alpha_1\circ\hat\alpha_2=\jmath(\alpha_1)\circ\jmath(\alpha_2)$.}

    \textbf{Claim 3:} \textit{$\jmath\circ\mbf\sigma=\id_\Gamma$}.

    {\small\textsc{proof:} Trivial. For all\/ $\theta\in\Gamma$, by definition we have\/ $\jmath(\mbf\sigma_\theta)=\theta$.}

    \textbf{Claim 4:} \textit{Let\/ $\Omega=\ker(\jmath)$. Then\/ $\Omega\cap\Inn(\Gamma)=\gen{\id_\Gamma}$.}
    
    {\small\textsc{proof:} Take $\omega\in\Omega\cap\Inn(\Gamma)$. Then $\omega=\mbf\sigma_\varphi$ for some $\varphi\in\Gamma$. From Claim~3 we get
    $$
        \id_\Gamma = \jmath(\omega) = \jmath(\mbf\sigma_\varphi)=\varphi,
    $$
    which implies $\omega=\mbf\sigma_{\id_\Gamma}=\id_\Gamma$.}

    \textbf{Conclusion:} Since both $\Omega$ and $\Inn(\Gamma)$ are normal in $\Aut(\Gamma)$, Claim~4 implies that $\Omega\leftrightarrow\Inn(\Gamma)$, i.e., $\Omega\subseteq C_{\Aut(\Gamma)}(\Inn(\Gamma))$, which is trivial by equation~$(\ref{eq-7.1})$. Thus, $\jmath$ is mono and the diagram
    $$
        \begin{tikzcd}
            \Inn(\Gamma)\arrow[r,"\iota",hook]
                &\Aut(\Gamma)\arrow[d,"\jmath"]\\
                &\Gamma\arrow[lu,"\cong"]\arrow[u,"\mbf\sigma", bend left=40]
        \end{tikzcd}
    $$
    commutes. As a result, the inclusion $\iota\colon\Inn(\Gamma)\to\Aut(\Gamma)$, is epi.
    
    \item The equation $\sigma_x^\theta=\sigma_{\theta(x)}$ shown in part~a) implies that $\Lambda\normal\Gamma$. In particular, if $\alpha\in\Aut(\Gamma)$, then $\alpha(\Lambda)\normal\Gamma$. If $G$ is simple, since $G\cong\Lambda$, we deduce that $\Lambda$ is simple too and the same goes for $\alpha(\Lambda)$. It follows that $\Lambda\cap\alpha(\Lambda)$ is trivial or $\Lambda$. In the latter case, $\Lambda=\alpha(\Lambda)$. In the former, since both subgroups are normal, we would have $\Lambda\leftrightarrow\alpha(\Lambda)$, i.e., $\alpha(\Lambda)\subseteq C_\Gamma(\Lambda)$, which is impossible by part~a). It follows that $\alpha(\Lambda)=\Lambda$. The conclusion follows from part~a).
\end{enumerate}
\end{solution}

\begin{exr}
    Let\/ $\text{GL}_2(\C)$ be the group of all invertible\/ $2\times2$-matrices over the field of complex numbers\/ $\C$, and let
    $$
        G := \Big\langle\begin{bmatrix}
            i   & \hphantom-0\\
            0 & -i
        \end{bmatrix},
        \begin{bmatrix}
            \hphantom-0 & 1\\
            -1  &0
        \end{bmatrix}\Big\rangle\subgroup \text{GL}_2(\C).
    $$
    The group\/ $G$ is called a \textsl{quaternion group} (of order\/ $8$).
    \begin{enumerate}[\rm a)]
        \item $|G| = 8$.
        \item $|Z(G)| = 2$.
        \item Every element of\/ $G \setminus Z(G)$ has order\/ $4$.
        \item $G$ contains exactly one element of order\/ $2$.
        \item Every subgroup of\/ $G$ is normal in\/ $G$.
        \item $G$ possesses an automorphism of order\/ $3$.
    \end{enumerate}
\end{exr}

\begin{solution} Let $\mbf i$ be the first generator of $G$ and $\mbf j$ the second. Then $G=\gen{\mbf i,\mbf j}$. Define
$$
    \mbf k= \mbf i\mbf j = \begin{bmatrix}
        0   &i\\
        i   &0
    \end{bmatrix}.
$$
Finally note that $\mbf i^2=-\id$, $\mbf j^2=-\id$ and $\mbf k^2=-\id$.

\begin{enumerate}[\rm a)]
    \item The set
    $$
        \set{\mbf i, \mbf j, \mbf k}
    $$
    produces the following multiplication table
    $$
        \begin{tabular}{|c||c|c|c|}
            \hline
            $\cdot$ & $\mbf i$ & $\mbf j$ & $\mbf k$ \\
            \hline
            \hline
            $\mbf i$ & $-\id$ & $\mbf k$ & $-\mbf j$\\
            \hline
            $\mbf j$ & $-\mbf k$ & $-\id$ & $\mbf i$\\
            \hline
            $\mbf k$ & $\mbf j$ & $-\mbf i$ & $-\id$\\
            \hline
        \end{tabular}
    $$
    which shows that
    $$
        G=\set{\pm\id, \pm\mbf i, \pm\mbf j, \pm\mbf k}.
    $$

    \item From the table above we see that $Z(G)=\set{\id,-\id}$.

    \item As the table shows, every element $q\in G\setminus Z(G)$ satisfies $q^2=-\id$.

    \item The only element of order $2$ is $-\id$.

    \item Let $H\subgroup G$. In the table we see that the product of two elements $q$ and $w$ satisfies $qw=\pm wq$. Therefore $q^w=\pm q$.

    \item From what we just saw, every conjugation is trivial or has order~$2$. Therefore, to find an automorphism of order $3$, we must discard them. Consider the rotation
    \begin{align*}
        \rho\colon G&\to G\\
        \pm\id&\mapsto\pm\id\\
        \pm\mbf i&\mapsto\pm\mbf j\\
        \pm\mbf j&\mapsto\pm\mbf k\\
        \pm\mbf k&\mapsto\pm\mbf i
    \end{align*}
    This is a morphism because $\mbf i\mbf j=\mbf k$, $\mbf j\mbf k=\mbf i$ and $\mbf k\mbf i=\mbf j$, and so
    \begin{align*}
        \rho(\mbf i\mbf j)=\rho(\mbf k)=\mbf i=\mbf j\mbf k=\rho(\mbf i)\rho(\mbf j)\\
        \rho(\mbf j\mbf k)=\rho(\mbf i)=\mbf j=\mbf k\mbf i=\rho(\mbf j)\rho(\mbf k)\\
        \rho(\mbf k\mbf i)=\rho(\mbf j)=\mbf k=\mbf i\mbf j=\rho(\mbf k)\rho(\mbf i).
    \end{align*}
\end{enumerate}
\end{solution}

\subsection{Exercises - Kurzweil \& Stellmacher - \S 1.4}

\begin{exr}
    Suppose that\/ $A\subgroup N\normal G$ and\/ $N$ is cyclic. Then\/ $A\normal G$.
\end{exr}

\begin{solution} It is enough to show that $A\ch N$. To see this take $\theta\in\Aut(N)$. Then $|\theta(A)|=|A|$, which implies $\theta(A)=A$ by Corollary~\ref{cyclic-subgroups}.  \end{solution}

\begin{exr}
    Let\/ $p$ and\/ $q$ be primes, and let\/ $G$ be a cyclic group of order\/ $pq$. Then, $G$ contains more than three subgroups if, and only if, $p \neq q$.
\end{exr}

\begin{solution} This is a consequence of Corollary~\ref{cyclic-subgroups}. If $p\ne q$, then $G$ has four subgroups, namely $\gen1$ and $G$, one of order $p$ and another of order $q$. If $p=q$, it only has three: $\gen1$, $G$ and one of order $p$.  \end{solution}

\begin{exr}
    Let\/ $G$ be a finite group. Suppose that\/ $|\set{x\in G \mid x^n = 1}| \le n$ for all\/ $n\in\N$. Then\/ $G$ is cyclic.
\end{exr}

\begin{solution} Take $a\in G$. Put $r=\ord(a)$. Given $x\in G$ with $\ord(x)\mid r$, we have
$$
    r+1-|\gen a\cap\set{x}|
        = |\gen a|+|\set{x}|-|\gen a\cap\set{x}|
        = |\gen a\cup\set{x}|\le r,
$$
which implies $x\in\gen a$. In consequence, (1)~every cyclic subgroup is normal (let $a^x$ play the role of $x$) and (2)~there is only one cyclic subgroup of any given order (take $x$ with $\ord(x)=r$).

Let $\xi$ have the maximum order, say $r$. Take $\zeta\in G$. Put $s=\ord(\zeta)$ and $d=\gcd(r,s)$. Therefore $\chi=\zeta^d$ has order $\frac sd\perp r$. Then $\gen\xi\leftrightarrow\gen\chi$ because they intersect trivially and, according to property~(1), both are normal. It follows that $\xi\chi$ has order $r\frac sd$. In consequence $r\frac sd\le r$, which means that $s=d\mid r$. Then, by property~(2) we get $\gen\zeta\subgroup\gen\xi$, which implies $\zeta\in\gen\xi$.  \end{solution}

\begin{exr}
    Let\/ $G$ be finite. Suppose that all maximal subgroups of\/ $G$ are conjugate. Then\/ $G$ is cyclic.
\end{exr}

\begin{solution} By Exercise~\ref{exercise-1.2.10} it is enough to show that $G$ contains only one maximal subgroup. Pick a maximal $M$. According to Exercise~\ref{exercise-1.1.12} we can pick an element
$$
    \xi\in G\setminus\bigcup_{x\in G}M^x.
$$
If $\gen\xi\varsubsetneq G$, there would be a maximal containing $\gen\xi$, which is impossible because that maximal would be one of the conjugates of $M$.  \end{solution}

\subsection{Exercises - Kurzweil \& Stellmacher - \S 1.5}

\begin{exr}\label{exercise-1.5.1}
    Let\/ $A$ be an abelian normal subgroup of\/ $G$ and\/ $x \in G$.
    \begin{enumerate}[\rm a)]
        \item The mapping\/ $\ct_x\colon A\to A$ given by\/ $a \mapsto [x, a]$ is a morphism.
        \item $[\gen x, A] = \set{[x, a] \mid a \in A}$.
    \end{enumerate}
\end{exr}

\begin{solution} Note that $\ct_x(A)\subseteq A$ because $[G,A]\subseteq A$ by Proposition~\ref{commutator-props}~c).

\begin{enumerate}[\rm a)]
    \item Take $a,b\in A$. Define $\ct_x(a)=[x,a]$. We have
    \begin{align*}
        \ct_x(ab) &= [x,ab]\\
            &= [x,a][x,b]^a &&\textrm{; Prop.~\ref{commutator-props}~b)}\\
            &= [x,a][x,b]   &&\textrm{; }a\leftrightarrow[x,b]\in A\\
            &= \ct_x(a)\ct_x(b).
    \end{align*}

    \item The RHS is clearly included in the LHS. Moreover, the RHS is a group because it is the image of $\ct_x$. Therefore, to prove the other inclusion it suffices to show that $[x^i,a]$ is an element of $\im(\ct_x)$ for every $i\in\Z$. But
    \begin{align*}
        [x,[x^i,a]] &= x[x^i,a]x^{-1}[x^i,a]^{-1}\\
            &= x^{i+1}ax^{-i}a^{-1}x^{-1}[x^i,a]^{-1}\\
            &= x^{i+1}ax^{-(i+1)}xa^{-1}x^{-1}[x^i,a]^{-1}\\
            &= x^{i+1}ax^{-(i+1)}a^{-1}axa^{-1}x^{-1}[x^i,a]^{-1}\\
            &= [x^{i+1},a][x,a]^{-1}[x^i,a]^{-1},
    \end{align*}
    i.e., $[x^{i+1},a]=[x,[x^i,a]][x^i,a][x,a]$. Inductively we get $[x^i,a]\in\im(\ct_x)$ for $i\ge0$. For $i<0$, take $j=n|G|+i$, where $n$ is such that $j\ge0$. Since $x^i=x^j$, we have $[x^i,a]=[x^j,a]\in\im(\ct_x)$.
\end{enumerate}
\end{solution}

\begin{exr}
    Let\/ $A$ and\/ $x$ be as in\/ \textrm{\rm Exercise~\ref{exercise-1.5.1}}. Suppose that for all\/ $a\in A$ we have $G = C_G(xa)A$. Then\/ $[G,A]=[\gen x,A]$.

    \textrm{\rm Hint [l.c.]: Given $y\in G$ and $a\in A$, write $y=zb$ with $z\leftrightarrow x$ and then $z=wc^{-1}$ with $w\leftrightarrow xa$.}
\end{exr}

\begin{solution} {[See this \href{https://math.stackexchange.com/a/1484227/269050}{MSE} answer]}

In what follows we make use of parts a) and b) of Proposition~\ref{commutator-props}.

Take $y\in G$ and $a\in A$. First note that for any $\beta\in A$, it is
\begin{equation}\label{eq6.2}
    [z\beta,a] = [z,a]
\end{equation}
because $[z\beta,a]=[\beta,a]^z[z,a]$ and $[\beta,a]=1$.

Write $y=zb$ with $b\in A$ and $z\leftrightarrow x$. Then,
\begin{align*}
    [y,a] &= [zb,a]\\
        &= [z,a]    &&;\ (\ref{eq6.2}).
\end{align*}
Now write $z=wc^{-1}$, where $c\in A$ and $w\leftrightarrow xa$. Then
\begin{align*}
    1 &= [w,xa]\\
        &= [w,x][w,a]^x\\
        &= [zc,x][zc,a]^x\\
        &= [zc,x][z,a]^x    &&;\ (\ref{eq6.2}),
\end{align*}
i.e., $[z,a]^x=[x,zc]$. Thus, for $d=c^{x^{-1}}\in A$, we get
\begin{align*}
    [z,a] &= [x,zc]^{x^{-1}}\\
        &= [x,zd]\\
        &= [x,z][x,d]^z\\
        &= [x,d]^z  &&;\ x\leftrightarrow z\\
        &= [x,d^z] \in [\gen x,A]   &&;\ A\normal G.
\end{align*}
 \end{solution}

\begin{exr}\label{exercise-1.5.3}
    Let\/ $|G| = p^n$, $p$ a prime, and let\/ $|G:C_G(x)|\le p$ for all\/ $x\in G$.
    \begin{enumerate}[\rm a)]
      \item $C_G(x) \normal G$ for all\/ $x\in G$.
      \item $G' \subseteq Z(G)$.
      \item \textrm{\rm [Knoche]} $|G'|\le p$.
    \end{enumerate}
\end{exr}

\begin{solution}
\begin{enumerate}[\rm a)]
    \item Let's start with 
    
    \textbf{Claim 1:} \textit{If\/ $z\notin C_G(x)$, then\/ $z^p\in Z(G)$.}
    
    First observe that $x\notin Z(G)$. Consider the set
    $$
        X = \set{z^iC_G(x)\mid 0\le i<p}.
    $$
    Clearly $|X|\le p$. Suppose that $|X|<p$. Then, there exist $0\le j<i<p$ such that $z^iC_G(x)=z^jC_G(x)$. Hence, $z^{i-j}\in C_G(x)$. But $i-j\perp\ord(z)$ because $\ord(z)$ is a power of $p$. Then $z\in\gen{z^{i-j}}\subseteq C_G(x)$; contradiction, i.e., $|X|=p$. Since, by hypothesis, there are exactly $p$ coclasses of $C_G(x)$, we deduce that $z^pC_G(x)\in X$. This means that $z^{p-i}\in C_G(x)$ for some $0\le i<p$ and the same argument as above shows that $i=0$. In particular, $x\leftrightarrow z^p$. Taking into account that $C_G(z^p)\supseteq C_G(z)$ and that both subgroups have orders at least $p^{n-1}$, it must happen that $z^p\in Z(G)$, as claimed.


    \textbf{Claim 2:} \textit{If\/ $z\in G$, then $z^p\in Z(G)$.}

    If $z\in Z(G)$, clearly $z^p\in Z(G)$. If $z\notin Z(G)$, there exists $x\in G$ such that $z\notin C_G(x)$ and so $z^p\in Z(G)$ by Claim~1.

    \needspace{2\baselineskip}
    \textbf{Claim 3:} \textit{If\/ $z\notin C_G(x)$, then\/ $C_G(x)\cap\gen z = \gen{z^p}$.}
    
    Indeed. Every subgroup of $\gen z$ is cyclic generated by $z^{p^d}$ for some integer $d\le n$. And given that, by Claim~2, $z^p\in Z(G)\subseteq C_G(x)$, we must have $d=1$.
    
    \textbf{Claim 4:} \textit{If\/ $z\notin C_G(x)$, then\/ $C_G(x)\gen z=G$.}
    
    This is a direct consequence of Claim 3 because.
    $$
        |G| \ge |C_G(x)\gen z|
            = \frac{p^{n-1}\ord(z)}{|C_G(x)\cap\gen z|}
            = p^{n-1}\frac{\ord(z)}{\ord(z^p)} = |G|,
    $$
    where the last equality derives from the equation $\ord(z^p)=\ord(z)/p$.

    \textbf{Conclusion:} $C_G(x)\normal G$.

    Suppose that $C_G(x)\ne C_G(x)^w$ for some $w\in G$. Pick $z\in C_G(x)^w\setminus C_G(x)$.  Claim~4 implies that $C_G(x)\gen z=G$. In particular $C_G(x)C_G(x)^w=G$. Therefore, $C_G(x)=G$ by Exercise~\ref{exercise-1.1.5}.

    \item Given $x\in G$, Part a) allows us to consider the quotient $G/C_G(x)$, whose order is $1$ or $p$. In either case, $G/C_G(x)$ is abelian. By Lemma~\ref{center-characterization},
    $$
        G' \subseteq\bigcap_{x\in G}C_G(x) = Z(G).
    $$

    \item Given $x\in G$ the commutator map $\ct_x\colon G\to G'$ is a morphism because
    \begin{align*}
        \ct_x(ab) &= [x,ab]\\
            &= [x,a][x,b]^a  &&\text{; Prop.~\ref{commutator-props}}\\
            &= [x,a][x,b]   &&\text{; }G'\subseteq Z(G)\\
            &= \ct_x(a)\ct_x(b).
    \end{align*}
    It follows that, given $[x,a]\in G'$, we have
    $$
        [x,a]^p = \ct_x(a)^p = \ct_x(a^p)=[x,a^p]=1
    $$
    because, according to Claim~2, $a^p\in Z(G)$. Thus, the elements of $G'$ have order $1$ or~$p$.

    Take $[a,b]\in G'$. If $a\in C_G(b)$ then $[a,b]=1\in\im(\ct_x)$. If $a\notin C_G(b)$, then $G=C_G(b)\gen a$ and we can write $x=za^i$ for some $z\leftrightarrow b$. Therefore, given $j\in\Z$, by Proposition~\ref{commutator-props}, we have
    $$
        \ct_x(b^j) = [x,b^j] = [za^i,b^j] = [z,b^j][a^i,b^j]= [a,b^{ij}].
    $$
    If $p\perp i$, taking $j$ such that $ij=1\pmod p$, we get
    $$
        [a,b]=[a,b^{ij}]=\ct_x(b^j)\in\im(\ct_x).
    $$
    If $p\mid i$, then $a^i\in Z(G)$ and $x=za^i\leftrightarrow b$. If $x\nleftrightarrow a$, what we just proved shows that $[b,a]\in\im(\ct_x)$, which implies $[a,b]$ is in the image too because the image is a subgroup. Thus, the remaining case is when $a,b\in C_G(x)$.
    
    Put $H=C_G(x)$. We just showed that $G'\setminus H'\subseteq\im(\ct_x)$. Consider the astriction $h_x\colon G/H\to G'$. Given that $|G:H|=p$ and $h_x$ is mono, we have $|{\im(\ct_x)}|=|{\im(h_x)}|=p$. In particular, $|(G' \setminus H')\cup\set1|\le p$, i.e.,
    $$
        |G'\setminus H'|\le p-1.
    $$

    \textbf{Claim 5:} \textit{The subgroup\/ $H$ is in the conditions of\/ $G$ for $n-1$. In other words, $|H|=p^{n-1}$ and\/ $|H:C_H(y)|\le p$ for all\/ $y\in H$.}

    To see this first observe that $C_H(y)=C_G(x)\cap C_G(y)$. Thus, if $C_H(y)\ne H$, from Claim~4 we deduce that
    $$
        p^n=|G|=|C_G(x)C_G(y)|=\frac{p^{2n-2}}{|C_H(y)|},
    $$
    which proves the claim.

    By induction on $n$, Claim~5 implies that $|H'|\le p$. Therefore,
    $$
        |G'|=|G'\setminus H'|+|H'|\le p-1 + p = 2p-1< 2p.
    $$
    The conclusion follows because $|G'|$ is a power of $p$.
\end{enumerate}
\end{solution}

\begin{exr}
    Let\/ $\alpha \in \Aut(G)$. Suppose that\/ $x^{-1}\alpha(x) \in Z(G)$ for all\/ $x\in G$. Then\/ $\alpha(x)=x$ for all\/ $x\in G'$.
\end{exr}

\begin{solution} For $i=1,2$, take $x_i\in G$ and put $\alpha(x_i)=z_ix_i$ for $z_i\in Z(G)$. According to Proposition~\ref{commutator-props}, we have
\begin{align*}
    \alpha[x_1,x_2] &= [\alpha(x_1),\alpha(x_2)]\\
        &= [z_1x_1,z_2x_2]\\
        &= [x_1,z_2x_2]^{z_1}[z_1,z_2x_2]\\
        &= [x_1,z_2x_2]\\
        &= [x_1,z_2][x_1,x_2]^{z_2}\\
        &= [x_1,x_2].
\end{align*}
 \end{solution}

\begin{exr} {\rm [Ito]}
    Let\/ $G = AB$, where\/ $A$ and\/ $B$ are abelian subgroups of\/ $G$. Then\/ $G'$ is abelian.

    \textrm{\rm Hint [l.c.]:} $[A,B]$ is abelian and $G'=[A,B]$.
\end{exr}

\begin{solution} {[Brought from \href{https://math.stackexchange.com/q/3488256}{MSE}]} Take $x,x_1\in A$ and $y,y_1\in B$. Write $x^{y_1}=ab$ and $y^{x_1}=b_1a_1$, with $a,a_1\in A$ and $b,b_1\in B$. Using Proposition~\ref{commutator-props}, we get
\begin{align*}
    [x,y]^{x_1y_1} &= [x^{y_1},y]^{x_1}\\
        &= [ab,y]^{x_1}\\
        &= ([b,y]^a[a,y])^{x_1}\\
        &= [a,y]^{x_1}    &&;\ b\leftrightarrow y\\
        &= [a,y^{x_1}]\\
        &= [a,b_1a_1]\\
        &= [a,b_1][a,a_1]^{b_1}\\
        &= [a,b_1]  &&;\ a\leftrightarrow a_1\\
    \intertext{Similarly,}
    [x,y]^{y_1x_1} &= [x,y^{x_1}]^{y_1}\\
        &= [x,b_1a_1]^{y_1}\\
        &= [x,b_1][x,a_1]^{y_1}\\
        &= [x^{y_1},b_1]  &&;\ x\leftrightarrow a_1\\
        &= [ab,b_1]\\
        &= [b,b_1]^a[a,b_1]\\
        &= [a,b_1].  &&\; b\leftrightarrow b_1
\end{align*}
Thus, $[x,y]^{x_1y_1}=[x,y]^{y_1x_1}$, i.e., $x_1y_1[x,y](x_1y_1)^{-1}=y_1x_1[x,y]x_1^{-1}y_1^{-1}$, or
$$
    [x_1^{-1},y_1^{-1}][x,y]=[x,y][x_1^{-1},y_1^{-1}].
$$
As a result, $[A,B]$ is abelian. By Proposition~\ref{commutator-props}, $[A,B]\normal G$. Consider $\bar G=G/[A,B]$. Given $a\in A$ and $b\in B$, we have $ab=[a,b]ba$, i.e., $\bar a\bar b=\bar b\bar a$, which shows that $\bar G$ is abelian. By the lemma of Exercise~\ref{exercise-1.5.3}, $G'\subseteq[A,B]$. Since $[A,B]$ is clearly included in $G'$, we deduce that $G'=[A,B]$.  \end{solution}

\begin{exr} \textrm{\rm[Burnside]}
    Let\/ $N$ be a normal subgroup of\/ $G$. Suppose that every element in\/ $G\setminus N$ has order\/~$3$. Then\/ $[A,A^x]=\gen1$ for all abelian subgroups\/ $A\subgroup N$ and\/ $x\in G\setminus N$.
\end{exr}

\begin{solution} {[Brought from \href{https://math.stackexchange.com/a/1485048/269050}{MSE}]} Fix $A\subgroup N$ and $x\in G\setminus N$. First observe that the result is trivial if $A\normal G$ because, in that case, $A^x\subseteq A$.

Back to the general case, take $b,c\in A$ and $x\in G\setminus N$. Since $b\in A\subseteq N$, we know that $bx\notin N$ and so its order is $3$. Then,
\begin{align*}
    bc^x &= bxcx^{-1}\\
        &= x^{-1}b^{-1}x^{-1}b^{-1}cx^{-1}  &&;\ bx=(bx)^{-2}\\
        &= x^{-1}b^{-1}xxb^{-1}cx^{-1}      &&;\ x^{-1}=x^2\\
        &= (b^{-1})^{x^{-1}}(b^{-1}c)^x.
\end{align*}
In particular, for $c=1$, we get $b=(b^{-1})^{x^{-1}}(b^{-1})^x$. Interchanging $b$ with $b^{-1}$, we get
$$
    b^{-1} = b^{x^{-1}}b^x.
$$
Take $a\in A$. Since $ax\notin N$, we can apply the last equation to it and obtain,
$$
    b^{-1} = b^{x^{-1}a^{-1}}b^{ax} = b^{x^{-1}}(b^x)^a.
$$
Combining both expressions for $b^{-1}$ we see that
$$
    (b^x)^a=b^x,
$$
i.e. $a\leftrightarrow b^x$.  \end{solution}

\section{Sylow Groups}

\begin{defns}
    A \textsl{$p$-group} is a finite group whose order is a power of a prime number $p$. A \textsl{Sylow $p$-subgroup} of a finite group $G$ is a $p$-subgroup $P$ whose order is as large as allowed by Lagrange's theorem, that is, $\abs P=p^e$ where $e > 0$ and $p\perp m$, where $\abs G = p^em$. The\/ \textrm{\rm Sylow-E Theorem~\ref{sylow-e}} states that Sylow subgroups always exist.
\end{defns}

\begin{lem}
    Let $p$ be a prime. Then $p\mid\binom pi$ for $0<i<p$.
\end{lem}

\begin{proof} Consider the identity
$$
    (p-i+1)\binom p{i-1}=\binom pii.
$$
Then reason by induction on $i$.  \end{proof}

\begin{prop}
    Let $p$ be a prime and $e\ge0$ and $m\ge1$ two integers. Then
    $$
        \binom{p^em}{p^e} \equiv m\pmod p.
    $$
\end{prop}

\begin{proof} Let $f(x)\in\Z_p[x]$ be a polynomial. The lemma implies that
$$
    (1+f(x))^p = 1 + f(x)^p.
$$
By induction on $e$,
$$
    (1+x)^{p^e}=\big((1+x)^{p^{e-1}}\big)^p = (1 + x^{p^{e-1}}\big)^p = 1 + x^{p^e}.
$$
It follows that
$$
    (1+x)^{p^em} = \big((1+x)^{p^e}\big)^m = \big(1 + x^{p^e}\big)^m.
$$
By looking at the coefficient of degree $p^e$, we deduce
$$
    \binom{p^em}{p^e} = \binom m1=m\qquad\textrm{in }\Z_p,
$$
as desired.  \end{proof}

\begin{thm}\label{sylow-e} {\rm[Sylow-E]}
     Let\/ $G$ be a finite group, and let\/ $p$ be a prime. Then there is a Sylow\/ $p$-subgroup such that its order is\/ $p^e$ for some non-negative integer\/ $e$.
\end{thm}

\begin{proof} Put $\abs G=p^em$ with $p\perp m$ and let $S\subseteq 2^G$ be the set of parts with cardinality~$p^e$. Note that
$$
    \abs S=\binom{p^em}{p^e} \equiv m \not\equiv 0\pmod p.
$$
Let $G$ act on $S$ by left product. Since the orbits of elements in $S$ partition $S$, the incongruence above implies the existence of some set $X\in S$ whose orbit satisfies $\abs{{\cal O}_X}\perp p$. By the Fundamental Counting Principle Teorem~\ref{fundamental-counting-principle}, this cardinality equals $\abs{G:G_X}=\abs G/\abs{G_X}$, where
$$
    G_X = \set{y\in G\mid yX=X}.
$$
It follows that $p^e\mid\abs{G_X}$ (actually, $e$ is the exponent of $p$ in $\abs{G_X}$). Take $x\in X$. Then $G_Xx\subseteq X$. In particular, $\abs{G_X}\le\abs X$. Thus,
$$
    p^e\mid\abs{G_X} \le\abs X= p^e
$$
and the proof is complete.  \end{proof}

\bigskip

The set of Sylow $p$-subgroups of a finite group $G$, denoted $\Syl_p(G)$, is nonempty for all primes $p$. The \textsl{$p$-core} of $G$ is defined as the intersection of all Sylow $p$-subgroups and denoted $O_p(G)$. This is a subgroup that plays an important role in finite group theory. Characteristic subgroups, such as the center $Z(G)$, the derived (or commutator) subgroup $G'$, and $O_p(G)$, are mapped onto themselves by all automorphisms of $G$, and are thus automatically normal. It is also true that characteristic subgroups are normal in an even more general context.

\begin{cor}\label{elementwise-pi}
    Let $H$ be a subgroup of a finite group $G$ and $\pi$ a set of primes. If\/ $\spec\ord(y)\subseteq\pi$, for all $y\in H$, then $\spec\abs H\subseteq\pi$.
\end{cor}

\begin{proof} Suppose by contradiction that $\spec\abs H\not\subseteq\pi$. Pick $p\in\spec\abs H\setminus\pi$. By Cauchy's Theorem {\rm [cf.~Problem~\ref{cauchy}]}, there exists $y\in H$ with $\ord(y)=p$.  \end{proof}


\subsection{Problems B}
\begin{probl}\label{problem-1.B.1}
    Let $P \in\Syl_p(G)$, where $G$ is a finite group.
    \begin{enumerate}[\rm a)]
        \item Let $Q \subseteq G$ be a $p$-subgroup. Show that $QP$ is a subgroup if, and only if, $Q \subseteq P$. In particular, $Q\subseteq N_G(P)\implies Q\subseteq P$.
        \item If\/ $P\normal G$, show that\/ $\Syl_p(G) = \set P$, and deduce that\/ $P\ch G$.
    \end{enumerate}
\end{probl}

\begin{solution}
\begin{enumerate}[\rm a)]
    \item Put $\abs P=p^e$ and $\abs Q=p^r$. By Lemma \ref{HK-cardinality},
    $$
        \abs{QP} = p^rp^e/\abs{Q\cap P}= p^{e+r-s},
    $$
    where $p^s=\abs{Q\cap P}$. It is now clear that equivalence follows directly from the definitions.

    The consequence is trivial because $Q\subseteq N_G(P)\implies QP=PQ$ [cf.~Remark~\ref{product-is-subgroup-condition}].

    \item Let $Q$ be another element of\/ $\Syl_p(G)$. Then $Q\subseteq G=N_G(P)$, and by part~a) we get $Q\subseteq P$. Equality is attained because both subgroups have the same order. Hence $P=O_p(G)$ is characteristic.
\end{enumerate}
\end{solution}

\begin{probl}\label{problem-1.B.2}
    Show that $O_p(G)$ is the unique largest normal\/ $p$-subgroup of\/~$G$. (This means that it is a normal\/ $p$-subgroup of\/ $G$ that contains every other normal\/ $p$-subgroup of\/ $G$.)
\end{probl}

\begin{solution} It is clear that $O_p(G)$ is a normal $p$-subgroup. Let $Q$ be a normal $p$-subgroup of $G$. Take $P\in\Syl_p(G)$. By Corollary~\ref{product-by-normal} $PQ$ is a subgroup. The preceding problem then implies $Q\subseteq P$. Since $P$ was arbitrarily chosen, it follows that $Q\subseteq O_p(G)$.  \end{solution}

\begin{probl}\label{problem-1.B.3}
    Let $P \in\Syl_p(G)$ and $N = N_G(P)$. Show that $N = N_G(N)$.

    \textrm{\rm{\bf Note.} For a generalization of this result see Problem~\ref{problem-1.C.1}.}
\end{probl}

\begin{solution} Let $\abs G=p^em$, where $p\perp m$ and $\abs P=p^e$. Since $p^e\mid\abs N\mid p^em$, we deduce that $P\in\Syl_p(N)$. Then $\Syl_p(N)=\set P$ by Problem~\ref{problem-1.B.1}.

Now take $z\in N_G(N)$. Then $P^z\in\Syl_p(N^z)$ and given that $N^z=N$, we get $P^z\in\Syl_p(N)$. Moreover, as $P\normal N$ and $z$-conjugation is an isomorphism, we get $P^z\normal N^z=N$. By Problem~\ref{problem-1.B.1} it follows that $\Syl_p(N)=\set{P^z}$. Thus, $P^z=P$ and $z\in N_G(P)=N$.

\end{solution}

\begin{probl}\label{sylow-e-2}
    Let\/ $P\subseteq G$ be a $p$-subgroup such that $\abs{G:P}$ is divisible by\/ $p$. Using Cauchy's theorem, but without appealing to Sylow's theorem, show that there exists a subgroup\/ $Q$ of\/ $G$ containing\/ $P$, and such that\/ $\abs{Q:P}=p$. Deduce that a maximal $p$-subgroup of\/ $G$ (which obviously must exist) must be a Sylow $p$-subgroup of\/~$G$.

    \textrm{\rm Hint. Use Problem~\ref{problem-1.A.10} and consider the group $N_G(P)/P$.}
\end{probl}

    \textrm{{\bf Note:} The solution to this problem is another proof of Theorem~\ref{sylow-e}.}
    
\begin{solution} Given that $P\normal N_G(P)$, we know that $N_G(P)/P$ is a group. Moreover, its order $\abs{N_G(P):P}$ is divisible by $p$ by Problem~\ref{problem-1.A.10}. Let $\bar y$ be an element of $N_G(P)/P$ with $\ord(\bar y)=p$. If $\varphi\colon N_G(P)\to N_G(P)/P$ is the projection onto the quotient, and $Q=\varphi^{-1}\gen{\bar y}$, Corollary~\ref{index-of-preimage} implies that
$$
    |N_G(P):Q|=|N_G(P)/P:\gen{\bar y}|
$$
which implies that $|Q:P|=\ord(\bar y)=p$. Then
$$
    |Q|=|Q:P||P|=p|P|,
$$
i.e., $Q$ is a $p$-subgroup of order greater than $|P|$. Thus, any maximal subgroup among the $p$-subgroups of $G$ must be a Sylow $p$-subgroup.

\end{solution}


\begin{probl}\label{problem-1.B.5}
    Let $\pi$ be any set of prime numbers. We say that a finite group\/ $H$ is a \textsl{$\pi$-group} if\/ $\spec\abs H\subseteq\pi$. Also, a\/ $\pi$-subgroup $H \subseteq G$ is a \textsl{Hall $\pi$-subgroup} of\/ $G$ if\/ $\abs{G:H}\perp\pi$ (so if\/ $\pi = \set p$, a Hall\/ $\pi$-subgroup is exactly a Sylow $p$-subgroup). Now let $\varphi\colon G \to K$ be an epimorphism of finite groups. 
    \begin{enumerate}[\rm a)]
        \item If\/ $H$ is a Hall\/ $\pi$-subgroup of\/ $G$, prove that $\varphi(H)$ is a Hall\/ $\pi$-subgroup of\/~$K$.

        \item Show that every Sylow $p$-subgroup of $K$ has the form $\varphi(H)$, where $H$ is some Sylow $p$-subgroup of\/ $G$. 
        
        \item Show that $|\Syl_p(G)|\ge|\Syl_p(K)|$ for every prime $p$.
    \end{enumerate}

    \textrm{\rm\textbf{Note.} If the set $\pi$ contains more than one prime number, then a Hall $\pi$-subgroup can fail to exist. But a theorem of\/ P.~Hall, after whom these subgroups are named, asserts that in the case where $G$ is solvable, Hall $\pi$-subgroups always do exist (see the Hall-E Theorem~\ref{hall-e}) We mention also that part b) of this problem would not remain true if ``Sylow $p$-subgroup'' were replaced by ``Hall $\pi$-subgroup''.}
\end{probl}

\begin{solution}
\begin{enumerate}[\rm a)]
    \item Begin by taking $p\in\spec\abs{\varphi(H)}$. Given that $\varphi$ is onto, there exists $y\in H$ with $\ord(\varphi(y))=p$. Then $m=\ord(y)\mid\abs H$. Since $\varphi(y)^m=1$, we must have $p\mid m\mid\abs H$. Then $p\in\pi$.

    
    Now take $q\in\spec\abs{K:\varphi(H)}$. Using Corollary~\ref{index-of-preimage} we obtain
    \begin{align*}
        \abs{G:H} &=|G:\varphi^{-1}(\varphi(H))||\varphi^{-1}(\varphi(H)):H|\\
            &= \abs{K:\varphi(H)}|\varphi^{-1}(\varphi(H)):H|,
    \end{align*}
    which implies $q\mid\abs{G:H}$. Therefore, $q\notin\pi$.
    
    \item Let $Q\subseteq K$ be a Sylow $p$-subgroup. Write $\abs K=p^em$ with $p\perp m$ and $\abs Q=p^e$. By Corollary~\ref{index-of-preimage} we know
    $$
        |G:\ker\varphi|=p^em\quad{\rm and}\quad
            |G:\varphi^{-1}(Q)| = \abs{K:Q}= m\perp p.
    $$
    Then $\Syl_p(\varphi^{-1}(Q))\subseteq\Syl_p(G)$. Pick a Sylow $p$-subgroup $P$ of $\varphi^{-1}(Q)$. By part a) applied to $\pi=\set p$, we know that $\varphi(P)$ is a Sylow $p$-subgroup of $K$. And given that $\varphi(P)\subseteq Q$, equality must be attained.
    
    \item This is a direct consequence of part b).
\end{enumerate}
\end{solution}

\begin{probl}\label{problem-1.B.6}
    Let $G$ be a finite group and\/ $K \subseteq G$ a subgroup. Suppose that\/ $H \subseteq G$ is a Hall\/ $\pi$-subgroup, where $\pi$ is some set of primes. Show that if\/~$KH$ is a subgroup, then $K \cap H$ is a Hall\/ $\pi$-subgroup of\/~$K$.

    \textrm{\rm\textbf{Note.} In particular, $K$ has a Hall $\pi$-subgroup, namely $K\cap H$, if either $H$ or $K$ is normal in $G$ since in that case, $KH$ is guaranteed to be a subgroup.}
\end{probl}

\begin{solution} Take a prime $p$ such that $p\mid\abs{K\cap H}$. Since $\abs{K\cap H}\mid\abs H$, then $p\mid\abs H$ and therefore $p\in\pi$.

Now take $q$ prime, $q\mid\abs{K:K\cap H}=\abs{KH:H}$. Since
$$
    \abs{G:H}=\abs{G:KH}\abs{KH:H},
$$
we deduce that $q\mid\abs{G:H}$ and so $q\notin\pi$.  \end{solution}

\begin{probl}\label{problem-1.D.7} Let $G$ be a finite group and $\pi$ any set of primes.
    \begin{enumerate}[\rm a)]
    \item Show that $G$ has a (necessarily unique) normal\/ $\pi$-subgroup\/ $N$ such that $N \supseteq M$ whenever $M\normal G$ is a $\pi$-subgroup.

    \item Show that the subgroup\/ $N$ of part\/~{\rm a)} is contained in every Hall\/~$\pi$-subgroup of\/~$G$.

    \item Assuming that $G$ has a Hall\/ $\pi$-subgroup, show that $N$ is exactly the intersection of all of the Hall\/ $\pi$-subgroups of\/~$G$.
    \end{enumerate}

    \textrm{\rm {\bf Note.} The subgroup $N$ of this problem is denoted $O_\pi(G)$. Because of the uniqueness in a), it follows that this subgroup is characteristic in $G$. If $p$ is a prime number, then $O_{\set p}(G) = O_p(G)$ [cf.~Problem~\ref{problem-1.B.2}].}
\end{probl}

\begin{solution}
\begin{enumerate}[\rm a)]
    \item For every\/ $p\in\pi$, let $O_p$ be the $p$-core of $G$. Put $F=\prod_{p\in\pi}O_p$. Since every $O_p$ is normal, $F$ is normal. Moreover, given that the intersection of two $p$-groups with coprime orders is $\gen1$, we see that
    $$
        \abs F= \prod_{p\in\pi}\abs{O_p}.
    $$
    If $q\mid\abs F$ is a prime factor, then $q\mid\abs{O_p}$ for some $p\in\pi$, i.e., $q\in\pi$, which shows that $F$ is a $\pi$-subgroup. 

    
    This allows us to invoke the existence of a maximal normal $\pi$-subgroup~$N$ (in general, \textit{maximal\/} refers to \textit{proper\/} subgroups, however, in this particular case, we extend the term to include the whole group). Now consider any other normal $\pi$-subgroup $M$. By Corollary~\ref{product-by-normal}, $NM$ is a group. It is also normal and a $\pi$-subgroup because
    $$
        \abs{NM}\abs{N\cap M} = \abs N\abs M.
    $$
    Given that $N$ is maximal, we must have $N=NM$. Thus, $M\subseteq N$.

    \item Let $H$ be a Hall $\pi$-subgroup. Since $N$ is normal, $HN$ is a group. By Problem~\ref{problem-1.B.6}, $H\cap N$ is Hall in $N$.

    Given that $N$ is a $\pi$-subgroup and $|N:H\cap N|\perp\pi$, we deduce that $|N:H\cap N|\perp|N|$. Therefore, from
    \begin{align*}
        \abs N&=\abs{H\cap N}\abs{N:H\cap N}
    \end{align*}
    we deduce that $\abs{N:H\cap N}=1$, i.e., $N=H\cap N\subseteq H$.
    
    \item From part b) we know that $N$ is included in the intersection of all Hall $\pi$-subgroups.

    For the other inclusion let $\Hall_\pi(G)$ denote the family of all Hall $\pi$-subgroups of $G$. Given $x\in G$ and $H\in\Hall_\pi(G)$, the conjugate $H^x$ is also in $\Hall_\pi(G)$, i.e., $\Hall_\pi(G)$ is closed under conjugation. In consequence,
    $$
        \big(\bigcap{\Hall_\pi(G)}\big)^x
            =\big(\bigcap_{H\in\Hall_\pi(G)}H\big)^x
            \subseteq \bigcap_{H\in\Hall_\pi(G)}H^x 
            = \bigcap{\Hall_\pi(G)},
    $$
    and the intersection is normal. Since the intersection is a subgroup of~$H$ for any $H\in{\Hall_\pi(G)}$, it is also a $\pi$-subgroup. The conclusion now follows from part~a).
\end{enumerate}
\end{solution}

\begin{probl}\label{problem-1.B.8}
    Let $G$ be a finite group and $\pi$ any set of primes.
    \begin{enumerate}[\rm a)]
    \item Show that\/ $G$ has a (necessarily unique) normal subgroup\/ $N$ such that\/ $G/N$ is a $\pi$-group and $M\supseteq N$ whenever $M\normal G$ and $G/M$ is a $\pi$-group.
    \item Show that the subgroup\/ $N$ of part\/ {\rm a)} is generated by the set of all elements of\/~$G$ that have order not divisible by any prime in $\pi$.
    \end{enumerate}

    \textrm{\rm\textbf{Note.} The characteristic subgroup $N$ of this problem is denoted $O^\pi(G)$.}
    
    \textrm{\rm {\bf Note.} For an MSE solution see~\href{https://math.stackexchange.com/q/688412/269050}{\it A problem about $\pi$-groups}.}
\end{probl}

\begin{solution}
\begin{enumerate}[\rm a)]
    \item Define $\hat\pi=\spec\abs G\setminus\pi$. By the previous problem, $N={\cal O}_{\hat\pi}(G)$ is a normal $\hat\pi$-subgroup such that $N\supseteq M$ whenever $M\normal G$ is a $\hat\pi$-subgroup.

    \begin{enumerate}[$\to\!\!|$]
        \item Let's show that $\bar G=G/N$ is a $\pi$-group. By the previous problem, $O_{\hat\pi}(\bar G)$ is the maximal normal $\hat\pi$-subgroup of $\bar G$. From the diagram:
        $$
            \begin{tikzcd}[column sep=small]
                1\arrow[r]&\tilde N\arrow[r]
                    &G\vphantom/\arrow[rr,"\varphi",bend left=40]\arrow[r]
                    &\bar G\arrow[r]
                    &\bar G/O_{\hat\pi}(\bar G)\arrow[r]
                    &1,
            \end{tikzcd}
        $$
        where $\tilde N=\ker(\varphi)$, we deduce that
        $$
            |G/\tilde N|=|\bar G/O_{\hat\pi}(\bar G)|,
        $$
        i.e.,
        $$
            |\tilde N|=\abs N|O_{\hat\pi}(\bar G)|
        $$
        and so $\spec|\tilde N|\subseteq\hat\pi$. But $\tilde N$ is normal, $\tilde N\supseteq N$ and $N$ maximal with these properties. It follows that $N=\tilde N$ and therefore $O_{\hat\pi}(\bar G)=\gen1$, i.e., $O_p(\bar G)=\gen1$ for all $p\in\hat\pi$. So \dots??? $\to\!\!|$

        \item Suppose that $\bar G=G/N$ is not a $\pi$-group. Then, there exists $p\in\hat\pi$ such that $p\mid|\bar G|$. 

        \item Suppose that $\bar G=G/N$ is not a $\pi$-group. Then, there exists $p\in\hat\pi$ and $z\in G$ such that $\ord(\bar z)=p$, where $\bar z$ is the class of $z\in\bar G$. Clearly $z^p\in N$. Consider the core
        $$
            \Core_G\gen z = \bigcap_{x\in G}\gen{z^x}.
        $$
        According to Theorem \ref{largest-normal-subgroup}, $\Core_G\gen z$ is normal. Moreover, since every element $y$ of the core can be written as $(z^n)^x$ for some $n\in\Z$ and $x\in G$, we see that $p\mid n$. Since $z^p\in N$, it follows that $y\in N^x=N$. Then $N\supseteq\Core_G\gen z$. Then
        $$
            \Core_G\gen z=\Core_G\gen z\cap N=\Core_G(\gen z\cap N)=\Core_G\gen{z^p},
        $$
        because $\gen z\cap N=\gen{z^p}$. Then
        $$
            \abs{N\gen z}=\frac{\abs N\abs{\gen z}}{|\gen{z^p}|}
                =\abs{N}|\gen z:\gen{z^p}|=\abs Np.
        $$
        In particular, $\spec\ord(z)=pr$ with $\spec r\subseteq\hat\pi$.
    
        From
        \begin{align*}
            |N\gen z^x| &=|Nx^{-1}\gen z|\\
                &=\abs N\abs{\gen z}/\abs{N\cap\gen z} &&\textrm{; Prob.~\ref{HwK}}\\
                &=\abs N\abs{\gen z}/|\gen z^p| &&\textrm{; }N\cap\gen z=\gen{z^p}\\
                &=\abs Np,
        \end{align*}
        we deduce that $N\gen z^x$ is a $\hat\pi$-group. Therefore, the core $\Core_G(N\gen z)$ is a normal $\hat\pi$-group. It follows that $\Core_G(N\gen z)\subseteq N$, i.e.,
        $$
            \Core_G(N\gen z)=N.
        $$
        So \dots??? $\to\!\!|$

        \item Suppose that $\bar G=G/N$ is not a $\pi$-group. We can write $|\bar G|=p^em$, with $p\in\hat\pi$, $e>0$ and $p\perp m$. Let $w\in G$ be such that its class $\bar w$ in $\bar G$ has order $\ord(\bar w)=p^dn$ for some $d>0$ and $p\perp n$. Then $z=w^n$ satisfies $\ord(\bar z)=p^d$. It follows that $\ord(z)=p^cr$ with $c\ge d$ and $p\perp r$. Thus $\ord(z^r)=p^c$ and $\ord(\bar z^r)=p^d$. So, after replacing $z$ with $z^r$ we may assume that $\ord(z)=p^c$ with $c\ge d$. In addition, $N\cap\gen z=\gen{z^{p^d}}$. Consider the quotient $N\gen z\to\gen z/N\cap\gen z$ [cf.~Proposition~\ref{prod-quotient}]. It produces a diagram
        $$
            \begin{tikzcd}
                1\arrow[d]\\
                P\arrow[rd]\arrow[d]\\
                N\gen z\arrow[r]&\gen z/\gen{z^{p^d}}\arrow[rd]\\
                &&1,
            \end{tikzcd}
        $$
        where $P\in\Syl_p(N\gen z)$ exists by Problem~\ref{problem-1.B.5} because $\gen z/\gen{z^{p^d}}$ is a Sylow $p$-group.
        
        Given $x\in P$, its class satisfies $\bar x=\bar z^k$ for some $k<p^d$, which implies that $x\in\gen z$. Thus, $P\subseteq\gen z$, i.e., $\gen z=P\in\Syl_p(N\gen z)$. It follows that $p\perp\abs N$ (and so $O_p(G)=\gen1$). Then $N\cap\gen z=\gen1$, i.e., $\ord(z)=p^d$.
        
        
        So \dots??? $\to\!\!|$
        
        \item[\checkmark] Let $M\normal G$ be such that $G/M$ is a $\pi$-group. Then
        $$
            \spec\abs{G:M}\subseteq\pi.
        $$
        Take $z\in N$ and consider the class $\bar z$ of $z$ in $G/M$. To prove that $z\in M$ we must show that $\bar z=1$. Let $m=\ord(\bar z)$ and suppose that $m>1$. Since $G/M$ is a $\pi$-group, $\spec m\subseteq\pi$. Pick $p\in\spec m$ and put $y=z^{m/p}$. If $\bar y$ is the class of $y$ in $G/M$, then $\ord(\bar y) = p\in\pi$. Let $r=\ord(y)$. Since $y\in N$, $\spec r\subseteq\hat\pi$. But $\bar y^r=1$ and so we should have $p\mid r$, which is impossible.
        
        In conclusion,
        $$
            N \subseteq \bigcap{\cal M},
        $$
        where $\cal M$ is the collection of all normal subgroups $M$ such that $G/M$ is a $\pi$-group.

        It remains to be seen that $G/N$ is a $\pi$-group. However, this might not be true or, at least, easy to check (see all the attempts that failed above). Moreover, there is no apparent reason for equality to be attained in the last inclusion. Therefore, we must change our mind and redefine $N$ as the intersection on the RHS, as shown in the MSE response we mentioned in the preceding Note.
        
        Since every $M\in\cal M$ is normal, we know that $N$ is normal. To see that $G/N$ is a $\pi$-group, suppose the contrary and pick $x\in G$ with $\ord(\bar x)\in\hat\pi$, where $\bar x$ is the class of $x$ in $G/N$. Take $M\in\cal M$ and denote by $\tilde x$ the class of $x$ in $G/M$. Consider the exact diagram
        $$
            \begin{tikzcd}
                1\arrow[r]&M\arrow[r]\arrow[d]&G\arrow[r]\arrow[d]
                    &G/M\arrow[r] \arrow[d,equal,]&1\\
                1\arrow[r]&M/N\arrow[r]&G/N\arrow[r]&G/M\arrow[r]&1
            \end{tikzcd}
        $$
        Since $\bar x\mapsto\tilde x$, $\spec\ord(\tilde x)\subseteq\spec\ord(\bar x)\subseteq\hat\pi$. And since $G/M$ is a $\pi$-group, the spec must also be included in $\pi$. But $\pi\cap\hat\pi=\emptyset$ and so $\bar x\in M/N$. It follows that $x\in M$. Given that $M$ was arbitrarily chosen, we deduce that $x\in N$, which is a contradiction. Now the conclusion of part~a) is clear.
        
    \end{enumerate}
    
    \item Introduce
    $$
        \hat N=\gen{x\in G\mid\spec\ord(x)\subseteq\hat\pi},
    $$
    which is characteristic because orders are invariant under automorphisms.
    
    \begin{description}[leftmargin=!,labelwidth=\widthof{$\hat N$:}]
    \item[$\hat N\supseteq N$:]
        By part a) it is enough to show that $G/\hat N$ is a $\pi$-group. For the sake of contradiction suppose that it isn't. Take $\bar z\in G/\hat N$ such that $\ord(\bar z)=p\in\hat\pi$. Pick a representative $z$ of $\bar z$. Write $\ord(z)=p^em$ with $p\perp m$. Note that $e>0$, i.e., $z^m\notin\hat N$. It follows that
        $$
            \emptyset\ne\spec\ord(z^m)\cap\pi=\spec p^e\cap\pi
                =\set p\cap\pi=\emptyset,
        $$
        a contradiction.
      \item[$\hat N\subseteq N$:] It is enough to show that $\spec\ord(x)\subseteq\hat\pi\implies x\in M$ for all $M\in\cal M$. Take $M\in\cal M$ and suppose that $x\notin M$. Then the class $\bar x$ of $x\in G/M$ satisfies $\spec\ord(\bar x)\subseteq\pi$. But this is impossible because
      $$
        \spec\ord(\bar x)\subseteq\spec\ord(x)\subseteq\hat\pi,
      $$
      which would imply $\spec\ord(\bar x)\subseteq\pi\cap\hat\pi=\emptyset$.
  \end{description}
\end{enumerate}
\end{solution}


\section{Sylow Theorems}

\begin{thm}\label{sylow-pre-d}
    Let $Q$ be a $p$-subgroup of a finite group $G$, and suppose that $P \in\Syl_p(G)$. Then $Q \subseteq P^\omega$ for some element $\omega\in G$.
\end{thm}

\begin{proof} Write $\abs G=p^em$ with $p$ prime, $p\perp m$. Then $\abs P=p^e$ and $\abs{G:P}=m$. Let $Q$ act on ${\cal L}=\set{xP\mid x\in G}$ by left product. Since $\abs{\cal L}=m$ and $\cal L$ can be partitioned as a disjoint union of orbits, there must be an orbit ${\cal O}_{\omega P}$ with $\abs{{\cal O}_{\omega P}}\perp p$. By the Fundamental Counting Principle Theorem~\ref{fundamental-counting-principle}, $\abs{{\cal O}_{\omega P}}=\abs{Q:Q_{\omega P}}$. But $\abs{Q:Q_{\omega P}}\mid\abs Q$ and $\abs Q$ is a power of $p$. Therefore, we must have $\abs{{\cal O}_{\omega P}}=1$, i.e.,
$$
    z\omega P=\omega P\qquad(z\in Q).
$$
Thus, $z\in P^\omega$ for all $z\in Q$, as desired.  \end{proof}

\begin{thm}\label{sylow-c} {\rm[Sylow-C]}
    If\/ $P$ and\/ $Q$ are Sylow $p$-subgroups of a finite group $G$, then $Q = P^\omega$ for some element $\omega\in G$.
\end{thm}

\begin{proof} This is actually a corollary of the previous theorem because, in this case, $\abs Q=\abs P$.  \end{proof}

\begin{rem}
    The theorem provides an alternative proof of part\/ {\rm b)} of {\rm Problem \ref{problem-1.B.1}}.
\end{rem}


\begin{cor}
    Let $P\in\Syl_p(G)$. Then $O_p(G)$ is the kernel of the action of\/ $G$ on the left-cosets of\/~$P$. In particular, if $\abs G=p^em$ with $p\perp m$ and $\abs{O_p(G)}=p^d$, then $p^{e-d}\mid(m-1)!$.
\end{cor}

\begin{proof} Let $\lco GP$ denote the set of left-cosets of $P$ and $\sigma\colon G\to\Sym(\lco GP)$ the group morphism induced by the action. We have an exact sequence
$$
    1\to \ker(\sigma)\to G\to\Sym(\lco GP),
$$
where
\begin{align*}
    \ker(\sigma) &= \set{y\in G\mid \sigma(y)=\id}\\
        &= \set{y\in G\mid yxP=xP,\; x\in G}\\
        &= \set{y\in G\mid y\in P^x,\; x\in G}\\
        &= \bigcap_{x\in G}P^x\\
        &=O_p(G).
\end{align*}
In particular,
$$
    p^em/p^d=\abs G/\abs{\ker(\sigma)}\mid\abs{G:P}!=m!,
$$
i.e., $p^{e-d}\mid(m-1)!$  \end{proof}

\begin{rem}
    The reader should understand the previous proof under the light of\/ {\rm Lemma~\ref{core}}, {\rm Definition~\ref{Core_G}}, {\rm Remark~\ref{G-to-Sym(X)}} and\/ {\rm Lemma~\ref{core=ker}}.
\end{rem}

\begin{cor}\label{frattini-argument} {\rm[Frattini Argument]}
    Let $N\normal G$, where $N$ is finite, and suppose that $P\in\Syl_p(N)$. Then $G=N_G(P)N$.
\end{cor}

\begin{proof} Take $x\in G$. Since $P\subseteq N$, $P^x\subseteq N^x=N$. Taking also into account that $\abs{P^x}=\abs P$, we deduce that $P^x\in\Syl_p(N)$. By the theorem, there exists $y\in N$ such that $P^x=P^y$. In particular, $xy^{-1}\in N_G(P)$, i.e., $x\in N_G(P)N$.  \end{proof}

\begin{rem}\label{frattini-argument-2}
    The\/ \textrm{\rm Frattini Argument} admits a more general statement:

    Let\/ $G$ be a group acting on a set\/ $X$. Suppose that\/ $N\normal G$ acts transitively on\/ $X$. Then\/ $G = G_\alpha N$ for every\/ $\alpha\in X$. In particular, if\/ $N_\alpha = \gen1$, then\/ $G_\alpha$ is a \textsl{complement} of\/ $N$ in\/ $G$, i.e., $G=NG_\alpha$ and\/ $N\cap G_\alpha=\gen1$.

    \textrm{\small\rm Take $x\in G$ and $\alpha\in X$. By the transitivity of $N$ there exists $y\in N$ with $x\cdot\alpha=y\cdot\alpha$. In consequence, $y^{-1}x\in G_\alpha$, i.e., $x\in NG_\alpha$.}
\end{rem}

\begin{thm}\label{sylow-d} {\rm[Sylow-D]}
    Let $P$ be a $p$-subgroup of a finite group $G$. Then $P$ is contained in some Sylow $p$-subgroup of\/~$G$.
\end{thm}

\begin{proof} This is a direct corollary of Theorem~\ref{sylow-pre-d}.  \end{proof}

\begin{cor}
    Let\/ $G$ be a finite group and\/ $p$ a prime. Then, every subgroup generated by elements whose orders are powers of\/ $p$ is a $p$-group if, and only if, $G$ has only one Sylow $p$-group.
\end{cor}
\needspace{2\baselineskip}
\begin{proof}
\begin{description}
    \item[\textrm{\rm{\it if\/} part:}] Suppose that $X$ is a set whose elements have orders that are powers of $p$. If $P$ is the unique element of $\Syl_p(G)$, then $X\subseteq P$. Therefore, $\gen X\subseteq P$, which shows that $\gen X$ is a $p$-group. 
    \item[\it only if\/\rm:] Consider the set $X$ consisting of elements in $G$ whose order is a power of $p$. Take $P\in\Syl_p(G)$. Since every element in $P$ has an order that divides $|P|$, we have $P \subseteq X$. Consequently, $P \subseteq \gen X$, which is a $p$-group. As $P$ has the largest possible order among $p$-groups, equality must be attained, i.e., $P=\gen X$, which implies the uniqueness of~$P$.
\end{description}
\end{proof}


\begin{ntn}
    If\/ $G$ is a group, the number of Sylow $p$-groups of\/ $G$ is denoted by $n_p(G)$, i.e.,
    $$
        n_p(G) = |\Syl_p(G)|.
    $$
\end{ntn}

\begin{cor}\label{n_p-is-index}
    If\/ $G$ is finite and $P\in\Syl_p(G)$, then $n_p(G) = |G:N_G(P)|$.
\end{cor}

\begin{proof} This is a direct consequence of Theorem~\ref{sylow-c} and Corollary~\ref{conjugate-count}.  \end{proof}

\begin{thm}\label{n_p(G)=1}
    Suppose that\/ $G$ is a finite group and that $P$ and\/ $Q$ are two distinct Sylow $p$-subgroups such that $|P \cap Q|$ is maximum among the pairs of Sylow $p$-subgroups. Then $n_p(G) \equiv 1 \pmod{|P:P \cap Q|}$.
\end{thm}

\begin{proof} Let $P$ act on $\Syl_p(G)$ by conjugation. Since $P\in\Syl_p(G)$, one of the orbits is $\set P$, which has size~$1$. Since $n_p(G)$ is the sum of the orders of all orbits, to prove the theorem it suffices to show that $\abs{P:P\cap Q}\mid\abs{{\cal O}_R}$, for all $R\ne P$. By the Fundamental Counting Principle,
$$
    |{\cal O}_R|=\abs{P:P_R},
$$
where $P_R=\set{y\in P\mid R^y=R}\subseteq N_G(R)$. By Problem~\ref{problem-1.B.1}, $P_R\subseteq R$. Since $P_R\subseteq P$, we get $P_R\subseteq P\cap R$. Then
$$
    |{\cal O}_R| = \abs{P:P_R} \ge \abs{P:P\cap R} \ge \abs{P:P\cap Q}
$$
because $P\ne R$ and $\abs{P\cap Q}$ is maximum. Now, both $|{\cal O}_R|$ and $\abs{P:P\cap Q}$ are powers of\/~$p$. Therefore, $\abs{P:P\cap Q}\mid|{\cal O}_R|$, as wanted.  \end{proof}

\begin{cor}\label{p|n_p-1}
    If\/ $G$ is finite and $p$ is a prime, then $n_p(G) \equiv 1 \pmod{p}$.
\end{cor}

\begin{proof} If $n_p(G)=1$, there is nothing to prove. Otherwise there exist $P\ne Q$ and the conditions required by the theorem are fulfilled. Therefore, to conclude the proof it is enough to show that $\abs{P:P\cap Q}\ne1$. But given that $\abs P=\abs Q$, we know that $\abs{P:P\cap Q}=\abs{Q:P\cap Q}$, and this number can be $1$ only if $P=Q$, which isn't.   \end{proof}

\begin{rem}\label{p'-part}
    It follows from {\rm Corollary \ref{n_p-is-index}} that
    $$
        n_p(G)|N_G(P):P|=\abs{G:P},
    $$
    which implies that $n_p(G)\mid\abs{G:P}$. Therefore, if\/ $\abs G=p^em$ with $p\perp m$, then $n_p(G)\mid m$.

    In general, $m=\abs{G:P}$ is known as the \textsl{$p'$-part} of\/~$G$.
\end{rem}


\begin{xmpl}
    Let $G$ be a group of order $2^67^3$ and let $n_7=n_7(G)$. Consider the following facts, where `$\,\leftarrow$' stands for \emph{because}:
    \begin{enumerate}[$1$.]
        \item $n_7\mid2^6$ $\leftarrow$ {\rm Remark \ref{p'-part}}.
        \item $n_7\equiv1\pmod7$ $\leftarrow$ {\rm Corollary \ref{p|n_p-1}}.
        \item $n_7\in\set{1, 2^3, 2^6}$ $\leftarrow$ $(1, 2, 2^2, 2^3, 2^4, 2^5, 2^6)\equiv(1, 2, 4, 1, 2, 4, 1)\pmod7$.
        \item $n_7>1\implies\exists P\ne Q\in\Syl_7\colon |P:P\cap Q|=7$ $\leftarrow$ $2^3,2^6\not\equiv1\pmod{7^2}$.
    \end{enumerate}
        Now assume\/ $n_7\ne1$ and let\/ $P$ and\/ $Q$ be as in part $4$. Also put $H=N_G(P\cap Q)$.
    \begin{enumerate}
        \item[$5$.] $P\cap Q\normal P$ $\leftarrow$ {\rm Problem~\ref{problem-1.A.1} for $G=P$}.
        \item[$6$.] $P\cap Q\normal Q$ $\leftarrow$ {\rm Problem~\ref{problem-1.A.1} for $G=Q$}.
        \item[$7$.] $P\cup Q\subseteq H$ $\leftarrow$ {\rm $5$ and $6$}.
        \item[$8$.] $n_7(H)>1$ $\leftarrow$ $P,Q\in\Syl_7(H)$.
        \item[$9$.] $n_7(H)\ge8$ $\leftarrow$ {\rm $8$ and Corollary \ref{p|n_p-1}}.
        \item[$10$.] $2^3\mid\abs H$ $\leftarrow$ {\rm $\spec\abs H=\set 2$ and $9$}. 
        \item[$11$.] $\abs{G:H}\le2^3$ $\leftarrow$ {\rm $7^3=\abs P\mid\abs H$ and $10$}.
        \item[$12$.] $\abs G\nmid\abs{G:H}!$ $\leftarrow$ {\rm $7!<\abs G$ and $\abs G\nmid8!$}.
        \item[$13$.] $H\ne G\implies G$ is not simple $\leftarrow$ {\rm $n!$~theorem~\ref{n!-theorem} and $12$}.
    \end{enumerate}
    \textsc{Claim: }$G$ is not simple.
    \begin{enumerate}
        \item[$14$.] If\/ $n_7=1$ then\/ $\Syl_7(G)=\set P$ with $P$ normal $\leftarrow$ $P=O_7(G)$.
        \item[$15$.] If\/ $n_7\ne1$, $G$ is not simple $\leftarrow$ {\rm $13$ or $\gen1\varsubsetneq P\cap Q\normal G$}.
    \end{enumerate}
    In the case where $P\cap Q\normal G$, given $R\in\Syl_7(G)$, pick $\omega\in G$ such that $R=P^\omega$. Then
    $$
        R=P^\omega\supseteq P^\omega\cap Q^\omega=(P\cap Q)^\omega=P\cap Q.
    $$
    In particular, taking into account that $\abs{P\cap Q}$ is maximum, the intersection of any pair of distinct Sylow $7$-subgroups would equal $P\cap Q$. Note that this also implies that $P\cap Q=O_7(G)$.
\end{xmpl}

\begin{thm}
    Let $G$ be a finite group and $p$ a prime. Choose $P, Q \in \Syl_p(G)$ such that $H = P\cap Q$ is minimal in the set of intersections of two Sylow $p$-subgroups of\/ $G$. Then $O_p(G)$ is the unique largest subgroup of $H$ that is normal in both $P$ and $Q$.
\end{thm}

\begin{proof} Let $K\subseteq H$ be normal in $P$ and $Q$. Put $N=N_G(K)$. Given $R\in\Syl_p(G)$ we have to prove that $K\subseteq R$. From $P,Q\subseteq N$ we infer $P,Q\in\Syl_p(N)$. Since $R\cap N$ is a $p$-subgroup, by Sylow Inclusion and Conjugate Theorems~\ref{sylow-c} and~\ref{sylow-d}, $R\cap N\subseteq P^y$ for some $y\in N$. Using that $Q^y\subseteq N$, we get
$$
    H^y = P^y\cap Q^y \supseteq R\cap N\cap Q^y = R\cap Q^y,
$$
i.e.,
$$
    H \supseteq R^{y^{-1}}\cap Q.
$$
By the minimality of $P\cap Q$, equality must be attained in the last inclusion. Therefore,
$$
    H = H^y = R\cap Q^y \subseteq R,
$$
as desired.  \end{proof}



\subsection{Problems C}

\begin{probl}\label{problem-1.C.1}
    Let $P\in\Syl_p(G)$, and suppose that $N_G(P)\subseteq H \subseteq G$, where $H$ is a subgroup. Prove that $H = N_G(H)$.

    \textrm{\rm {\bf Note}. This generalizes Problem \ref{problem-1.B.3}}.
\end{probl}

\begin{solution} Put $N=N_G(H)$. Since $\abs N\mid\abs G=p^em$, with $p\perp m$, and $\abs P=p^e$, we deduce that $P\in\Syl_p(N)$. By the Frattini Argument~(\ref{frattini-argument}) applied to $P$, $G=N$ and $N=H$, we deduce that
$$
    N=N_N(P)H\subseteq N_G(P)H\subseteq HH=H\subseteq N_G(H)=N.
$$
 \end{solution}

\begin{probl}\label{problem-1.C.2}
    Let $H \subseteq G$, where $G$ is a finite group.
    \begin{enumerate}[\rm a)]
        \item If\/ $P\in\Syl_p(H)$, prove that $P=H\cap Q$ for some $Q\in \Syl_p(G)$.

        \item Show that $n_p(H) \le n_p(G)$ for all primes $p$.
    \end{enumerate}
\end{probl}

\begin{solution}
\begin{enumerate}[\rm a)]
    \item Put $\abs G=p^em$ with $p\perp m$. Then $\abs H=p^dn$ with $d\le e$ and $n\mid m$.
    Given that $P$ is also a $p$-subgroup of $G$, by Theorem~\ref{sylow-d}, there exists $Q\in\Syl_p(G)$ such that $P\subseteq Q$. Therefore, $P\subseteq H\cap Q$, where $\abs{H\cap Q}=p^r$ for some $r\le d$.

    Since $H\cap Q$ is a $p$-subgroup of $H$, it must be included in some Sylow $p$-subgroup $R\in\Syl_p(H)$. Then $P\subseteq H\cap Q\subseteq R$, which implies $P=R$ because both groups have order $p^d$.
    
    \item This is a direct consequence of part a).
\end{enumerate}
\end{solution}

\begin{probl}\label{problem-1.C.3}
    Let\/ $G$ be a finite group, and let $X$ be the subset of\/ $G$ consisting of all elements whose order is a power of\/ $p$, where $p$ is some fixed prime.
    \begin{enumerate}[\rm a)]
        \item Show that $X = \bigcup\Syl_p(G)$.
        \item If\/ $p\mid\abs G$, then $p\mid\abs X$.
        
        \textrm{\rm Hint. Let a Sylow $p$-subgroup of $G$ act on $X$.}
    \end{enumerate}

    \textrm{\rm\textbf{Note.} In other words, if $p\mid|G|$, the number of elements whose order is a positive power of $p$ is congruent to $-1$ modulo $p$}
\end{probl}

\begin{solution}
\begin{enumerate}[\rm a)]
    \item This is a direct consequence of the Sylow Inclusion Theorem~\ref{sylow-d}.

    \item If $p\mid\abs G$, then $G$ includes some nontrivial Sylow $p$-group $P$. Let $P$ act on $X$ by conjugation. The action is well defined because conjugations preserve order. By the the Fundamental Counting Principle Theorem~\ref{fundamental-counting-principle}, given $x\in X$, $|{\cal O}_x|=\abs{P:C_P(x)}$ [cf.~Remark~\ref{basic-stabilizers}]. Since $\abs{P:C_P(x)}\mid\abs P$ and $\abs P$ is a power of $p$, we deduce that $\abs{{\cal O}_x}$ divides some power of $p$. In other words, $\abs{{\cal O}_x}=1$ or $p\mid \abs{{\cal O}_x}$.
    
    Suppose that $\abs{P:C_P(x)}=1$. Then $C_P(x)=P$, i.e., $x\leftrightarrow P$. Then $\gen xP$ is a group. And since its elements have orders that are powers of~$p$, we must have $\gen xP=P$, i.e., $x\in P$. Therefore, $\abs{{\cal O}_x}=1$ exactly $\abs P$ times and for all $x\notin P$, $p\mid\abs{{\cal O}_x}$.
\end{enumerate}
\end{solution}

\begin{probl}
    Let $\abs G = 120 = 2^3 \cdot 3 \cdot 5$. Show that $G$ has a subgroup of index $3$ or a subgroup of index\/ $5$ (or both).

    \textrm{\rm Hint: Analyze separately the four possibilities for $n_2(G)$}.
\end{probl}

\begin{solution} Let $n_2=n_2(G)$, $n_3=n_3(G)$ and $n_5=n_5(G)$. 
\begin{enumerate}[1.]
    \item $n_2\mid3\cdot5$ $\leftarrow$ Remark \ref{p'-part}
    \item $n_2\in\set{1, 3, 5, 3\cdot5}$
    \item $n_2\in\set{3,5}$ $\implies$ we are done
        $\leftarrow$ Corollary \ref{n_p-is-index}
    \item Interesting cases are $n_2=1$ and $n_2=3\cdot5$
    \item $n_5\in\set{1, 6}$ $\leftarrow$ $n_5\equiv1\mod 5$ and $n_5\mid 2^3\cdot 3$
    \item Case $n_2=1$:
    \begin{enumerate}[\rm -]
        \item $\Syl_2(G)=\set{P}$, with $P\normal G$
            $\leftarrow$ Problem \ref{problem-1.B.1}
        \item $Q\in\Syl_3(G)\implies\abs{G:PQ}=5$
            $\leftarrow$ $PQ$ is group and $P\cap Q=\gen1$
        \item $R\in\Syl_5(G)\implies\abs{G:PR}=3$
            $\leftarrow$ $PR$ is group and $P\cap R=\gen1$
    \end{enumerate}
    \item Case $n_2=3\cdot5$:
        \begin{enumerate}[\rm -]
            \item $P_1\ne P_2\in\Syl_2(G)\implies\abs{P_1:P_1\cap P_2}= 2$
                $\leftarrow$ Theorem \ref{n_p(G)=1}
            \item $P_1\ne P_2\in\Syl_2(G)\implies\abs{P_1\cap P_2}= 2^2$
            \item $P_1\cap P_2\normal P_1$ and ${}\normal P_2$,
                so $P_1\cup P_2\subseteq H=N_G(P_1\cap P_2)$
            \item  $2^3\mid\abs H$ and $9\le\abs H$
            \item $\abs H<\abs G$ $\leftarrow$ see below
            \item $\abs H=2^3\cdot3$ or $\abs H=2^3\cdot5$
            \item $\abs{G:H}\in\set{3,5}$
        \end{enumerate}
\end{enumerate}

\begin{enumerate}
    \item[$H\varsubsetneq$] $G$: Suppose otherwise. Pick $P\in\Syl_2(G)$. The Frattini Argument implies that $G=N_G(P)(P_1\cap P_2)$. Since $P\nnormal G$, $\abs{N_G(P)}=2^3q$ with $q\in\set{1,3,5}$. But $\abs{P_1\cap P_2}=4$ and so $\spec \abs{N_G(P)(P_1\cap P_2)}\varsubsetneq\set{2,3,5}$.
\end{enumerate}
\end{solution}

\begin{probl}\label{problem-1.C.5}
    Let $P\in\Syl_p(G)$, where $G=A_{p+1}$, the alternating group on $p+1$ symbols. Show that $\abs{N_G(P)} = p(p-1)/2$.

    \textrm{\rm Hint. Count the elements of order $p$ in $G$.}
\end{probl}

\begin{solution} The result is false for $p=2$ because $|A_{2+1}|=2^0\cdot 3$ and so $\gen1$ is the only $2$-subgroup of $G$, with $N_G\gen1=A_3$.

Let's recall that $|A_{p+1}|=(p+1)!/2$. In the case $p>3$, we have
$$
    (p+1)!/2 = (p+1)p(p-1)\cdots3,
$$
i.e., $|A_{p+1}|=pm$, where $m=(p+1)(p-1)\cdots3\perp p$. If $p=3$, then $|A_{p+1}|=3\cdot2^2$ and $m=4$. Hence, in both cases, $\abs P=p$.

Recall that a cyclic permutation in $S_n$ has the form $\zeta=(a_1\dots a_n)$ where the $a_i$ are all different and $z(a_i)=a_{i+1\pmod n}$. By induction on $n$ it is easy to see that $\ord(\zeta)=n$ and that $\zeta=(a_1 a_n)(a_1 a_{n-1})\cdots(a_1 a_2)$, which implies that $\sg(\zeta)=n-1$. In addition, every element $\sigma\in S_n$ can be written as a product
$$
    \sigma=\zeta_1\cdots\zeta_k,
$$
where the $\zeta_i$ are cyclic and disjoint (in the natural sense of the meaning). Since the $\zeta_i$ commute among themselves, $\ord(\sigma)=\lcm\set{\ord(\zeta_i)\mid 1\le i\le k}$.

If we apply what we just recalled to the case $n=p+1$, we see that all permutations $\sigma$ of order $p$ in $A_{p+1}$ are decomposable in disjoint cycles
$$
    \sigma=\zeta_1\cdots\zeta_k,
$$
where $\ord(\zeta_i)=p$. In particular, $pk\le p+1$ and so $k=1$. This means that the permutations of order $p$ are all the cyclic permutations of $p$ symbols, which amount for a total of $(p+1)(p-1)!$ cycles: for every fixed $k\in\nset{p+1}$, consider the $(p-1)!$ cyclic permutations of the remaining $p$ integers.

Thus, to count the number of $p$-subgroups we have to see when two cycles of length $p$ generate the same group.

If $\sigma$ is such a cycle, then there are exactly $p-1$ cycles that generate the same group, namely $\sigma, \sigma^2,\dots,\sigma^{p-1}$. Therefore, the number of $p$-subgroups is 
$$
    n_p(G)=(p+1)(p-1)!/(p-1) = (p+1)(p-2)!
$$
Thus, $(p+1)(p-2)!=|G:N_G(P)|$. It follows that
$$
    |N_G(P)| = \frac{(p+1)!}{2(p+1)(p-2)!}=p(p-1)/2.
$$
 \end{solution}

\begin{probl}
    Let $G = HK$, where $H$ and $K$ are subgroups, and let $p$ be a fixed prime. Show that
    \begin{enumerate}[\rm a)]
        \item There exists $P\in\Syl_p(G)$ such that $P\cap H\in\Syl_p(H)$ and $P\cap K\in\Syl_p(K)$.
        
        \textrm{\rm Hint. First choose $Q\in\Syl_p(G)$ and $x\in G$ such that $Q\cap H\in\Syl_p(H)$ and $Q^x\cap K\in\Syl_p(K)$. Write $x^{-1}y=w$, with $y\in H$ and $w\in K$.}

        \item If $P$ is as in\/ {\rm a)}, then $P=(P\cap H)(P\cap K)$.
    \end{enumerate}
\end{probl}

\begin{solution}

\begin{enumerate}[\rm a)]
    \item By Problem~\ref{problem-1.C.2} there exists $Q\in\Syl_p(G)$ such that $Q\cap H\in\Syl_p(H)$. Similarly, there exists also $Q_1\in\Syl_p(G)$ such that $Q_1\cap K\in\Syl_p(K)$. By Theorem~\ref{sylow-c}, $Q_1=Q^x$ for some $x\in G$. Since $G=KH$, we can write $x^{-1}=wy^{-1}$ with $y\in H$ and $w\in K$. Take $P=Q^y$. Then
    \begin{align*}
        P\cap H &= Q^y\cap H = Q^y\cap H^y = (Q\cap H)^y\in\Syl_p(H)\\
        P\cap K &= Q_1^{x^{-1}y}\cap K = Q_1^w\cap K = (Q_1\cap K)^w\in\Syl_p(K)\\        
    \end{align*}

    \item Put $\abs G=p^em$ with $p\perp m$, $P\cap H=p^{e_1}m_1$ with $p\perp m_1$ and $K=p^{e_2}m_2$ with $p\perp m_2$. Then
    \begin{align*}
        \abs{(P\cap H)(P\cap K)} &= \frac{\abs{P\cap H}\abs{P\cap K}}
                {\abs{P\cap H\cap K}}
            = \frac{p^{e_1+e_2}}{\abs{P\cap H\cap K}}.
    \end{align*}
    Moreover, from $G=HK$ we obtain
    $$
        \abs{H\cap K} = p^{e_1+e_2-e}\frac{m_1m_2}{m}.
    $$
    In particular $\abs{P\cap H\cap K}\le p^{e_1+e_2-e}$. Thus
    $$
        \abs{(P\cap H)(P\cap K)} \ge \frac{p^{e_1+e_2}}{p^{e_1+e_2-e}}=p^e=\abs P.
    $$
    Since $(P\cap H)(P\cap K)\subseteq P$, equality is attained.
\end{enumerate}
\end{solution}


\begin{probl}\label{problem-1.C.7}
    Let $G$ be a finite group in which every maximal subgroup has prime index, and let $p$ be the largest prime divisor of\/ $|G|$. Show that a Sylow $p$-subgroup of\/ $G$ is normal.

    \textrm{\rm Hint. Otherwise, let $M$ be a maximal subgroup of $G$ containing $N_G(P)$, where $P\in\Syl_p(G)$. Compare $n_{p}(M)$ and $n_p(G)$.}
\end{probl}

\begin{solution} Suppose that $P\in\Syl_p(G)$ is not normal, i.e., $P\subseteq N_G(P)\varsubsetneq G$. Let $N_G(P)\subseteq M\varsubsetneq G$ be a maximal subgroup ($M$ does exist because $G$ is finite). Put $\abs G=p^em$ with $p\perp m$. By hypothesis, $\abs{G:M}=q$, where $q$ is a prime and $q\le p$. Since $\abs P=p^e$ and $P\subseteq M$, we see that $P\in\Syl_p(M)$. %Hence, $q\mid m$. In particular, $q<p$.

%By Problem~\ref{problem-1.C.1}, $N_G(M)=M$. By Problem~\ref{problem-1.C.2}, $n_p(G)\ge n_p(M)$.

Taking into account that
$$
    N_M(P)=\set{x\in M\mid P^x=P} = N_G(P)\cap M= N_G(P),
$$
we deduce that %$|N_G(P)|=|N_M(P)|$. Therefore,
\begin{align*}
    n_p(G) &= |G:N_G(P)|\\
        &= \abs{G:M}|M:N_G(P)|\\
        &= q\abs{M:N_M(P)}\\
        &= qn_p(M).
\end{align*}
Applying Corollary~\ref{p|n_p-1} to both sides of the equation, we obtain
$$
    q\equiv1\pmod p,
$$
which implies $q=1$, a contradiction because $q$ was prime.  \end{solution}

\begin{probl}\label{problem-1.C.8}
    Let\/ $P$ be a Sylow\/ $p$-subgroup of\/ $G$. Show that for every nonnegative integer\/ $d$, the numbers of subgroups of order\/ $p^d$ in\/ $P$ and in\/ $G$ are congruent modulo\/~$p$.

    \textrm{\rm {\bf Note.} If $p^d = |P|$, then the number of subgroups of order $p^d$ in $P$ is clearly~$1$, and it follows that the number of such subgroups in $G$ is congruent to~$1$ modulo~$p$. This provides a somewhat different proof that $n_p(G) = 1 \mod p$. It is true in general that if $p^d < |P|$, then the number of subgroups of order $p^d$ in $P$ is congruent to $1$ modulo $p$, and thus it follows that if $p^d$ divides the order of an arbitrary finite group $G$, then the number of subgroups of order $p^d$ in $G$ is congruent to $1$ mod $p$ [cf.~Corollary~\ref{p-subgroups}].}
\end{probl}

\begin{solution}
Consider the sets
    \begin{align*}
        A &= \set{H\subseteq G\mid \abs H=p^d}\\
        B &= \set{H\subseteq P\mid \abs H=p^d}.
    \end{align*}
    We can let $P$ act on $A$ and $B$ by conjugation. The orbits of any given~$H$ would have cardinalities $|P:P\cap N_G(H)|$ ($H\in A$) and $|P:N_P(H)|$ ($H\in B$). Since these numbers are divisible by $p$ unless $P\subseteq N_G(H)$ or $H\normal P$, in order to compute $\abs A$ and $\abs B$ modulo~$p$ it is enough to compare the cardinalities of
    \begin{align*}
        \hat A &= \set{H\subseteq G\mid \abs H=p^d,\;P\subseteq N_G(H)}\\
        \intertext{and}
        \hat B &= \set{H\normal P\mid \abs H=p^d}.
    \end{align*}
    Clearly, $\hat B\subseteq\hat A$. We claim that $\hat A\subseteq\hat B$ also. To see this, take $H\in\hat A$ and write $N=N_G(H)$. We have $P\subseteq N$ and so $P\in\Syl_p(N)$. Since $H\subseteq N$, according to Theorem~\ref{sylow-d}, there exists $y\in N$ such that $H\subseteq P^y$. Put $z=y^{-1}$. Then $H^z\subseteq P$ and, since $z\in N$, $H=H^z\subseteq P\subseteq N_G(H)$. Hence, $H\in\hat B$.  \end{solution}

\section{Normal Series}

\begin{thm}\label{nontrivial-center}
    Let $G$ be a finite $p$-group and let $N\ne\gen1$ be a normal subgroup of\/~$G$. Then $\abs{N\cap Z(G)}>1$. In particular, if\/ $G$ is nontrivial, $Z(G)\ne\gen1$.

    \textrm{\rm[See also Corollaries \ref{p-groups-have-center} and \ref{normal-nilpotent-center}].}
\end{thm}

\begin{proof} Let $G$ act on $N$ by conjugation. The action is well-defined because $N^x=N$ for all $x\in G$. Given $y\in N$ its orbit is
$$
    {\cal O}_y = \set{xyx^{-1}\mid x\in G}.
$$
In particular,
$$
    |{\cal O}_y| = 1 \iff y\in Z(G)\cap N.
$$
By the Fundamental Counting Principle Theorem \ref{fundamental-counting-principle},
$$
    |{\cal O}_y| = |G: G_y|,
$$
where
$$
    G_y = \set{x\in G\mid x\leftrightarrow y} = C_G(y).
$$
Given that $p\mid|G:G_y|$ whenever $y\notin Z(G)$, we obtain
$$
    p\mid |N\setminus Z(G)\cap N|=\abs N-\abs{Z(G)\cap N}.
$$
Since $p\mid\abs N$, we deduce that $p\mid\abs{Z(G)\cap N}$. The conclusion is now clear.  \end{proof}

\needspace{2\baselineskip}
\begin{defns}Let $G$ be a group.
    \begin{enumerate}[\rm i)]
        \item A \textsl{normal series} is a finite sequence
        $$
            \gen1=N_0\subseteq N_1\subseteq\cdots\subseteq N_r=G,
        $$
        where the $N_i$ are normal groups.

        \item A \textsl{central series} is a normal series that verifies
        $$
            N_i/N_{i-1}\subseteq Z(G/N_{i-1})
        $$
        \item $G$ is \textsl{nilpotent} if it has a central series.
    \end{enumerate}
\end{defns}

\begin{rem}\label{nontrivial-nilpotent-center}
    If\/ $G$ is nilpotent and nontrivial, the first nontrivial subgroup of a central series is included in the center\/ $Z(G)$. In particular, $Z(G)\ne\gen1$.
\end{rem}

\begin{ntns}
    Recall that the \textsl{commutator} of two elements $x$ and $y$ in a group $G$ is denoted by
    $$
        [x,y] = xyx^{-1}y^{-1}.
    $$
    The function
    \begin{align*}
        \ct_x\colon G&\to G\\
        y&\mapsto[x,y]
    \end{align*}
    is called the \textsl{commutator map} of~$x$.
\end{ntns}

\begin{prop}\label{ad-nilpotent-test}
    A normal series $N_0\subseteq N_1\subseteq\cdots\subseteq N_r$ in a group $G$ is central if, and only if, for every $x\in G$ the commutator map of~$x$ satisfies
    $$
        \ct_x(N_i)\subseteq N_{i-1}\qquad(1\le i\le r).
    $$
\end{prop}
\begin{proof}
\begin{align*}
    \text{The series is central}
        &\iff N_i/N_{i-1}\subseteq Z(G/N_{i-1})&&(1\le i\le r)\\
        &\iff xy\in yxN_{i-1}&&(x\in G, y\in N_i, 1\le i\le r)\\
        &\iff x^{-1}y^{-1}\in y^{-1}x^{-1}N_{i-1}
            &&(x\in G, y\in N_i, 1\le i\le r)\\
        &\iff [x,y]\in N_{i-1}
            &&(x\in G, y\in N_i, 1\le i\le r)\\
        &\iff \ct_x(N_i)\subseteq N_{i-1}&&(x\in G,1\le i\le r)
\end{align*}
 \end{proof}

\begin{cor}\label{normal-nilpotent-center}
    Let\/ $G$ be nilpotent. If\/ $\gen1\ne N\normal G$ then\/ $N\cap Z(G)\ne\gen1$.
\end{cor}

\begin{proof} Take the central series $\gen1=N_0\subseteq N_1\subseteq\cdots\subseteq N_r=G$. Define
$$
    m = \max\set{i\in\nset r\mid N_i\cap N=\gen1}.
$$
Since $N\ne\gen1$, $m<r$ and $N_{m+1}\cap N\ne\gen1$. Take $x\in G$. According to the proposition,
$$
    \ct_x(N_{m+1}\cap N)\subseteq\ct_x(N_{m+1})\cap\ct_x(N)\subseteq N_m\cap N
        =\gen1
$$
because $\ct_x(N)\subseteq N^xN=N$. In consequence,
$$
    \gen1\ne N_{m+1}\cap N \subseteq\bigcap_{x\in G}\ct_x^{-1}(1)=Z(G).
$$
 \end{proof}



\begin{cor}\label{nilpotent-subgroups-and-quotients}
    If\/ $G$ is nilpotent, then every subgroup and every quotient of\/ $G$ are nilpotent.
\end{cor}

\begin{proof} Let $H$ be a subgroup of $G$. With the preceding notations,
$$
    \gen1=H\cap N_0\subseteq N_1\subseteq\cdots\subseteq H\cap N_r=H
$$
is a normal series in $H$ because $H\cap N_i\normal H$. In addition, the series is central because, for every $y\in H$, we have
$$
    \ct_y(H\cap N_i)\subseteq\ct_y(H)\subseteq\ct_y(N_i)\subseteq H\cap N_{i-1}.
$$
Let now $\varphi\colon G\to H$ be a epimorphism of groups. Then,
$$
    \gen1=\varphi(N_0)\subseteq\varphi(N_1)\subseteq\cdots\subseteq\varphi(N_r)=H
$$
is a normal series because, for every $N\normal G$ and $x\in G$, we have
$$
    \varphi(N)^{\varphi(x)}=\varphi(N^x)=\varphi(N),
$$
which proves that $\varphi(N_i)\normal H$. The series is central because the diagrams
$$
    \begin{tikzcd}
        G \arrow[r, "\ct_x"] \arrow[d, "\varphi"'] & G \arrow[d, "\varphi"] & N_i \arrow[r] \arrow[d] & N_{i-1} \arrow[d] \\
        H \arrow[r, "\ct_{\varphi(x)}"'] & H & \varphi(N_i) \arrow[r, dashed] & \varphi(N_{i-1}) 
    \end{tikzcd}
$$
are well-defined and commute.  \end{proof}

\begin{rem}\label{preimage-of-normal}
    Given that the kernel of a morphism of groups is always normal in the domain, the preimage of a normal group is normal because it is the kernel of the composition of the morphism with the projection onto the quotient by the normal subgroup.
    $$
        \begin{tikzcd}
            G \arrow[r,"\phi"] \arrow[rd,"\varphi\circ\phi"'] & H \arrow[d,"\varphi"] \\
            & H/N
        \end{tikzcd}\qquad \phi^{-1}(N)=\phi^{-1}(\ker(\varphi))=\phi^{-1}(\varphi^{-1}\gen1)=\ker(\varphi\circ\phi).
    $$
\end{rem}

\begin{defn}
    The \textsl{upper central series} of a group $G$ is the increasing sequence of normal groups inductively defined by
    \begin{align*}
        Z_0 &=\gen1\\
        Z_{i+1} &=\varphi_i^{-1}(Z(G/Z_i))\quad(i\ge0),
    \end{align*}
    where $\varphi_i\colon G\to G/Z_i$ is the projection onto the quotient.
\end{defn}

\begin{prop}\label{nilpotent-center-series}
    Let $G$ be a finite group. Then the following conditions are equivalent
    \begin{enumerate}[\rm a)]
    \item $G$ is nilpotent.
    \item Every nontrivial homomorphic image of\/ $G$ has a nontrivial center.
    \item $G$ appears as a member of its upper central series.
    \end{enumerate}
\end{prop}

\begin{proof}${}$
\begin{enumerate}[\rm a)]
    \item $\Rightarrow$ b) This follows from Corollary~\ref{nilpotent-subgroups-and-quotients} and the fact that the first nontrivial term of a central series is included in the center.
    \item $\Rightarrow$ c) Let $Z_i$ be the $i$th group of the upper central series. By condition b), if $Z_i\varsubsetneq G$ then $Z(G/Z_i)\ne\gen1$. Therefore $Z_i\varsubsetneq Z_{i+1}$, and the sequence will eventually reach the finite group~$G$.
    \item $\Rightarrow$ a) Trivial.
\end{enumerate}
\end{proof}


\begin{cor}\label{p-groups-are-nilpotent}
    Every finite $p$-group is nilpotent.
\end{cor}

\begin{proof} If $G$ is a finite $p$-group, every nontrivial homomorphic image of $G$ is also a finite $p$-group. By Theorem~\ref{nontrivial-center}, such an image has a nontrivial center. Then $G$ is nilpotent by the proposition.  \end{proof}

\begin{thm}
    Let\/ $G$ be a (not necessarily finite) nilpotent group with central series
    $$
        \gen1 = N_0 \subseteq N_1 \subseteq \cdots \subseteq N_r = G,
    $$
    as usual, and
    $$
        \gen1 = Z_0 \subseteq Z_1 \subseteq \cdots
    $$
    its upper central series. Then $N_i \subseteq Z_i$ for $0\le i\le r$. In particular, $Z_r = G$.
\end{thm}

\begin{proof} Let's prove the inclusion $N_i\subseteq Z_i$ by induction on~$i$. The case $i=0$ is trivial. Assuming that $N_{i-1}\subseteq Z_{i-1}$, the projection $\phi\colon G\to G/Z_{i-1}$ induces an epimorphism $\bar\varphi\colon G/N_{i-1}\to G/Z_{i-1}$. Since $N_i/N_{i-1}\subseteq Z(G/N_{i-1})$ and since every epimorphism maps the center of its domain into the center of its codomain, we deduce that $\bar\varphi(N_i/N_{i-1})\subseteq Z(G/Z_{i-1})$, i.e., $N_i\subseteq\varphi^{-1}(Z(G/Z_{i-1}))=Z_i$.  \end{proof}

\begin{defn}
    If\/ $G$ is nilpotent, the integer $r$ such that the $r$th upper central group $Z_r$ equals $G$ is called the \textsl{nilpotent class} of\/~$G$ and denoted by~$c(G)$.
\end{defn}

\begin{rem}
    $G$ is abelian if, and only if, $c(G)\le1$. Also, $c(G)=2$ if, and only if, $G/Z(G)$ is abelian.
\end{rem}

\begin{cor}
    The nilpotence class of a nilpotent group is exactly the length of its shortest possible central series.
\end{cor}

\begin{proof} Let $r$ be the length of the shortest possible central series of a nilpotent group $G$. This number is well-defined because the upper central series is actually a central series of~$G$. If $N_r$ is the $r$th group of a central series that realizes the smallest length, then the theorem implies that $Z_r=G$ and so $r=c(G)$.  \end{proof}

\begin{thm}\label{normalizers-grow}
    Let\/ $H\varsubsetneq G$ be a proper subgroup of a (not necessarily finite) nilpotent group\/ $G$. Then\/ $H\varsubsetneq N_G(H)$.
\end{thm}

\begin{proof} Pick a central series $(N_i)_{0\le i\le r}$ of $G$. Let $k\in\nset r$ be such that $N_k\subseteq H$ but $N_{k+1}\not\subseteq H$. Such a $k$ exists because $N_0=\gen1\subseteq H$ and $H$ is proper. Let $\varphi\colon G\to G/N_k$ be the projection onto the quotient. Put $\bar G=G/N_k$ and $\bar H=H/N_k$. Then
$$
    \varphi(N_{k+1})=N_{k+1}/N_k\subseteq Z(\bar G) \subseteq N_{\bar G}(\bar H)=\varphi(N_G(H)),
$$
where the last equality holds by Proposition~\ref{image-of-normalizer}. Since $N_k=\ker(\varphi)$ and $N_{k+1}\supseteq N_k$ and $N_G(H)\supseteq N_k$, we get $N_{k+1}\subseteq N_G(H)$. Hence, $H\ne N_G(H)$, as desired.  \end{proof}

\begin{lem}\label{subgroup-of-index-p}
    Let\/ $G$ be a finite\/ $p$-group and suppose that\/ $N\varsubsetneq M$ are normal in\/ $G$. Then there exists\/ $L\normal G$ such that\/ $N \subseteq L \subseteq M$ and\/ $\abs{L:N}=p$.
\end{lem}

\begin{proof} Consider the quotient $\varphi\colon G\to G/N$. Then $\bar G=G/N$ is a $p$-group and $\bar M=M/N$ a nontrivial normal subgroup. By Theorem~\ref{nontrivial-center}, $\bar M\cap Z(\bar G)$ is nontrivial. Since it is also a $p$-group we can pick $\bar z\in\bar M\cap Z(\bar G)$ with $\ord(\bar z)=p$. In particular, $L=N\gen z$ is normal because $\varphi(z^x) = \varphi(z)$ and so $z^x\in zN\subseteq N\gen z$. By Lemma~\ref{HK-cardinality},
$$
    \abs{L:N}=|N\gen z|/|N|=|\gen z|=p
$$
because $N\cap\gen z=\gen1$.  \end{proof}

\begin{thm}\label{p^d-subgroups}
    Let\/ $G$ be a $p$-group of order $p^e$. Then, for every integer~$d$ with $0\le d<e$, there exists $L\normal G$ such that $|L| = p^d$.
\end{thm}

\begin{proof} The lemma for $N=\gen1$ and $M=G$ implies the existence of $L\normal G$ such that $\abs{G:L}=p$. Thus $\abs L=p^{e-1}$ and the proof follows by decreasing induction on $d$ replacing $M$ with $L$.  \end{proof}

\medskip

\textbf{Note.} For the following two lemmas and the theorem that comes after them refer to the MSE answer: \href{https://math.stackexchange.com/a/1586494/269050}{\it The number of\/ $p$-subgroups of a group}.

\begin{lem}\label{lemma1}
     Let\/ $G$ be a $p$-group and $H$ a proper subgroup. Suppose that $\abs{H}=p^d$. Then the number of subgroups of order $p^{d+1}$ which contain $H$ is congruent to $1$ modulo $p$.
\end{lem}

\begin{proof} Define
$$
    A=\set{K\mid \abs K=p^{d+1},\; H\subseteq K}.
$$
Then the lemma predicts that $\abs A\equiv1\pmod p$. Note also that the condition $H\subseteq K$ is equivalent to $H\normal K$ because of Problem~\ref{problem-1.A.1}. Put $N=N_G(H)$. Then,
$$
    A=\set{K\subseteq N\mid \abs K=p^{d+1},\; H\normal K}.
$$
Moreover, if $\bar N=N/H$, $A$ is in biunivocal correspondence with
$$
    \bar A=\set{\bar K\subseteq\bar N\mid|\bar K|=p}.
$$
Because of their order, each of the elements of $\bar A$ is cyclic. According to Problem \ref{cauchy}, the number $q$ of elements in $\bar N$ with order $p$ satisfies $q\equiv-1\pmod p$. And since the $p-1$ powers $x, x^2,\dots, x^{p-1}$ of each of these elements are the ones that generate the same cyclic subgroup $\gen x$, we deduce
$$
    |\bar A| = \frac q{p-1}.
$$
Hence $(p-1)|\bar A|=q$, which implies $|\bar A|\equiv1\pmod p$.  \end{proof}

\begin{lem}\label{lemma2}
    Let\/ $G$ be a $p$-group of order $p^e>1$. Then the number of subgroups of order\/ $p^{e-1}$ is congruent to $1\pmod{p}$.
\end{lem}

\begin{proof} Define
$$
    A = \set{H\subgroup G\mid\abs H=p^{e-1}}.
$$
Then $A\ne\emptyset$ by Theorem~\ref{p^d-subgroups}. Note also that, by Problem~\ref{problem-1.A.1}, we can write
$$
    A = \set{H\normal G\mid\abs H=p^{e-1}}.
$$
Fix $H\in A$ and introduce the equivalence relation in $A$ given by
$$
    K\sim_H L\iff K\cap H= L\cap H.
$$
Note that $K\sim_HH$ implies $K=H$, i.e., the class $[H]$ of $H$ in $A/{\sim_H}$ is a singleton.

Given $K\ne L$ in $A$, the equation $\abs{KL}=p^{2e-2}/\abs{K\cap L}$ implies
$$
    G = KL\quad\text{and}\quad\abs{K\cap L}=p^{e-2}.
$$
Let $K\not\sim_HH$. Its class in $A/{\sim_H}$ satisfies
$$
    [K] = \set{L\in A\mid L\supseteq K\cap H}\setminus\set H.
$$
By the previous lemma applied to $H=K\cap H$, we get 
$$
    |\set{L\in A\mid L\supseteq K\cap H}| \equiv 1\pmod p
$$
and so $|[K]|\equiv0\pmod p$. Since $[H]$ is a singleton and
$$
    A = \bigg(\bigcup_{[K]\ne[H]}[K]\bigg)\cup [H],
$$
it follows that $\abs A\equiv1\pmod p$.  \end{proof}

\begin{thm}
    Let\/ $G$ be a group of order $p^e$, where $p$ is prime. Then
    $$
        |\set{H\subgroup G\mid |H|=p^d}|\equiv1\pmod{p}
    $$
    for\/ $0 \le d \le e$.
\end{thm}

\begin{proof} Given $0\le d\le e$ define
$$
    S_d=\set{H \subgroup G\mid \abs H=p^d}
$$
and, for $d<e$, introduce the function
\begin{align*}
    \chi_d\colon S_d\times S_{d+1}&\to\set{0,1}\\
    (H,K)&\mapsto\begin{cases}
        1   &H\subseteq K,\\
        0   &\rm otherwise.
    \end{cases}
\end{align*}
Then, Lemma~\ref{lemma1} implies
$$
    \sum_{H,K}\chi_d(H,K) = \sum_H\sum_{H\subgroup K}1
        \equiv\sum_{H\in S_d}1\pmod p
$$
while Lemma~\ref{lemma2}
$$
    \sum_{H,K}\chi_d(H,K) = \sum_K\sum_{H\subgroup K}1
        \equiv\sum_{K\in S_{d+1}}1\pmod p.
$$
In sum,
$$
    |S_d|\equiv|S_{d+1}|\pmod p,
$$
i.e.,
$$
    |S_0|\equiv|S_1|\equiv\cdots\equiv|S_e|\pmod p.
$$
The conclusion follows because $\gen1$ is the only subgroup of order $p^0$.  \end{proof}

\begin{cor}\label{p-subgroups}
     If\/ $p^d$ divides the order of\/ $G$, the number of\/ $p$-subgroups of\/~$G$ of order\/ $p^d$ is congruent to\/ $1$ modulo\/ $p$.
\end{cor}

\begin{proof} By Problem \ref{problem-1.C.8} the number of $p$-subgroups of order $p^d$ equals the order of $p$-subgroups of any of its Sylow $p$-groups. By the theorem, the number of $p$-subgroups of any such Sylow group is congruent to $1$ modulo $p$.  \end{proof}

\begin{thm}\label{nilpotent-equivalences}
    Let\/ $G$ be a finite group. The following are equivalent:
    \begin{enumerate}[\rm a)]
        \item $G$ is nilpotent.
        \item $H\varsubsetneq N_G(H)$ for every proper subgroup $H\varsubsetneq G$.
        \item Every proper maximal subgroup of\/ $G$ is normal.
        \item Every Sylow subgroup of\/ $G$ is normal.
        \item $G$ is the (internal) direct product of its nontrivial Sylow subgroups.
    \end{enumerate}
\end{thm}

\begin{proof}${}$

\begin{enumerate}[\rm a)]
    \item $\Rightarrow$~b) See Theorem~\ref{normalizers-grow}.
    \item $\Rightarrow$~c) Let $H$ be a proper maximal subgroup. Since $H$ is a proper subgroup of $N_G(H)$, we must have $N_G(H)=G$.
    \item $\Rightarrow$~d) Let $P$ be a Sylow $p$-subgroup. If $N_G(P)$ is not $G$, there exists a maximal subgroup $N$ including $N_G(P)$. Then $N\normal G$ and the Fattini Argument~(\ref{frattini-argument}) implies that $G=N_G(P)N=N$, which is impossible.
    \item $\Rightarrow$~e) Let $F=\prod_{P\in\Syl(G)}O_p(G)$ be the product of all $p$-cores of $G$. By Problem~\ref{problem-1.B.1}, $O_p(G)$ is a normal Sylow $p$-group. By Theorem~\ref{direct-product}, $F$ is the direct product of the $p$-cores because their orders are pairwise coprime. Moreover, $F=G$ because both groups have the same order.
    \item $\Rightarrow$~a) By Proposition \ref{nilpotent-center-series} to see that $G$ is nilpotent it is enough to show that every nontrivial homomorphic image of $G$ has a nontrivial center. Since $G$ is the direct product of its nontrivial Sylow subgroups, every Sylow subgroup of $G$ is normal, i.e., e) $\Rightarrow$ d). Then every homomorphic image of $G$ inherits this property (Proposition~\ref{image-of-normalizer}). Then, every image of $G$ satisfies e). And since the center of a direct product is the direct product of the centers, we are done.
\end{enumerate}
\end{proof}

\begin{defn}
    The \textsl{fitting group} of\/ $G$ is the direct product
    $$
        F(G)=\prod_{p\;\mid\;\abs G}O_p(G),
    $$
    which, incidentally, is characteristic in\/ $G$.
\end{defn}

\begin{cor}\label{normal-nilpotent-fitting}
    Let\/ $G$ be a finite group. Then\/ $F(G)$ is a normal nilpotent subgroup of\/ $G$. It contains every normal nilpotent subgroup of\/ $G$, and so it is the unique largest such subgroup.
\end{cor}

\begin{proof} Since $F(G)$ is characteristic, it is normal. Given that every $O_p(G)$ is the maximal $p$-subgroup in itself, it is maximal in $F(G)$. Then $F(G)$ is nilpotent by part~e) of the theorem.

Take $N\normal G$ and nilpotent. If $P\in\Syl_p(N)$, by the theorem, $P\normal N$. Then $P=O_p(N)$, which is characteristic in $N$. In particular, $P$ is normal in $G$. Then $P\subseteq O_p(G)\subseteq F(G)$. Since $N$ is the product of its Sylow subgroups, we deduce that $N\subseteq F(G)$.  \end{proof}

\begin{cor}
    Let\/ $K$ and\/ $L$ be nilpotent normal subgroups of a finite group\/ $G$. Then\/ $KL$ is nilpotent.
\end{cor}

\begin{proof} Both $K$ and $L$ are subgroups of $F(G)$ and so is $KL$. The conclusion follows because $F(G)$ is nilpotent.  \end{proof}

\begin{cor}\label{fitting-intersection-normal}
    Let $G$ be a finite group and $N\normal G$. Then
    \begin{enumerate}[\rm a)]
        \item $p\in\spec|N|\implies O_p(N)=O_p(G)\cap N$.
        \item $F(N)=F(G)\cap N$.
    \end{enumerate}
\end{cor}

\begin{proof}${}$
\begin{enumerate}[\rm a)]
    \item Take $p\in\spec|N|$. Given $Q\in\Syl_p(N)$ there exists $P\in\Syl_p(G)$ such that $Q=P\cap N$. Consequently, given $x\in G$ we have $Q^x=P^x\cap N\in\Syl_p(N)$ because it is a $p$-subgroup of $N$ with $|Q^x|=|Q|$. This shows that the intersection of a Sylow $p$-group of $G$ with $N$ is a Sylow $p$-subgroup of $N$ and, conversely, every Sylow $p$-subgroup of $N$ is the intersection of a Sylow $p$-group of $G$ with $N$.

    \item Firstly, $F(G)\cap N\normal G$. Secondly $F(G)\cap N$ is nilpotent because it is a subgroup of $F(G)$. From Corollary~\ref{normal-nilpotent-fitting} we obtain $F(G)\cap N\subseteq F(N)$.

    For the other inclusion observe that $O_p(N) = O_p(G)\cap N\subseteq F(G)\cap N$ by part~a). Therefore, $F(N)\subseteq F(G)\cap N$.
\end{enumerate}
\end{proof}

\begin{prop}\label{nontrivial-fitting-center}
    Let\/ $G$ be a finite group and\/ $N\normal F(G)$ nontrivial. Then $N\cap Z(F(G))\ne\gen1$. The same conclusion holds if\/ $N\subgroup G$ and\/ $N\cap F(G)\normal F(G)$ is nontrivial.
\end{prop}

\begin{proof} Replacing $N$ with $N\cap F(G)$ we may assume that $N\normal F(G)$ is nontrivial. Pick $p\in\spec|N|$. Then $N\cap O_p(G)\normal O_p(G)$. Moreover, $N\cap O_p(G)=O_p(N)$ is nontrivial. According to Theorem~\ref{nontrivial-center}, $N\cap Z(O_p(G))\ne\gen1$. The conclusion follows because $Z(O_p(G))\subseteq Z(F(G))$ [Proposition~\ref{product-center-and-maximal}].  \end{proof}

\subsection{Problems D}

\begin{probl}\label{problem-1.D.1}
    Let $Q\in\Syl_p(H)$, where $H \subgroup G$. Suppose that\/ $N_G(Q)\subseteq H$. Show that\/~$p$ does not divide $|G:H|$.
\end{probl}

\begin{solution} By Problem~\ref{problem-1.C.2}, there exists $P\in\Syl_p(G)$ such that $Q=H\cap P$. Then
$$
    N_P(Q)=N_G(Q)\cap P\subseteq H\cap P=Q \subseteq N_P(Q).
$$
We must conclude that $Q=P$, otherwise we would have $Q\varsubsetneq N_P(Q)$ by Theorem~\ref{normalizers-grow} applied to $G=P$ and $H=Q$, whose hypothesis is satisfied because $P$ is nilpotent by Corollary~\ref{p-groups-are-nilpotent}. It follows that $Q\in\Syl_p(G)$ and so $p\perp\abs{G:Q}$. Since $\abs{G:Q}=\abs{G:H}\abs{H:Q}$, the conclusion becomes evident. 

\textbf{Note.} Another way to put this result is by means of the implication
$$
    Q\in\Syl_p(H)\text{ and }N_G(Q)\subseteq H\implies Q\in\Syl_p(G).
$$
 \end{solution}


\begin{probl}
    Fix a prime\/ $p$ and suppose that a subgroup\/ $H \subseteq G$ has the property that\/ $C_G(y) \subseteq H$ for every element $y \in H$ having order\/ $p$. Show that\/~$p$ cannot divide both\/ $|H|$ and\/ $|G:H|$.
\end{probl}

\begin{solution} [Javier Burroni] 
Suppose that $p\mid|H|$. Given that $p\perp|G:H|$ iff $\Syl_p(H)\subseteq\Syl_p(G)$, it suffices to prove the inclusion. Take $Q\in\Syl_p(H)$. By Problem~\ref{problem-1.C.2} there exists $P\in\Syl_p(G)$ such that $Q=P\cap H$. Pick $y\in Q\setminus\gen1$. After replacing $y$ with its $(\ord(y)/p)$-th power, we may assume that $\ord(y)=p$. Since $y\in P$ we know that $Z(P)\subseteq C_G(y)$. Moreover, $C_G(y)\subseteq H$ because $y\in H$. As a result, $Z(P)\subseteq P\cap H=Q$. Now pick $z\in Z(P)\setminus\set1$, which is possible by Corollary~\ref{p-groups-have-center}. As before, we may assume $\ord(z)=p$. Firstly note that $P\subseteq C_G(z)$. Secondly, since $z\in Q\subseteq H$, $C_G(z)\subseteq H$. It follows that $P\subseteq H$, i.e., $Q=P$, as desired.  \end{solution}

\begin{probl}\label{problem-1.D.3}
    Let $H \subgroup G$ have the property: $H \cap H^x = \gen1$ for all $x\in G\setminus H$.
    \begin{enumerate}[\rm a)]
        \item Show that $N_G(K) \subseteq H$ for all subgroups with $\gen1\ne K \subseteq H$.
        \item Show that $H$ is a Hall subgroup of\/ $G$, i.e., $|H|\perp|G:H|$.
    \end{enumerate}
    \textbf{Note.} A subgroup $H\subgroup G$ that satisfies the hypothesis of this problem is usually referred to as a \textsl{Frobenius complement} in $G$.

\end{probl}

\begin{solution}
\begin{enumerate}[\rm a)]
    \item Take $x\in N_G(K)$. Pick $z\in K\setminus\set1$. Then $z^x\in K^x=K\subseteq H$. Since $z^x\in K^x\subseteq H^x$, it also follows that $x\notin G\setminus H$, i.e., $x\in H$.

    \item Suppose that $|H|$ and $|G:H|$ are not coprime. Pick a prime number $p$ such that $p\mid|H|$ and $p\mid|G:H|$. Take $y\in H$ with $\ord(y)=p$. Since $\gen1\ne\gen y\subseteq H$, we deduce from part~a) that $C_G(y)\subseteq H$. Now the conclusion is a direct consequence of the previous problem.
\end{enumerate}
\end{solution}


\begin{probl}
    Let\/ $G = NH$, where $\gen1 \ne N\normal G$ and\/ $N \cap H = \gen1$. Show that\/~$H$ is a Frobenius complement (see above) if, and only if, $C_N(y) = \gen1$ for all non-identity elements $y \in H$.

    \textrm{\rm Hint. If $y$ and $y^z$ lie in $H$, where $z \in N$, observe that $y^{-1}y^z \in N$.}
\end{probl}

\begin{solution} Let's first verify the hint. Take $y\in H$ and $z\in N$ with $y^z\in H$. Then,
$$
    \big(y^{-1}y^z\big)^y
        =y^zy^{-1}
        =zyz^{-1}y^{-1}
        =zz^y\in NN^y,
$$
and the conclusion follows because $N$ is normal.
\begin{description}[leftmargin=!,labelwidth=\widthof{\it if:}]
    \item[\textrm{\rm{\it if\/} part:}] Take $y^x\in H\cap H^x$ with $y
    \in H$ and $x\in G\setminus H$. By hypothesis we can write $x=\omega z$ for some $z\in N$ and $\omega\in H$. Then $(y^z)^\omega=y^x\in H$ and hence, $y^z\in H$. By the hint $y^{-1}y^z\in N$, and since it is also in $H$, we get $y^z=y$. Thus, $z\leftrightarrow y$, i.e., $z\in C_N(y)=\gen1$, unless $y=1$. Since $\omega z=x\notin H$, we must have $z\ne1$. Then $y=1$ and hence $y^x=1$.
    \item[\textrm{\rm{\it only if\/}:}]  Take $y\in H\setminus\set1$ and $z\in C_N(y)$. Then $y=y^z\in H\cap H^z$. Since $H$ is a Frobenius complement, we deduce $z\notin G\setminus H$, i.e., $z\in H$. Thus, $z\in N\cap H=\gen1$.
\end{description}
\end{solution}

\begin{probl}
    Let $H\subgroup G$, and suppose that $N_G(P)\subseteq H$ for all $p$-subgroups $P \subseteq H$, for all primes $p$. Show that\/ $H$ is a Frobenius complement in\/ $G$.

    \textrm{\rm Hint. Observe that the hypothesis is satisfied by $H \cap H^x$ for $x \in G$. If this intersection is nontrivial, consider a nontrivial Sylow subgroup $Q \subseteq H \cap H^x$ and show that $Q$ and $Q^{x^{-1}}$ are conjugate in $H$.}
\end{probl}

\begin{solution} Let's start by verifying the observation claimed in the hint. Fix an element $x\in G$ and a $p$-subgroup $P\subseteq H\cap H^x$. We have to show that, $N_G(P)\subseteq H\cap H^x$. The hypothesis implies that $N_G(P)\subseteq H$. Take $z\in N_G(P)$ and let's prove that $z^x\in H$.

By definition, $P^z=P$. Then
$$
    (P^x)^{z^x} = P^{z^xx} = P^{xz} = (P^z)^x = P^x,
$$
i.e., $z^x\in N_G(P^x)\subseteq H$ because $P^x$ is a $p$-subgroup of $H$. Given that $x$ is generic, the hint's first claim is now clear.

Let's say that a subgroup $H$ is \textsl{good\/} whenever $H$ has the property we are considering. Then, what we just proved can be restated as
$$
    H \text{ is good} \implies H\cap H^x\text{ is good}
$$
Applying this to $x^{-1}$ we get
$$
    H \text{ is good} \implies H\cap H^{x^{-1}}\text{ is good}.
$$
Put $K=H\cap H^{x^{-1}}$ and assume that $x\notin H$. Our goal is to prove that $K=\gen1$. Suppose $K\ne\gen1$ for a contradiction. Pick $p\in\spec\abs K$ and $Q\in\Syl_p(K)$. Since $K$ is good and $Q\subseteq K$ is a $p$-group, we get $N_G(Q)\subseteq K$. We claim that $Q\in\Syl_p(H)$. Otherwise $p$ must divide both $\abs{H:K}$ and $\abs K$. But this is impossible by Problem~\ref{problem-1.D.1} applied to $Q$ and $K$ because it ensures that $p\perp\abs{G:K}$ and $\abs{G:K}=\abs{G:H}\abs{H:K}$. Then $Q\in\Syl_p(H)$. It follows that $Q^x\in\Syl_p(H)$. Thus, $Q^x=Q^y$ for some $y\in H$, as hold by the hint's second claim.

In the end, $Q^{y^{-1}x}=Q$. Hence, $y^{-1}x\in N_G(Q)\subseteq H\cap H^x\subseteq H$. So $x\in H$; a contradiction.  \end{solution}

\begin{probl}\label{problem-1.D.6}
    Show that a subgroup of a nilpotent group is maximal if, and only if, it has prime index.
\end{probl}

\begin{solution}

\begin{description}%[labelwidth=\widthof{\it if:}]
    \item[\rm{\it if\/} part:] This is trivial. If $H\subseteq G$ is a subgroup with index $\abs{G:H}=p$ prime and $H\subseteq K\subseteq G$, then $\abs{G:K}\mid p$. So $K=G$, for $\abs{G:K}=1$, or $K=H$, for $\abs{G:K}=p$.
    
    \item[\rm{\it only if\/}:] Let $M$ be a (proper) maximal subgroup of the nilpotent group $G$. By Theorem~\ref{nilpotent-equivalences} $M$ is normal and by Theorem~\ref{nilpotent-center-series}, $\bar G=G/M$ has a nontrivial center. Pick $\bar z\in Z(\bar G)\setminus\gen1$. The cyclic group $\gen{\bar z}$ is normal and since $z\notin M$, it must be $\gen zM=G$, i.e., $\bar G=\gen{\bar z}$. Finally, $ord(\bar z)$ has to be prime, otherwise, there would be some subgroup between $M$ and $G$.
\end{description}
\end{solution}

\begin{probl}
    For any finite group $G$, the Frattini subgroup $\Phi(G)$ is the intersection of all maximal subgroups. Show that\/ $\Phi(G)$ is exactly the set of "useless" elements of\/ $G$, by which we mean the elements\/ $x \in G$ such that if\/ $\gen{X \cup\set x} = G$ for some subset\/ $X$ of\/ $G$, then $\gen X = G$.
\end{probl}

\begin{solution} This is nothing but Remark \ref{non-generators}.  \end{solution}

\begin{probl}\label{problem-1.D.8}
    A finite $p$-group\/ $G$ is \textsl{elementary abelian} when it is abelian and every nonidentity element has order\/ $p$. If\/ $G$ is nilpotent, show that $G/\Phi(G)$ is abelian. Furthermore, if\/ $G$ is a $p$-group, then\/ $G/\Phi(G)$ is elementary abelian.

    \textrm{\rm {\bf Note.} It is not hard to see that if $G$ is a $p$-group, then $\Phi(G)$ is the unique normal subgroup of $G$ minimal with the property that the quotient group is elementary abelian [cf.~\href{https://math.stackexchange.com/q/1577150/269050}{Frattini subgroup of a finite elementary abelian $p$-group is trivial}].}
\end{probl}

\begin{solution} Assume that $G$ is nilpotent. Fix to elements $x,y\in G$. Take a maximal subgroup $M$. Since $M$ is normal by Theorem~\ref{nilpotent-equivalences} and, according to Problem~\ref{problem-1.D.6}, $|G/M|$ is a prime number, we know that $G/M$ is cyclic, hence abelian. Let $\varphi\colon G\to G/M$ the projection onto the quotient. Then $\varphi[x,y]=[\varphi(x),\varphi(y)]=1$, i.e., $[x,y]\in\ker\varphi = M$. It follows that $[x,y]\in\Phi(G)$. Thus $G/\Phi(G)$ is abelian and the proof of the first part is complete.

Assume, for the second, that $G$ is a $p$-group with $\Phi(G)=\gen1$. By the first part,~$G$ is abelian. Suppose that there exists $x\in G\setminus\set1$ such that $\ord(x)=p^d$ with $d>1$.

Define $y=x^{p^{d-1}}$. We claim that $y$ is in every maximal subgroup $M$. Suppose otherwise. Then $\gen y M=G$ and we can write
$$
    x = y^jz = (x^j)^{p^{d-1}}z
$$
for some integer $j$ and some $z\in M$. Then
$$
    x^p=(x^j)^{p^d}z^p=\big(x^{p^{d}}\big)^jz^p=z^p\in M.
$$
In particular, $y=x^{p^{d-1}}=(x^p)^{p^{d-2}}\in M$. It follows that $y\in\Phi(G)=\gen1$, which is impossible because $\ord(y)=p$. 

\bigskip

Regarding the note after the problem, if $G$ is a $p$-group and $N$ a normal subgroup such that $\bar G=G/N$ is elementary abelian, we can inductively define a sequence $x_1,\dots,x_k$ of elements such that $\gen{\bar x_i}\cap\gen{\bar x_j}=\gen1$ for $i\ne j$, because any given sequence can be extended taking the next element in $\bar G\setminus\gen{\bar x_1}\cdots\gen{\bar x_k}$ whenever the running product hasn't reached~$\bar G$. By Theorem~\ref{direct-product}, we deduce that~$\bar G$ is the inner direct product
$$
    \bar G = \gen{\bar x_1}\cdots\gen{\bar x_n}.
$$
In particular, $\bar M_i=\gen{\bar x_1}\cdots\overbrace{\gen1}^i\cdots\gen{\bar x_n}$ is maximal because $\ord(\bar x_i)=p$. Moreover,
$$
    \gen1 = \bigcap_{i=1}^n\bar M_i \supseteq\Phi(\bar G).
$$
Thus, if $\varphi\colon G\to \bar G$ is the projection onto the quotient and $M_i=\varphi^{-1}(\bar M_i)$ is the preimage of $\bar M_i$, then $M_i$ is maximal and
$$
    \Phi(G)\subseteq\bigcap_{i=1}^nM_i=N.
$$

 \end{solution}

\begin{probl}\label{problem-1.D.9}
    If $G$ is a noncyclic $p$-group, show that $|G:\Phi(G)|\ge p^2$, and deduce that a group of order $p^2$ must be either cyclic or elementary abelian.
\end{probl}

\begin{solution} As we saw in the previous problem $G/\Phi(G)$ is an inner direct product of cyclic groups of order $p$. If $G$ is not cyclic, the product has at least two factors and so $\abs{G:\Phi(G)}=\abs{G/\Phi(G)}\ge p^2$.

If $\abs G=p^2$ and $G$ is not cyclic, it follows that $\Phi(G)=\gen1$, which implies that $G$ is elementary abelian.  \end{solution}

\begin{probl}
    Let $A$ be maximal among the abelian normal subgroups of a $p$-group\/ $G$. Show that $A=C_G(A)$, and deduce that $|G : A|$ divides $(|A| - 1)!$.

    \textrm{\rm Hint. If $A\varsubsetneq C_G(A)$, apply Lemma~\ref{subgroup-of-index-p}.}
\end{probl}

\begin{solution} Let's start by observing that $C_G(A)$ is normal. Take $z\in C_G(A)$, $x\in G$ and $a\in A$. Then
\begin{align*}
     z^xa &= xzx^{-1}a\\
        &=xz(x^{-1}ax)x^{-1}\\
        &= x(x^{-1}ax)zx^{-1} &&\text{; }x^{-1}ax\in A\leftrightarrow z\\
        &= az^x.
\end{align*}
Let's suppose for a contradiction that $A\varsubsetneq C_G(A)$ and apply Lemma~\ref{subgroup-of-index-p}, as suggested, to find $L\normal G$ such that
$$
    A\subseteq L\subseteq C_G(A)\quad\text{with}\quad \abs{L:A}=p.
$$
Then $A\normal L$, and $L/A\cong\Z_p$. In particular $L=A\gen y$ for some (actually, any) $y\in L\setminus A$, which proves that $L$ is abelian, since $y\in C_G(A)$. This contradicts the fact that $A$ is maximal among the abelian normal subgroups of $G$. In consequence, $A=C_G(A)$, as desired.

Now consider the map
\begin{align*}
    \sigma\colon G&\to\Sym(A)\\
    x&\mapsto\sigma_x: (a\mapsto a^x).
\end{align*}
Given that
\begin{align*}
    x\in\ker(\sigma) &\iff \sigma_x=\id_A\\
        &\iff a^x=a \text{ for all }a\in A\\
        &\iff x\in C_G(A)\\
        &\iff x\in A,
\end{align*}
we deduce that
$$
    \ker(\sigma) = A.
$$
Therefore, the induced morphism $\bar\sigma\colon G/A\to\Sym(A)$ is a monomorphism. Since $\sigma_x(1)=1$,
 $\im(\bar\sigma)$ is included in the subgroup $F$ of $\Sym(A)$ of bijections that fix~$1$. Clearly, $\abs F=(\abs A-1)!$ and $\im(\bar\sigma)$ is a subgroup of $F$, isomorphic to $G/A$. Thus, the order of $G/A$ must divide $(\abs A-1)!$  \end{solution}

 \begin{probl}
     Let\/ $n$ be the maximum of the orders of the abelian subgroups of a finite group\/ $G$. Show that\/ $|G|$ divides\/ $n!$.
    
    \textrm{\rm Hint. Show that for each prime $p$, the order of a Sylow $p$-subgroup of $G$ divides~$n$!}

    \textrm{\rm {\bf Note.} There exist infinite groups in which the abelian subgroups have bounded order, so finiteness is essential here.}
 \end{probl}

\begin{solution} Let's say that a group is \textsl{good} if it satisfies the conclusion of the problem. Take $p\in\spec|G|$, write $|G|=p^em$ with $p\perp m$ and pick $P\in\Syl_p(G)$. Suppose for a moment that $P$ is good. If $r$ is the maximum order of an abelian subgroup of $P$, then $p^e\mid r!$. The definition of $n$ implies $r\le n$, i.e., $r!\mid n!$, which implies $p^e\mid n!$. As a consequence, the problem can be reduced to the case where~$G$ is a $p$-group.

Let $A$ be maximal among the abelian normal subgroups of $G$. By the previous problem $|G|\mid |A|!$. Since $|A|\le n$, the conclusion follows.  \end{solution}

\begin{probl}
    Let\/ $p$ be a prime dividing the order of a group\/ $G$. Show that the number of elements of order\/ $p$ in\/ $G$ is congruent to\/ $-1$ modulo\/ $p$.
\end{probl}

\begin{solution} This is nothing by Problem \ref{cauchy}. It can also be derived from Corollary~\ref{p-subgroups} because subgroups of order $p$ are equal or have trivial intersection and each of them has $p-1$ elements of order $p$. Therefore, if $n$ is the number of such subgroups, then $n\equiv1\pmod p$ and there are a total of $n(p-1)$ elements of order $p$ in $G$. Note however that it isn't clear that the proof of Corollary~\ref{p-subgroups} we gave is independent from this fact, known as Cauchy's Theorem.  \end{solution}

\begin{probl}\label{problem-1.D.13}
    If $Z \subseteq Z(G)$ and $G/Z$ is nilpotent, show that\/ $G$ is nilpotent.
\end{probl}

\begin{solution} Let $\gen1\subseteq \bar N_1\subseteq\cdots\subseteq \bar N_r=G/Z$ be the upper central series of $G/Z$ (see Proposition~\ref{nilpotent-center-series}) and $\varphi\colon G\to G/Z$ the projection onto the quotient. Then $N_i=\varphi^{-1}(\bar N_i)$ is normal for $1\le i\le r$. Take $x\in G$ and consider the commutator map $\ct_x$. If $\bar x=\varphi(x)$, then
$$
    \varphi(\ct_x(N_i))=\ct_{\bar x}(\bar N_i)\subseteq\bar N_{i-1},
$$
which implies
$$
    \ct_x(N_i)\subseteq N_{i-1}Z\subseteq N_{i-1}
$$
for $i>1$ because $Z\subseteq N_i$ for those indexes. Since $Z\subseteq Z(G)$, we can extend the sequence to the left with $Z\subseteq N_1$ and get a normal series for $G$. The conclusion now follows from Proposition~\ref{ad-nilpotent-test} because $\ct_x(N_1)\subseteq Z\subseteq Z(G)$.  \end{solution}

\begin{probl}
    Show that the Frattini subgroup\/ $\Phi(G)$ of a finite group\/ $G$ is nilpotent.

    \textrm{\rm Hint. Apply the Frattini argument. The proof here is somewhat similar to the proof that (c) implies (d) in Theorem~\ref{nilpotent-equivalences}}.
    
    \textrm{\rm {\bf Note.} This problem shows that $\Phi(G) \subseteq F(G)$.}
\end{probl}

\begin{solution} Let $Q\in\Syl_p(\Phi(G))$. The Frattini Argument~(\ref{frattini-argument}) implies that $G=\Phi(G)N_G(Q)$. Then $N_G(Q)=G$~(\ref{frattini}), i.e., $Q\normal G$. In particular, $Q\normal\Phi(G)$. The conclusion is now a direct consequence of Theorem~\ref{nilpotent-equivalences}.

The note follows from Corollary \ref{normal-nilpotent-fitting}.  \end{solution}

\begin{probl}\label{problem-1.D.15}
    Suppose that\/ $\Phi(G)\subseteq N\normal G$ and that\/ $N/\Phi(G)$ is nilpotent. Show that\/ $N$ is nilpotent. In particular, if\/ $G/\Phi(G)$ is nilpotent, then\/ $G$ is nilpotent.

    \textrm{\rm{\bf Note.} This generalizes the previous problem by setting $N = \Phi(G)$. Note that this problem proves that $F(G/\Phi(G)) = F(G)/\Phi(G)$.}
\end{probl}

\begin{solution} Without loss of generality we may assume that $\Phi(G)\varsubsetneq N$, otherwise the result would be a direct consequence of the previous problem. Pick a maximal subgroup $K\subgroup N$ such that $\Phi(G)\subseteq K$. Put $\bar K=K/\Phi(G)$ and $\bar N=N/\Phi(G)$. By Problem~\ref{problem-1.D.6}, $\bar K\normal\bar N$ and $p=|\bar N:\bar K|$ is prime. Therefore, $K\normal N$ and $|N:K|=p$ is prime. In particular, $N/K$ is cyclic, hence abelian. Moreover, since $\bar K$ is nilpotent [cf.~Corollary~\ref{nilpotent-subgroups-and-quotients}], by induction on $|N|$ we may assume that $K$ is nilpotent. Given that $K\subseteq N$ will extend any central series for $K$ to a central series for $N$.  \end{solution}

\textbf{Note.} The equation $F(G/\Phi(G)) = F(G)/\Phi(G)$ holds because on both sides we have the largest normal nilpotent subgroup of $G/\Phi(G)$, since such a group would correspond to a normal subgroup $N\normal G$ in the hypothesis of the problem.

\textbf{Note.} For a different solution see \href{https://math.stackexchange.com/a/209225/269050}{\it Why is a normal subgroup containing $\Phi(G)$ with a nilpotent factor group nilpotent?}


\begin{probl}\label{problem-2.D.16}
    Let\/ $N\normal G$, where\/ $G$ is finite. Show that\/ $\Phi(N) \subseteq \Phi(G)$.

    \textrm{\rm Hint. If some maximal subgroup $M$ fails to contain $\Phi(N)$, then $\Phi(N)M= G$, and it follows that $N = \Phi(N)(N\cap M)$.}
\end{probl}

\begin{solution} Suppose for a contradiction that $\Phi(N)\not\subseteq\Phi(G)$ and let $M$ be a proper maximal subgroup of $G$ such that $\Phi(N)\not\subseteq M$. Then $\Phi(N)M=G$. Given that $\Phi(N)\subseteq N$, we deduce that $N=\Phi(N)(M\cap N)$. By Proposition~\ref{frattini}~c) we get $M\cap N=N$, i.e., $N\subseteq M$. A contradiction because $\Phi(N)\subseteq N$.  \end{solution}

\begin{probl}
    Let $N\normal G$, where $N$ is nilpotent and\/ $G/N'$ is nilpotent. Prove that\/ $G$ is nilpotent.
    
    \textrm{\rm Hint. The derived subgroup $N'$ is contained in $\Phi(N)$ by Problem~\ref{problem-1.D.8}.}

    \textrm{\rm {\bf Note.} If we weakened the assumption that $G/N'$ is nilpotent and assumed instead that $G/N$ is nilpotent, it would not follow that $G$ is necessarily nilpotent.}
\end{probl}

\begin{solution} Recall from Remark~\ref{derived-group} that $N'$ is normal and that's why the quotient $G/N'$ makes sense.

Let $\varphi\colon N\to N/\Phi(N)$ be the projection onto the quotient. Since $N$ is nilpotent, by Problem~\ref{problem-1.D.8}, $N/\Phi(N)$ is abelian. In particular, given $x,y\in N$, $\varphi[x,y]=1$, i.e., $[x,y]\in\Phi(N)$, i.e., $N'\subseteq\Phi(N)$.

We can now consider the s.e.s.
$$
    1\to \Phi(N)/N'\to G/N'\to G/\Phi(N)\to 1,
$$
which implies that $G/\Phi(N)$ is nilpotent by Corollary \ref{nilpotent-subgroups-and-quotients}. In consequence, $G$ is nilpotent by Problem~\ref{problem-1.D.15}.

\end{solution}

\begin{probl}
    Show that $F(G/Z(G)) = F(G)/Z(G)$ for all finite groups~$G$.
\end{probl}

\begin{solution} Let's first observe that, in fact, $Z(G)\subseteq F(G)$. Indeed. Since every $p$-Sylow group $P$ is nilpotent, it has a nontrivial center $Z(P)$ which naturally includes $Z(G)$. Thus, $Z(G)\subseteq O_p(G)$ and so $Z(G)\subseteq F(G)$.

Since $F(G)$ is nilpotent (Corollary \ref{normal-nilpotent-fitting}), $F(G)/Z(G)$ is nilpotent too (Corollary~\ref{nilpotent-subgroups-and-quotients}). It is normal in $G/Z(G)$ because $F(G)\normal G$. Therefore, the RHS is included into the LHS.

To verify that equality is attained, we have to show that the RHS is the largest normal nilpotent subgroup of $G/Z(G)$. Let $\bar N$ be such a subgroup of $G/Z(G)$. Let $N$ be the preimage of $\bar N$ in $G$ via the projection onto the quotient. By Remark~\ref{preimage-of-normal}, $N$ is normal in $G$.

It remains to be seen that $N$ is nilpotent because, in that case, we will have $N\subseteq F(G)$. Clearly, $Z(G)\subseteq Z(N)$. In particular, the identity induces an epimorphism $N/Z(G)\to N/Z(N)$. Since $N/Z(G)=\bar N$ is nilpotent, according to Corollary~\ref{nilpotent-subgroups-and-quotients}, $N/Z(N)$ is nilpotent too. Hence, the result is a direct consequence of Problem~\ref{problem-1.D.13} applied to $N$ and $Z(N)$.  \end{solution}

\begin{probl}\label{problem-1.D.19}
    Let $F = F(G)$, where $G$ is an arbitrary finite group, and let $C = C_G(F)$. Show that $C/C\cap F$ has no nontrivial abelian normal subgroup.

    \textrm{\rm Hint. Observe that $F(C)\normal G$.}
\end{probl}

\begin{solution} Note that $F(C)\normal G$ because $F(C)\ch C\normal G$, which enables us to invoke Lemma~\ref{normal-transitivity}. In fact, $C\ch G$. To see this take an automorphism $\sigma\colon G\to G$. Since $F\ch G$, $\sigma(F)=F$. Moreover, given $z\in C$ and $y\in F$, we can write
$$
    \sigma(z)y=\sigma(z\sigma^{-1}(y))=\sigma(\sigma^{-1}(y)z)=y\sigma(z).
$$
Accordingly, $\sigma(z)\in C$ and so $C\ch G$.

But $F(C)$ is nilpotent because it is a fitting group. So, $F(C)\subseteq F(G)=F$ by Corollary~\ref{normal-nilpotent-fitting}. It follows that $F(C)\subseteq C\cap F$. Since $C\cap F=Z(F)$, we deduce that $F(C)\subseteq Z(F)$. Thus,
$$
    Z(C)\subseteq F(C)\subseteq Z(F).
$$
The inclusion $Z(F)\subseteq Z(C)$ is easily seen. Indeed, given $z\in Z(F)=C\cap F$, we first have $z\in C$ and second, for any $y\in C=C_G(F)$, it is $zy=yz$ because $z\in F$. Thus, $z\leftrightarrow C$. i.e., $z\in Z(C)$. Therefore,
$$
    Z(C)= F(C)=Z(F).
$$
Let $A \supseteq Z(F)$ be a subgroup of $C$ such that $\bar A$ is abelian normal in $C/Z(F)$. By Remark~\ref{preimage-of-normal}, we know that $A\normal C$. In sum,
$$
    Z(C)= F(C) = Z(F)\subseteq A\normal C,
$$
which, by Problem \ref{problem-1.D.13} applied to $A$ and $Z(F)\subseteq Z(A)$, shows that $A$ is nilpotent because $A/Z(F)=\bar A$ is abelian, hence nilpotent. By Corollary~\ref{normal-nilpotent-fitting}, it follows that $A\subseteq F(C)=Z(F)$. Thus, $\bar A=\gen1$, as desired.  \end{solution}

\section{Nonsimplicity Theorems}

\begin{defn}\label{composition-series}
    Given a group $G$, a \textsl{composition series} for $G$ is a sequence
    $$
        \gen1=N_0\normal N_1\normal\cdots\normal N_r=G,
    $$
    where the so called \textsl{composition factors} $N_i/N_{i-1}$ are simple.
\end{defn}

\begin{thm}\label{thm-pq}
    Let $\abs G=pq$, where $q<p$ are primes. Then $G$ has a normal Sylow $p$-subgroup. Also, $G$ is cyclic unless $q$ divides $p-1$.
\end{thm}

\begin{proof} By Problem \ref{problem-1.A.9}, $G$ contains a unique subgroup $P$ of order $p$, which in this case is a Sylow $p$-group. It is clearly normal because of Theorem~\ref{sylow-c}.

Since $n_q(G)\mid p$ by Remark~\ref{p'-part} then $n_q(G)=1$ or $n_q(G)=p$. Therefore, if we assume that $q\perp p-1$, we must have $n_q(G)=1$ because $n_q(G)\equiv1\pmod q$, in accordance with Corollary~\ref{p|n_p-1}. Thus, there is only one Sylow $q$-group $Q$ of $G$. Since $\abs{P\cup Q}=p+q-1<2p\le pq=\abs G$, there must be an element $x\in G\setminus P\cup Q$. As $\ord(x)\mid pq$ and $\ord(x)$ is not $p$ or $q$, we deduce that $\ord(x)=pq=\abs G$. Hence, $G=\gen x$ is cyclic.  \end{proof}

\begin{thm}\label{thm-p^2q}
    Let\/ $\abs G = p^2q$, where $p$ and $q$ are primes. Then $G$ has either a normal Sylow $p$-subgroup or a normal Sylow $q$-subgroup.
\end{thm}

\begin{proof} Suppose for a contradiction that $n_p=n_p(G)>1$ and $n_q=n_q(G)>1$. By Corollary~\ref{p|n_p-1}, $n_p>p$ and $n_q>q$. Since $n_p\mid q$ by Remark~\ref{p'-part}, we deduce that $n_p=q$. Similarly, $n_q=p$ or $n_q=p^2$. But $n_q>q=n_p>p$ and so $n_q=p^2$.

The intersection of any two Sylow $q$-groups must be trivial. And since each of these groups has $q-1$ elements of order $q$, it follows that there are $p^2(q-1)$ elements of order $q$ in $G$. Consider the sets
$$
    Y = \bigcup\Syl_q(G)\setminus\set1\quad\mathrm{and}\quad X=G\setminus Y.
$$
As we just saw, $\abs Y=p^2(q-1)$. Therefore, $\abs X=p^2q-p^2(q-1)=p^2$ and all elements in $G$ with order divisible by $p$ must belong to $X$. Thus, if $P\in\Syl_p(G)$, we must have $P\subseteq X$. By cardinality, $P=X$, and so $X$ is the only one element in $\Syl_p(G)$, a contradiction with the fact that $n_p>1$.  \end{proof}


\begin{thm}\label{thm-p^3q}
    Let $|G|=p^3q$, where $p$ and $q$ are primes. Then $G$ has either a normal Sylow $p$-subgroup or a normal Sylow $q$-subgroup, except when $|G|=24$.
\end{thm}

\begin{proof} We can repeat the same arguments we gave in the proof of the previous theorem, except that this time we will arrive at two possibilities: (1)~$n_q=p^2$ or (2)~$n_q=p^3$.

In case (2), a similar reasoning to the one of the last proof, allows us to conclude that $n_p$ must be $1$. Therefore, we are left with case~(1) where $n_q=p^2$. Since $n_q\equiv1\pmod q$, it follows that $q\mid(p+1)(p-1)$. Thus $q\mid p+1$ or $q\mid p-1$. In both cases $q\le p+1$. But, as we saw in the previous proof, $p<q$. Therefore, $q=p+1$, i.e., $p=2$ and $q=3$, as wanted.  \end{proof}

\begin{xmpl}
    Let's analyze the symmetric group $S_4$ so to better understand its composition. Consider the following facts:
    \begin{enumerate}[\rm 1.]
        \item It contains $\binom42=6$ transpositions.
        \item The remainder elements of order $2$ have the pattern $(xy)(\cdot\;\cdot)$, where the second transposition is determined by the first by complement. There are $\binom42/2=3$ of them.
        \item From facts {\rm1} and {\rm2}, there are $9$ elements of order $2$.
        \item It contains $\binom43=4$ cycles of order $3$.
        \item The remainder elements of order $3$ are products of two transpositions with an element in common, i.e., with the pattern $(xy)(yz)$. There are $\binom43=4$ of them.
        \item From facts {\rm4} and {\rm5}, there are $8$ elements of order $3$.
        \item There are $3!=6$ elements of order $4$, which are cycles.
        \item If $P\in\Syl_2(S_4)$, then $\abs P=8$. It follows from fact~{\rm3} that $n_2>1$.
        \item Since $n_2\mid 3$, it follows from fact~{\rm8} that $n_2=3$.
        \item If $Q\in\Syl_3(S_4)$, then $\abs Q=3$. By fact~{\rm6}, we deduce that $n_3>1$.
        \item Every subgroup of order $3$ must follow the pattern $\set{1,x,x^{-1}}$, for some\/ $x$ of order\/~$3$. By fact~{\rm6}, there are\/ $8$ possibilities for\/ $x$ and\/ $x^{-1}$, which implies that\/ $n_3=4$.
        \item $A_4\normal S_4$ and $\abs{A_4}=12$.
    \end{enumerate}
\end{xmpl}

\begin{thm}\label{thm-24}
    Let\/ $G$ be a group of order $24$ and suppose\/ $n_2(G)>1$ and\/ $n_3(G)>1$. Then $G\cong S_4$.
\end{thm}

\begin{proof} We know that $n_3=n_3(G)>3$, $n_3\equiv1\pmod3$ and that $n_3\mid 8$. Therefore $n_3=4$. Take $P\in\Syl_3(G)$ and $H=N_G(P)$. By Corollary~\ref{n_p-is-index}, $\abs{G:H}=4$. In particular, $\abs H=6$.

Let $N=\Core_G(H)$. By Theorem~\ref{largest-normal-subgroup} we can identify $G/N$ with a subgroup of $S_4$. Thus, to complete the proof, it is enough to show that $N=\gen1$.

Since $N\subseteq H=N_G(P)$, we see that $P\normal NP$. But $P\in\Syl_3(NP)$ and so $P\ch NP$. And given that $P$ is not normal in $G$, we conclude by Lemma~\ref{normal-transitivity} that $NP$ cannot be normal in $G$. But $NP/N$ is a Sylow $3$-subgroup of $G/N$ because
$$
    \abs{NP/N} = \frac{\abs P}{\abs{P\cap N}},
$$
which is a power of $3$. It follows that $n_3(G/N)>1$. In particular, $G/N$ is not a $2$-group and so $\abs N$ is not divisible by~$3$.

Since $\abs N\mid\abs H=6$, we must have $\abs N=1$ or $\abs N=2$. Therefore, we only need to show that $\abs N\ne2$. Suppose it is. Then $\abs{G/N}=12$ and we can apply Theorem~\ref{thm-p^2q} to deduce that $n_2(G/N)=1$ or $n_3(G/N)=1$. But we just established that $n_3(G/N)>1$, so we can write $n_2(G/N)=1$. Thus, $\Syl_2(G/N)=\set{Q/N}$ which implies that $Q$ is a normal Sylow $2$-subgroup of $G$. But this is impossible because $n_2(G)>1$ by hypothesis.  \end{proof}

\begin{lem}\label{index-2-is-normal}
    Every subgroup of index $2$ is normal.
\end{lem}

\begin{proof} {\rm[cf.~Exercise~\ref{exr:index-2-is-normal}]} Let $H$ be a subgroup of $G$ with $\abs{G:H}=2$. To show that $H$ is normal it suffices to show that $H^\omega\subseteq H$ for any $\omega\in G\setminus H$. Since $\omega H\ne H$, we may write $G=H\cup\omega H$. Now suppose that there is an element $\zeta\in H$ such that $\zeta^\omega\notin H$. Then
$$
    \zeta^\omega\notin H\implies\zeta^\omega\in\omega H
        \implies\zeta\omega^{-1}\in H\implies \omega\in H,\text{ contradiction.}
$$
 \end{proof}

\begin{lem}
    Let\/ $G$ act on a finite set $X$, and suppose that some element of\/~$G$ acts ``oddly''. Then $G$ has a normal subgroup of index $2$.

    \textrm{\rm{\bf Note.} In this context, ``oddly'' means that the permutation on $X$ associated with the element via $G\to\Sym(X)$ is odd. Correspondingly, when the image lies inside $\Alt(X)$, we say that the element acts ``evenly''.}
\end{lem}

\begin{proof} Let $\omega\in G$ be an element for which the permutation $x\mapsto\omega\cdot x$ is odd. Let $H$ be the subset of $G$ that consists of all elements acting evenly on $X$. Then $H$ is a subgroup of $G$ because it contains $1$ and is clearly closed under multiplication. The complement $C=G\setminus H$ of $H$ isn't empty because $\omega\notin H$. We claim that $C\subseteq \omega^{-1}H$. To see this take $\sigma\in C$. Since $\sigma$ induces an odd permutation on $X$, $\omega\cdot\sigma$ induces an even permutation. Therefore, $\omega\cdot\sigma\in H$, i.e., $\sigma\in\omega^{-1}H$. Conversely, every element of $\omega^{-1}H$ belongs in $C$ because it acts oddly on $X$. This means that $C$ and $H$ are the only left cosets of $H$, which proves that $\abs{G: H}=2$. The normality of $H$ follows from the previous lemma.  \end{proof}

\begin{thm}\label{thm-2odd}
    Suppose that\/ $\abs G=2n$, where $n$ is odd. Then\/ $G$ has a normal subgroup of index $2$.
\end{thm}

\begin{proof}${}$

By Cauchy Theorem [cf.~Problem~\ref{cauchy}], there exists $\tau\in G$ with $\ord(\tau)=2$. The permutation induced by $\tau$ on $G$ can be decomposed in transpositions of the form $(\omega,\tau\omega)$. Since there are $n$ of them, $\tau$ acts oddly on $G$ [cf.~Lemma~\ref{sign-of-permutation}] and the conclusion follows from the previous lemma.  \end{proof}

\begin{xmpl}\label{A5-elements}
    The $5!/2=60$ elements of $A_5$ can be grouped as follows:
    \begin{enumerate}[-]
        \item {\rm identity}
        \item {\rm $3$-cycles:} there are $20$: $5\cdot4\cdot3$ with the pattern $(xyz)$, divided by $3$ because $(xyz)=(zxy)=(yzx)$.
        \item {\rm $5$-cycles:} there are $4!=24$ of these.
        \item {\rm disjoint $2$-cycles:} there are $15$: $\binom52\binom32$ with the pattern $(xy)(zw)$ divided by $2$ because $(xy)(zw)=(zw)(xy)$.
    \end{enumerate}
\end{xmpl}

\begin{lem}\label{order-perp-index}
    If\/ $N$ is a normal subgroup of\/ $G$, then $N$ contains every element which has order coprime to $|G:N|$.
\end{lem}

\begin{proof} Let $x\in G$ be an element with order $m\perp\abs{G:N}$. The image $\bar x$ of $x$ in the quotient $G/N$, has an order that divides both $m$ and $\abs{G/N}=\abs{G:N}$, i.e., order~$1$.  \end{proof}

\begin{lem}\label{normal-order2}
    Suppose $N\normal G$ and $|N| = 2$. Then $N$ is central in $G$.
\end{lem}

\begin{proof} By hypothesis, $N=\set{1,y}$, with $y\ne1$ and $y^2=1$. Take $x\in G$. Then $[x,y]=xyx^{-1}y^{-1}$. Since $N$ is normal, $xyx^{-1}\in N=\set{1,y}$. But $xyx^{-1}\ne1$ because $y\ne1$. Thus $[x,y]=yy^{-1}=1$.  \end{proof}

\begin{lem}\label{Z(An)=1}
    If $n\ge4$, $Z(A_n)=\gen1$.
\end{lem}

\begin{proof} Take $\zeta\in Z(A_n)$, $\zeta\ne1$. Decompose $\zeta$ in disjoint cycles and consider the longest one. The length of this cycle needs to be at least $3$. Say that $\zeta(i)=j$ and $\zeta(j)=k$. Define $\sigma=(i\,j\,k)$. Then
$$
    \zeta(k)=\zeta\sigma(j)=\sigma\zeta(j)=\sigma(k)=i,
$$
which implies that longest cycle of $\zeta$ has length $3$. Now pick $\ell\notin\set{i,j,k}$ and write $\rho=(i\,j\,\ell)$. We have
$$
    \sigma\rho(i)=\sigma(j) = k \ne \ell = \rho(j)=\rho\sigma(i),
$$
which contradicts the fact that $\zeta$ is central.  \end{proof}

\begin{defn}
    A group is \textsl{simple} if it has no nontrivial proper normal subgroups.
\end{defn}

\begin{prop}\label{A_5-is-simple}
    The alternate group $A_5$ is simple.
\end{prop}

\begin{proof} {[cf.~\href{https://math.stackexchange.com/a/329293/269050}{Alternative proofs that  $A_5$ is simple}]}

Suppose toward a contradiction that $A_5$ has proper normal subgroup $N$. Put $n=\abs N$. Note that
$$
    n\in\set{2, 3, 4, 5, 6, 10, 12, 15, 20, 30}.
$$
Firstly note that
\begin{enumerate}[$n\ne$]
    \item $2$ by Lemma~\ref{normal-order2} and Lemma \ref{Z(An)=1}.
\end{enumerate}
Now, by Lemma \ref{order-perp-index} and Example \ref{A5-elements}
\begin{enumerate}[$n\ne$]
    \item $3$ because $N$ would include $20$ $3$-cycles.
    \item $4$ because $N$ would include $15$ disjoint $2$-cycles.
    \item $5$ because $N$ would include $24$ $5$-cycles.
    \item $6$ because $N$ would include $20$ $3$-cycles.
    \item $10$ because $N$ would include $24$ $5$-cycles.
    \item $12$ because $N$ would include $20$ $3$-cycles.
    \item $15$ because $N$ would include $20$ $3$-cycles.
    \item $20$ because $N$ would include $24$ $5$-cycles.
    \item $30$ because $N$ would include $24$ $5$-cycles and $20$ $3$-cycles.
\end{enumerate}
 \end{proof}

\begin{prop}
    Every simple group of order less than\/ $60$ is abelian.
\end{prop}

\begin{proof} Suppose that $G$ is a nonabelian simple group with $n=\abs G<60$. Consider the following facts lead to contradiction:
\begin{enumerate}[\rm 1.]
    \item Since $Z(G)$ is normal (it is characteristic), we must have $Z(G)=\gen1$. Therefore, $n$ is not a power of a prime.
    \item By Theorem~\ref{thm-pq}, $n$ is not the product of two primes.
    \item By Theorem \ref{thm-p^2q}, $n$ is not of the form $p^2q$.
    \item It is easy to check that all odd numbers below $60$ are powers of a prime or have the form $p^2q$ for $p$ and $q$ primes. Thus, $n$ must be even.
    \item So far $n=2m$. By Theorem \ref{thm-2odd}, we may write $n=2^2\cdot m$. Of course, $m\le14$.
    \item By fact {\rm1}, $m$ is not a power of $2$, i.e., $m\in\set{3, 5, 6, 7, 9, 10, 11, 12, 13, 14}$.
    \item By fact {\rm3}, $m$ is not prime, i.e., $m\in\set{6,9,10,12}$.
    \item By Theorem \ref{thm-p^3q}, $m$ is not $2\cdot p$, with $p$ prime, i.e., $m\in\set{9,12}$.
    \item By fact {\rm8}, $n\in\set{2^2\cdot 3^2,\;2^4\cdot3}$.
    \item If $n=2^2\cdot3^2$, take $P\in\Syl_3(G)$. Then $\abs{G:P}=4$. By the $n!$~theorem~\ref{n!-theorem}, there exists $N\normal G$, with $N\subseteq P$ and $\abs{G:N}\mid4!=24$. Since $G$ is simple, $N=\gen1$ and $\abs{G:N}=36$, impossible.
    \item If $n=2^4\cdot3$, take $P\in\Syl_2(G)$. Then $\abs{G:P}=3$ and the same theorem as in fact~{\rm10} shows the impossibility of this case.
\end{enumerate}
 \end{proof}

\begin{thm}
    Suppose that\/ $G = p^eq$, where $p$ and $q$ are primes and $e>0$. Then $G$ is not simple.
\end{thm}

\begin{proof} Suppose that $n_p=n_p(G)>1$. The following facts lead to a contradiction:
\begin{enumerate}[1.]
    \item Since $n_p\mid q$, $n_p=q$.
    \item Pick $P$ and $Q$ in $\Syl_p(G)$ so that $\abs{P\cap Q}$ is as large as possible. Let $H=P\cap Q$. There are two cases: $\abs H=1$ or $\abs H>1$.
    \item If $\abs H=1$, all Sylow $p$ subgroups intersect trivially. By fact~1, there are $q(p^e-1)$ nontrivial elements whose orders are powers of $p$. The remanding $q$ have orders divisible by $q$. Therefore, there can only be one element in $\Syl_q(G)$, namely the one that consists of the $q$ elements whose order divides~$q$. Such a $q$-subgroup is normal, and we are done.
    \item We can assume that $\abs H>1$. Put $L=N_G(H)$. Since $N_P(H)=L\cap P$ and $H\varsubsetneq P$ [cf.~Corollary~\ref{p-groups-are-nilpotent} and Theorem~\ref{normalizers-grow}], we can write $H\varsubsetneq L\cap P$. Similarly, $H\varsubsetneq L\cap Q$.
    \item\label{Lnotp} $L$ is not a $p$-group. Otherwise, there would exist $R\in\Syl_p(G)$ such that $L\subseteq R$. Then $R\cap P\supseteq L\cap P\varsupsetneq H$, in contradiction with fact~2.
    \item $q\mid\abs L$ by fact \ref*{Lnotp} Take $S\in\Syl_q(L)\subseteq\Syl_q(G)$. Then $P\cap S=\gen1$ because their orders are coprime. In particular, $\abs{PS}=p^eq=\abs G$, which implies that $G=PS$.
    \item If $x\in G$, $H\subseteq P^x$. Indeed, write $x=zy$ with $y\in P$ and $z\in S\subseteq L$. Then $P^x=P^{zy}=(P^y)^z=P^z\supseteq H^z=H$.
    \item It follows that $H\subseteq O_p(G)$ and so $O_p(G)$ is normal and nontrivial.
\end{enumerate}
 \end{proof}

\subsection{Problems E}

\begin{probl}
    Let\/ $G$ be a group of order $p^2q^2$, where $p > q$ are primes. Prove that $n_p(G) = 1$ unless $|G| = 36$.

    \textrm{\rm\textbf{Note.} If $|G|=36$, then a Sylow $3$-subgroup really can fail to be normal, but in that case, one can show that a Sylow $2$-subgroup of $G$ is normal.}
\end{probl}

\begin{solution} Suppose that $n_p=n_p(G)>1$. Since $n_p\equiv1\pmod p$, it must be $n_p>p$. Since $n_p\mid q^2$ and $n_p>p>q$, we must have $n_p=q^2$. Therefore, $q^2\equiv1\pmod p$, i.e., 
$$
    p\mid(q-1)(q+1).
$$
Since $p>q$, it must be $p\mid q+1$, which may only happen when $q=2$ and $p=3$. hence $\abs G=36$ as wanted.  \end{solution}

\textbf{Note.} The very same argument holds valid if $|G|=p^eq^2$ whenever $e>0$ and $(p,q)\ne(3,2)$.


\begin{probl}
    Let\/ $G$ be a group with\/ $\abs G = pqr$, where\/ $p<q<r$ are primes. Show that\/ $n_r(G)=1$.

    \textrm{\rm Hint. Otherwise, show by counting elements that a Sylow $q$-subgroup must be normal and consider the factor group, of order $pr$.}

    \textrm{\rm\textbf{Note.} In general, if $G$ is a product of distinct primes, then a Sylow subgroup for the largest prime divisor of $G$ is normal. We prove this theorem of Burnside when we study transfer theory in Chapter 5.}
\end{probl}

\begin{solution} Suppose that $n_r(G)>1$. Since $n_r(G)\mid pq$ and $n_r(G)>q>p$, the inescapable consequence is that $n_r=n_r(G)=pq$.

Following the hint, let's analyze what would happen if $n_q=n_q(G)>1$. Firstly, since $n_q>q>p$ and $n_q\mid pr$ we would have $n_q=pr$ or $n_q=r$. Since two different Sylow subgroups of $G$ intersect trivially, we deduce that to the $pq(r-1)$ elements of order $r$ we could add no less than $r(q-1)$ elements of order $q$ plus $p-1$ of order $p$. This is impossible because there would be at least 
$$
    pqr - pq + rq - r + r - 1 + 1 = pqr + (r-p)q
$$
elements in $G$, which renders the assumption unviable because $(r-p)q>0$. It follows that $n_q=1$, i.e., there is a normal Sylow $q$-subgroup $Q$ in $G$.

Let's now consider $\bar G=G/Q$, which is a group of order $pr$. By Theorem~\ref{thm-pq}, $\bar G$ has a normal Sylow $r$-subgroup~$\bar N$. The preimage $N$ of $\bar N$ in $G$ is normal of order $\abs N=qr$. Moreover, $\Syl_r(N)\subseteq\Syl_r(G)$. In particular, if $R\in\Syl_r(N)$, then we must have $N=QR$. It follows that for $x\in G$, $R^x\subseteq Q^xR^x=N^x=N$, i.e., $\Syl_r(N)=\Syl_r(G)$. Therefore, $n_r(N)=n_r=pq$, which is impossible because $p$ doesn't divide $\abs N$.  \end{solution}

\begin{probl}
    Show that there is no simple group of order\/ $315 = 3^2\cdot5\cdot7$.

    \textrm{\rm\textbf{Note.} This, of course, is also a consequence of the Feit-Thompson odd-order theorem, which asserts that no group of odd non-prime order can be simple.}
\end{probl}

\begin{solution} Suppose that $n_3$, $n_5$ and $n_7$ are greater than $1$. Then
\begin{enumerate}[\rm1.]
    \item $n_3\mid5\cdot7$, $n_5\mid3^2\cdot7$ and $n_7\mid3^2\cdot5$.
    \item Since $n_3\equiv1\pmod3$, $n_3=7$.
    \item Since $n_5\equiv1\pmod5$, $n_5=3\cdot7$.
    \item Since $n_7\equiv1\pmod7$, $n_7=3\cdot5$.
    %\item The intersection of two different $3$-subgroups is trivial or has order $3$.
    %\item The intersection of $5$- and $7$-subgroups is trivial.
    %\item There are $3\cdot7(5-1)$ elements of order $5$ and $3\cdot5(7-1)$ of order $7$.
    %\item Suppose all intersections between Sylow $3$-subgroups are trivial. Then there are $7\cdot(3-1)$ elements of order $3$.
    %\item If $P\in\Syl_3(G)$, $\abs{G:N_G(P)}=7$, i.e., $|N_G(P)|=3^2\cdot5$.
    %\item If $Q\in\Syl_5(G)$, $\abs{G:N_G(Q)}=3\cdot7$, i.e., $|N_G(Q)|=3\cdot5$.
    %\item If $R\in\Syl_7(G)$, $\abs{G:N_G(R)}=3\cdot5$, i.e., $|N_G(R)|=3\cdot7$.
    \item By Theorem \ref{n_p(G)=1}, there exist $P_1,P_2\in\Syl_3(G)$ such that $\abs{P_1\cap P_2}=3$, otherwise we would have $7\equiv1\pmod9$, which we haven't.
    \item By Problem \ref{problem-1.A.1}, $P_1\cap P_2\normal P_1$ and $P_1\cap P_2\normal P_2$.
    \item Put $H=N_G(P_1\cap P_2)$. Then $P_1\cup P_2\subseteq H$. In particular, $3^2\mid\abs H$ and $\abs H\ge 18-3=15$.
    \item The possibilities for $\abs H$ are $3^2\cdot5$, $3^2\cdot7$ and $3^2\cdot5\cdot7$.
    \item Moreover, $n_3(H)>3$ and so $n_3\in\set{5,7,5\cdot7}$. Since $n_3(H)\equiv1\pmod3$, it must be $\abs H=3^2\cdot7$.
    \item By the $n!$~theorem~\ref{n!-theorem}, $H$ contains a normal group $N$ such that $|G:N|\mid5!$. Thus $|G:N|\mid5\cdot3$. In particular $N\ne\gen1$.
\end{enumerate}
\end{solution}

\begin{probl}
    If\/ $|G|=144=2^4\cdot3^2$, show that\/ $G$ is not simple.
\end{probl}

\begin{solution} Suppose that $n_2$ and $n_3$ are greater than $1$. Then
\begin{enumerate}[\rm1.]
    \item $n_2\mid3^2$ and $n_3\mid2^4$.
    \item By Corollary \ref{p|n_p-1}, $n_2\in\set{3,3^2}$, $n_3\in\set{2^2, 2^4}$.
    \item If $n_2=3$, pick $Q\in\Syl_2(G)$. Then $|G:N_G(Q)|=3$ and there would exist $N$ normal such that $|G:N|\mid3!$, in particular $N\ne\gen1$, and we are done.
    \item If $n_3=2^2$, pick $P\in\Syl_3(G)$. Then $|G:N_G(P)|=2^2$ and there would exist $N$ normal such that $|G:N|\mid 4!$. In particular $N\ne\gen1$, and we are done.
    \item It follows that $n_2=3^2$ and $n_3=2^4$.
    %\item By Theorem \ref{n_p(G)=1}, there exist $P_1,P_2\in\Syl_2(G)$ such that $|P_1\cap P_2|\ge2$. Otherwise, $n_2\equiv1\pmod{2^4}$.
    \item By Theorem \ref{n_p(G)=1}, there exist $Q_1,Q_2\in\Syl_3(G)$ such that $|Q_1\cap Q_2|=3$. Otherwise, $n_3\equiv1\pmod{3^2}$.
    \item By Problem \ref{problem-1.A.1}, $Q_1\cap Q_2\normal Q_1, Q_2$.
    \item Therefore, if $K=N_G(Q_1\cap Q_2)$, then $Q_1\cup Q_2\subseteq K$. In particular, $|K|\ge 18-3=15$ and $3^2\mid|K|$. Thus, $2\cdot3^2\mid|K|$.
    \item If $|K|=2^4\cdot3^2$, then $K=G$ and we are done.
    \item If $|K|=2^3\cdot3^2$, by the $n!$~theorem~\ref{n!-theorem}, $K$ contains a normal group $N$ such that $|G:N|\mid2!$, which implies $N\ne\gen1$.
    \item If $|K|=2^2\cdot3^2$, by the $n!$~theorem~\ref{n!-theorem}, $K$ contains a normal group $N$ such that $|G:N|\mid4!$, which implies $N\ne\gen1$.
    \item If $|K|=2\cdot3^2$, by the $n!$~theorem~\ref{n!-theorem}, $K$ contains a normal group $N$ such that $|G:N|\mid8!$, which includes the case $N=\gen1$ because $8!/(2^4\cdot3^2)=280$.
    \item So, the case $|K|=2\cdot3^2$ requires further analysis.
    \item By Theorem \ref{thm-2odd}, there is a subgroup $Q\normal K$ such that $|K:Q|=2$.
    \item Then $|Q|=3^2$, meaning that $Q\in\Syl_3(K)$. Then $\Syl_3(K)=\set Q$, which contradicts the fact that $\Syl_3(K)$ had at least two elements $Q_1$ y $Q_2$.
\end{enumerate}
\end{solution}

\begin{probl}
    If\/ $|G|=2^4\cdot3\cdot7$, show that\/ $G$ is not simple.

    \textrm{\rm Hint. If $G$ is simple, compute $n_7(G)$ and use Problem~\ref{problem-1.C.5}}
\end{probl}

\begin{solution} Suppose that $G$ is simple. Then $n_2>2$, $n_3>3$ and $n_7>7$. Consider the following facts:
\begin{enumerate}[\rm1.]
    \item $n_7\mid2^4\cdot3$ and $n_7\equiv1\pmod7$ imply $n_7\in\set{2^3, \cancel{2^2\cdot3}, \cancel{2^4}, \cancel{2^3\cdot3}, \cancel{2^4\cdot3}}$, i.e., $n_7=8$.
    \item Let $S=\Syl_7(G)$. Then $|S|=8$ and $G$ acts on $S$ by conjugation and defines a single orbit.
    \item The action define an exact sequence $1\to\ker(\sigma)\to G\to\Sym(S)$, where the second arrow is $\sigma\colon x\mapsto(P\mapsto P^x)$.
    \item The kernel is given by
    \begin{align*}
        x\in\ker(\sigma) &\iff P^x=P, \text{ for all }P\in S\\
            &\iff x\in N_G(P), \text{ for all }P\in S\\
            &\iff x\in\bigcap_{P\in S}N_G(P).
    \end{align*}
    \item From 4 it follows that $\ker(\sigma)\varsubsetneq G$, otherwise $N_G(P)=G$ for all $P\in S$, and $G$ wouldn't be simple.
    \item Therefore, we are left with the case, $\ker(\sigma)=\gen1$.
    \item Fix $x\in G$. Since $\sigma$ is mono, $\ord(\sigma(x))=\ord(x)$. Thus $\sigma(x)\in \Alt(S)$ whenever $\ord(x)=3$, because in that case $\sigma(x)$ is a product of disjoint $3$-cycles.
    \item Let $A=\sigma^{-1}(\Alt(X))$ be the preimage of the alternating group on $S$.
    \item Since $A$ is normal, $\sigma$ is mono and we are assuming that $G$ is simple, it follows that $A=G$, i.e., $\sigma$ can be coastrictec to $\sigma\colon G\to\Alt(S)$.
    \item We have now restricted ourselves to the case where $G$ is a subgroup of $A_8$.
    \item By Problem \ref{problem-1.C.2}, every Sylow subgroup of $G$ is the intersection with $G$ of a Sylow subgroup of $A_8$.
    \item Fix a pair $P$, $Q$ as in 11, i.e., $P=G\cap Q$. Then $G\cap N_{A_8}(Q)\subseteq N_G(P)$.
    \item Since $|A_8|=8!/2 = 2^6\cdot3\cdot5\cdot7$, it follows that $|Q|=7$, i.e., $P=Q$. Thus, $G\cap N_{A_8}(P)=N_G(P)$.
    \item By Problem \ref{problem-1.C.5}, $N_{A_8}(P)=7(7-1)/2=3\cdot7$.
    \item Since $n_7=8$, we deduce that $8=n_7=|G:N_G(P)|=2^4\cdot3\cdot7/|N_G(P)|$. Hence, $|N_G(P)|=2\cdot3\cdot7$. A contradiction, because $N_G(P)$ would be a subgroup of $N_{A_8}(P)$, which is smaller.
\end{enumerate}
\end{solution}

\begin{probl}
    If\/ $G=2^2\cdot3^2\cdot5$, show that\/ $G$ is not simple.
\end{probl}

\begin{solution} As in the preceding problems, suppose that $G$ is simple and consider the following facts:
\begin{enumerate}[\rm1.]
    \item $n_2\in\set{3, 5, 3^2, 3\cdot5, 3^2\cdot 5}$.
    \item By the $n!$~theorem~\ref{n!-theorem} we can rule out $3$ and $5$. Thus, $n_2\in\set{3^2,3\cdot5,3^2\cdot5}$.
    \item $n_3\in\set{\cancel2,2^2, \cancel5, 2\cdot 5, \cancel{2^2\cdot5}}=\set{2^2,2\cdot5}$.
    \item By the $n!$~theorem~\ref{n!-theorem}, we can rule out $2^2$. Hence $n_3=2\cdot5$.
    \item $n_5\in\set{\cancel2, \cancel3, \cancel2^2, 2\cdot3, \cancel{3^2}, \cancel{2^2\cdot3}, \cancel{2\cdot3^2}, 2^2\cdot3^2} = \set{2\cdot3,2^2\cdot3^2}$.
    \item {\it Suppose} that $n_5=2\cdot3$ and let $G$ act by conjugation on $S=\Syl_5(G)$. This defines the morphism $\sigma\colon G\to\Sym(S)$, $x\mapsto(R\mapsto R^x)$, which must be mono as shown in the previous problem. Reasoning as in that problem, we are also reduced to the case where $G$ is a subgroup of $A_6$. 
    \item Since $|A_6|=2^3\cdot3^2\cdot5$, the value of $n_5$ for both $G$ and $A_6$ is the same, which according to Problem~\ref{problem-1.C.5} implies $|N_G(R)|=5(5-1)/2=2\cdot5$ for all $R\in\Syl_5(G)$. But this is impossible because it would violate the equation $|G|\ne|G:N_G(R)||N_G(R)|$.
    \item Therefore, the supposition of part 6 doesn't work and we are left with the case $n_5=2^2\cdot3^2$.
    \item {\it Suppose\/} that $|Q_1\cap Q_2|=3$ for $Q_1,Q_2\in\Syl_3(G)$. Let $K=N_G(Q_1\cap Q_2)$. By Problem~\ref{problem-1.A.1}, $Q_1\cap Q_2\normal Q_1,Q_2$ and so $Q_1\cup Q_2\subseteq K$. It follows that $|K|\ge15$. Then $|K|\ge18$, because $9\mid|K|$.
    \item If $|K|=18$, we would have $|K:Q_1|=2$ and, by Lemma~\ref{index-2-is-normal}, $Q_1\normal K$, which is impossible because $Q_2\ne Q_1$.
    \item Thus $|K|>18$. Therefore $2^2$ or $5$ must divide $|K|$. But, since we are under the assumption that $n_3(K)\ge2$ and $n_3(K)\equiv1\pmod3$, it must be $2^2\mid|K|$ or $2\cdot5\mid|K|$.
    \item If $|K|=|G|$ we are done. So, we may assume $|K|=2\cdot3^2\cdot5$ or $2^2\cdot3^2$.
    \item If $|K|=2\cdot3^2\cdot5$, then $K\normal G$ by Lemma~\ref{index-2-is-normal}. So, $|K|=2^2\cdot3^2$. But in this case $|G:K|=5$ and by the $n!$~theorem~\ref{n!-theorem}, there would exist $N\subseteq K$ normal with $|G:N|\mid 5!$, which makes it impossible for $N$ to be $\gen1$.
    \item Therefore, the supposition made in part 9 isn't viable, which means that all pairs of Sylow $3$-subgroups have trivial intersections.
    \item {\it Suppose\/} that $|P_1\cap P_2|=2$ for $P_1,P_2\in\Syl_2(G)$. Let $H=N_G(P_1\cap P_2)$. By Problem~\ref{problem-1.A.1}, $P_1\cap P_2\normal P_1,P_2$ and so $P_1\cup P_2\subseteq H$.
    \item By definition $P_1\cap P_2\normal H$. By Lemma~\ref{normal-order2}, $P_1\cap P_2\subseteq Z(H)$. In particular, $\gen1\subseteq P_1\cap P_2\subseteq H$ is a central series for $H$. Thus, $H$ is nilpotent and so $P_1\normal H$ by Theorem~\ref{nilpotent-equivalences}, which is impossible because $P_1\ne P_2\subseteq H$.
    \item Thus, supposition at 15 doesn't hold, which means that any two Sylow $2$-subgroups of $G$ intersect trivially.
    \item By fact 2, there are at least $3^2\cdot(2^2-1)+2\cdot5\cdot(3^2-1)+2^2\cdot3^2\cdot(5-1)$ elements whose orders are positive powers of $2$, $3$ or $5$. But this is impossible because there aren't that many elements in $G$.
\end{enumerate}
\end{solution}

\begin{probl}
    If\/ $|G|=240=2^4\cdot 3\cdot 5$, show that $G$ is not simple.
\end{probl}

\begin{solution} Suppose that $G$ is simple and consider the following facts:
\begin{enumerate}[\rm1.]
    \item $n_2\in\set{3,5,3\cdot5}$.
    \item $n_3\in\set{2^2,2^4,2^3\cdot5}$.
    \item $n_5\in\set{2\cdot3,2^4}$.
    \item By the $n!$~theorem~ref{n!-theorem}, we can rule out $3$, $4$ and $5$. Therefore, $n_2=3\cdot5$ and $n_3\in\set{2^4,2^3\cdot5}$.
    \item Let $H=P_1\cap P_2$ be the largest possible intersection of two different Sylow $2$-subgroups.
    \item We claim that $|H|>1$. Otherwise, there would be $3\cdot5\cdot(2^4-1)$ elements whose orders are positive powers of $2$ plus at least $2^2(3-1)$ with powers of $3$ plus $2\cdot3(5-1)$ with powers of $5$. A total of $257$ in a group of $240$.
    \item {\it Suppose\/} that $|H|=2^3$. Let $L=N_G(P_1\cap P_2)$. By Problem~\ref{problem-1.A.1}, $P_1\cap P_2$ is normal in both $P_1$  and $P_2$. Therefore, $P_1\cup P_2\subseteq L$. It follows that $|L|\ge 2^5-2^3=2^3\cdot3$. Since $2^4\mid|L|$ and $G$ is not simple, we deduce that $|L|\in\set{2^3\cdot3,\,2^3\cdot5,\,2^4\cdot3,\,2^4\cdot5}$.
    \item By the~$n!$~theorem~\ref{n!-theorem}, we can rule out $2^4\cdot3$. Thus, $|L|\in\set{2^3\cdot3,2^3\cdot5}$. We can now rule these two values invoking the $n!$~theorem~\ref{n!-theorem}. So, our hypothesis at 7 doesn't work, meaning that $|H|\in\set{2,2^2}$.
    \item {\it Suppose\/} that $|H|=2^2$. By Theorem \ref{n_p(G)=1}, $n_2\equiv1\pmod{|P_1:H|}$, which is not the case because $n_2=3\cdot5$ and $|P_1:H|=4$.
    \item Therefore, the only possibility is $|H|=2$. In this case Theorem~\ref{n_p(G)=1} says that $n_2\equiv1\pmod8$. But $n_2=3\cdot5\equiv-1\pmod8$.
\end{enumerate}
\end{solution}

\begin{probl}
    If\/ $|G|=252=2^2\cdot3^2\cdot7$, show that\/ $G$ is not simple.
\end{probl}

\begin{solution} Suppose that $G$ is simple. Then,

\begin{enumerate}[\rm1.]
    %\item $n_2\in\set{3^2, 3\cdot7,3^2\cdot7}$ ($3$ would violate Theorem \ref{n_p(G)=1}.)
    \item $n_3\in\set{7,2^2\cdot7}$ ($2^2$ would violate Theorem \ref{n_p(G)=1}.)
    \item $n_7=2^2\cdot3^2$.
    \item Let $K=Q_1\cap Q_2$ be the largest possible intersection of two different Sylow $3$-subgroups.
    \item If $|K|=1$, then there would be $2^2\cdot3^2(7-1)$ elements of order $7$ plus no less than $7(9-1)$ of order $3$ or $9$, which gives a total of $272$ in a group of order $252$. Impossible. Then $|K|=3$.
    \item Define $J=N_G(K)$. Then $K\normal Q_1,Q_2$ and we get $Q_1\cup Q_2\subseteq J$. Hence, $|J|\ge 15$. Since $9\mid|J|$, we get $|J|\ge18$. Then $|J|\ge2\cdot3^2$.
    \item If $|J|=2^2\cdot3^2\cdot7$, then $J=G$ and we are done.
    \item If $|J|=2\cdot3^2\cdot7$, then $|G:J|=2$ and we are also done (Lemma~\ref{index-2-is-normal}).
    \item If $|J|=2\cdot3^2$, then $|J:Q_1|=2$ which is impossible because that would mean $Q_1\normal J$.
    \item Therefore, we are left with the case $|J|=2^2\cdot3^2$.
    \item Let $S=\bigcup\Syl_7(G)\setminus\gen1$. Then $|S|=2^2\cdot3^2\cdot(7-1)$. Moreover, $S\cap J=\emptyset$, because elements in $S$ have order $7$ and elements in $J$ orders with spec in $\set{2,3}$. Since $|S\cup J|=|S|+|J|=|G|$, we get a partition of $G=S\cup J$.
    \item Now take $Q\in\Syl_3(G)$, since $Q\cap S=\emptyset$, it follows that $Q\subseteq J$. In other words, $J$ includes all Sylow $3$-subgroups of $G$, i.e., $n_3(J)=n_3$. But this is impossible because $7\mid n_3$ (fact 1) and $n_3(J)\mid2^2$.
\end{enumerate}
\end{solution}

\section{Abelian Groups}

\begin{thm}
    Let\/ $G$ be an abelian group and\/ $C$ a cyclic subgroup of maximal order in\/ $G$. Then, for all\/ $x \in G$,\/ $\ord(x)$ divides\/ $|C|$.
\end{thm}

\begin{proof} This is nothing but part e) of Lemma~\ref{order-properties}.  \end{proof}

\begin{thm}
    Let\/ $G$ be abelian and\/ $C$ a cyclic subgroup with maximal order. Then there exists a complement\/ $A$ of\/ $C$ in\/ $G$, i.e. $G = CA$ and\/ $C\cap A=\gen1$.
\end{thm}

\begin{proof} Let $y\in G$ have minimum order among the elements of $G\setminus C$. Pick a generator $c$ of $C$. The previous theorem implies that $\ord(y)\mid|C|=\ord(c)$. If $n=\ord(c)=\ord(y)m$, then $\ord(c^m)=\ord(y)$

Take $p\in\spec(\ord(y))$. Since $\ord(y^p)=\ord(y)/p<\ord(y)$, we must have $y^p\in C$. Thus, $\gen{y^p}$ is a subgroup of $C$ of order $\ord(y)/p=n/(pm)$, hence generated by $c^{pm}$ [Corollary~\ref{cyclic-subgroups}]. It follows that $y^p=c^{pr}$ for some integer $n$. Therefore, $(yc^{-pr})^p=1$ with $yc^{-pr}\in G\setminus C$. By the definition of $y$ this implies that $\ord(y)=p$. Since $y\notin C$, it follows that
$C\cap\gen y=\gen1$.

Let $\bar G=G/\gen y$. Then
$$
    \bar C= C\gen y / \gen y \cong C/C\cap\gen y\cong C,
$$
which implies $|\bar C|=|C|$. It follows that $\bar C=\gen{\bar c}$ is cyclic of maximal order in $\bar G$. Thus, induction on $|G|$ implies the existence of a complement $\bar A$ of $\bar C$ in $\bar G$, where $\gen y\subseteq A$.

Therefore, to complete the proof it is enough to show that $A$ is a complement of $C$ in $G$. But $\bar G=\bar C\bar A\implies G=CA\gen y=CA$, and
$$
    \bar C\cap\bar A=\gen 1\implies C\cap A\subseteq\gen y
        \implies C\cap A\subseteq C\cap\gen y=\gen1.
$$
 \end{proof}

\begin{cor}\label{product-of-cyclic}
    Every finite abelian group is the direct product of cyclic subgroups.
\end{cor}

\begin{proof} Pick a cyclic subgroup $C$ of maximal order. By the theorem, $C$ has a complement $A$ in $G$. Then $G=CA$ is a direct product. The conclusion follows by induction on $|G|$ which allows us to assume that $A$ is a direct product of cyclic subgroups.  \end{proof}

\begin{cor}\label{subgroup-of-compatible-order}
    Let $G$ be finite and abelian. If $m\mid|G|$, then $G$ contains a subgroup of order~$m$.
\end{cor}

\begin{proof} According to the previous corollary, $G$ can be decomposed as a direct product $C_1\cdots C_r$ of cyclic groups. In particular $m\mid|G|=|C_1|\cdots|C_r|$ and we can write $m=m_1\cdots m_r$ with $m_i\mid|C_i|$. Since each of the $C_i$ contains a subgroup of order $m_i$, the product of these subgroups is a direct product of order~$m$.  \end{proof}

We can now prove the following corollary without using any of the Sylow theorems.

\begin{cor}
    Let $G$ be an abelian group of order $p^em$, where $p$ is prime and $p\perp m$. Then
    $$
        G_p =\set{x\in G\mid\spec(\ord(x))=\set p}
    $$
    is a group of order $p^e$.
\end{cor}

\begin{proof} Since $G$ is abelian, $G_p$ is a subgroup. By the previous corollary, there exists a subgroup $P$ of order $p^e$. Clearly $P\subseteq G_p$. Suppose that $P\ne G_p$. Then there is a subgroup $H$ of of $G_p$ of order $|G_p:P|>1$. But this is impossible because $|G_p|\mid|G|=p^em$ and $|G_p|=|P||H|$ imply $|H|\mid m\perp p$.  \end{proof}

\begin{cor}\label{product-of-all-Gp}
    Let\/ $G$ be an abelian group. Then
$$
    G = \prod_{p \in \spec|G|} G_p
$$
\end{cor}

\begin{proof} The product on the RHS is direct because $G_p\cap G_q=\gen1$ whenever $p\ne q$. Then the order of the product equals $G$ by the previous corollary.  \end{proof}

\begin{thm}
    Let\/ $G$ be an abelian group\/ $G$. Then the following statements are equivalent:
    \begin{enumerate}[\rm a)]
      \item $G$ is cyclic.
      \item For all\/ $p \in \spec|G|$, there exists exactly one subgroup of order\/ $p$ in\/ $G$.
      \item $G_p$ is cyclic for all\/ $p \in \spec|G|$.
    \end{enumerate}
\end{thm}

\begin{proof}${}$
\begin{enumerate}[\rm a)]
    \item $\Rightarrow$~b) If $G$ is cyclic there exists exactly one subgroup for an divisor of~$|G|$ [cf.~Proposition~\ref{cyclic-subgroups}].

    \item $\Rightarrow$~c) By Corollary~\ref{product-of-cyclic} we know that $G_p$ is a direct product of cyclic subgroups. Should the product have more than one factor, there would be more than one subgroup of order $p$ in $G$.

    \item $\Rightarrow$~d) This is a consequence of Corollary~\ref{product-of-all-Gp} because the product of two cyclic groups of coprime orders is cyclic [cf.~Lemma~\ref{order-properties}~d)].
\end{enumerate}
\end{proof}

\begin{thm}
    Let\/ $G$ be an elementary abelian\/ $p$-group of order\/ $p^n > 1$. Then
    \begin{enumerate}[\rm a)]
    \item $G$ is the direct product of\/ $n$ cyclic groups of order\/ $p$.
    \item If\/ $G$ is written additively, the scalar multiplication
    $$
        kx := x + \cdots + x \quad \text{for } k \in \Z_p \text{ and } x \in G,
    $$
    makes\/ $G$ into an\/ $n$-dimensional vector space\/ $\mathbb V$ over the prime field\/ $\Z_p$. The subgroups of\/ $G$ correspond to the subspaces of\/ $\mathbb V$, and the automorphisms of\/ $G$ to the automorphisms of\/~$\mathbb V$.
    \end{enumerate}
\end{thm}

\begin{proof}${}$
\begin{enumerate}[\rm a)]
    \item By Corollary~\ref{product-of-cyclic}, $G$ is the direct product of cyclic groups, which in this case must have order $p$ because nontrivial elements of $G$ have order~$p$.

    \item Since $G$ is a $\Z$-module and $x^p=1$ for $x\in G$, the additive notation induces a structure of $\Z_p$-vectorial space on $G$, that we may denote by $\mathbb V$. The subgroups of $G$ are submodules of the $\Z$-module $G$, hence subspaces of the $\mathbb V$. Similarly, if $\phi\in\Aut(G)$, then $\phi$ is an automorphism of $\Z$-modules, hence of $\Z_p$-vector spaces.
\end{enumerate}
\end{proof}

\begin{ntn}
    Given an abelian\/ $p$-group\/ $G$ we introduce the subgroups
    $$
        \Omega_i(G) = \set{x\in G\mid x^{p^i}=1}
    $$
    for\/ $i\ge0$.
\end{ntn}

\begin{rem}\label{omega-props}${}$
    \begin{enumerate}[\rm a)]
        \item $\Omega_i(G)\subseteq\Omega_{i+1}(G)$. 
        \item $\Omega_i(G)$ is characteristic.
        \item $\Omega_{i+1}(G)/\Omega_i(G) = \Omega_1(G/\Omega_i(G))$.
        \item $G=\Omega_1(G)\iff G$ is elementary abelian.
        \item If\/ $G=\gen c$ then\/ $|\Omega_{i+1}(G)/\Omega_i(G)|\le p$, with equality attained if, and only if, $\ord(c)\ge p^{i+1}$.
    \end{enumerate}
    \textrm{\rm Part e) follows from part c) in the case $G$ cyclic.}
\end{rem}

\begin{thm}\label{cyclic-factors-of-order-p^i}
    Let\/ $G$ be an abelian\/ $p$-group such that
    \begin{equation}\label{eq.factors}\tag{*}
        G = A_1 \times \cdots \times A_n
    \end{equation}
    is the direct product of\/ $n$ cyclic groups\/ $A_i \ne\gen1$. Then
    $$
        |\Omega_1(G)| = p^n.
    $$
    More precisely, if
    $$
        |\Omega_i(G)/\Omega_{i-1}(G)| = p^{n_i},
    $$
    then\/ $n_i - n_{i+1}$ is the number of factors of order\/ $p^i$ in the direct product~$(\ref{eq.factors})$.
\end{thm}

\begin{proof} Take $x\in G$, $x=a_1\cdots a_n$. Then $x\in\Omega_i(G)$ if, and only if, $a_1^{p^i}\cdots a_n^{p^i}=1$, which is equivalent to $a_i^{p^i}=1$ for all $i$, as it follows from Theorem~\ref{direct-product}. In other words,
\begin{equation}\label{eq.omega-product}
    \Omega_i(G) = \Omega_i(A_1)\times\cdots\times\Omega_i(A_n).
\end{equation}
For the first equation put $i=1$ and note that if $A=\gen a$ is a cyclic $p$-group, then $(a^j)^p=1$ if, and only if, $a^j\in\gen{a^{\ord(c)/p}}$. Thus, $|\Omega_1(A)|=p$.

Since $\Omega_i(G)$ is normal (characteristic) in $G$ and $\Omega_i(A_j) =\Omega_i(G)\cap A_j$, equation~$(\ref{eq.omega-product})$ allows us to invoke Proposition~\ref{quotient-of-products} for $N=\Omega_i(G)$ to deduce that
\begin{equation}\label{eq.omega-quotient-product}
    G/\Omega_i(G) \cong A_1/\Omega_i(A_1)\times\cdots\times A_n/\Omega_i(A_n).
\end{equation}
Therefore, applying the decomposition given in $(\ref{eq.omega-product})$ to $(\ref{eq.omega-quotient-product})$, we get
\begin{align*}
    \Omega_{i+1}(G)/\Omega_i(G) &= \Omega_1(G/\Omega_i(G))
                &&\text{; \ref{omega-props} c)}\\
        &= \Omega_{i+1}(A_1)/\Omega_i(A_1)\times\cdots\times
            \Omega_{i+1}(A_n)/\Omega_i(A_n),
\end{align*}
which shows that
\begin{align*}
    n_{i+1} &= \set{j\mid \Omega_i(A_j)\ne\Omega_{i+1}(A_j)}\\
        &= |\set{j\mid|A_j|\ge p^{i+1}}| &&\rm;\ \ref{omega-props}~e)
\end{align*}
and so,
$$
    n_i - n_{i+1} = |\set{j\mid|A_j|=p^i}|,
$$
as wanted.  \end{proof}

\begin{defn}
    A group is \textsl{traversal} if every subgroup has a complement; \textsl{nontraversal} if it has proper, nontrivial subgroups, none of which has a complement, and \textsl{semitraversal} if it is not traversal but some proper, nontrivial subgroup has a complement.
\end{defn}

\begin{thm}
    Let\/ $G$ be finite abelian group, $G\cong \Z_{p_1^{e_1}}\times\cdots\times\Z_{p_k^{e_k}}$, with\/ $k\ge 0$, distinct primes\/ $p_1,\dots,p_k$ and\/ $e_i\ge1$. Then\/ $G$ is
    \begin{enumerate}[\rm a)]
        \item traversal when\/ $e_i=1$ for all\/ $i$,
        \item semitraversal when\/ $k>1$ and\/ $e_i>1$ for some\/ $i$ and
        \item nontraversal when\/ $k=1$ and\/ $e_1>1$.
    \end{enumerate}
\end{thm}

\begin{proof}${}$
\begin{enumerate}[\rm a)]
    \item Suppose that $e_i>1$. Then no nontrival subgroup of $\Z_{p_i^{e_i}}$ has a complement because nontrivial subgroups of this group intersect trivially. In consequence, $G$ is not traversal.
    
    Conversely, since $\Z_{p_1}\times\cdots\times\Z_{p_k}$ is cyclic and equals $\Z_n$, where $n=p_1\cdots p_k$, is a product of some of the $\Z_{p_i}$, and hence has a complement.

    \item As we observed in part a) if $e_i>1$ for some $i$, then $G$ is not traversal. And since $k>1$, it is clearly semitraversal.

    \item One implication follows from the first observation of part~a). The other is trivial.
\end{enumerate}
\end{proof}

\subsection{Exercises - Kurzweil \& Stellmacher - \S 2.1}

\begin{exr}
    Let\/ $G$ be an abelian group with exponent $e \in \N$. Then there exists an element\/ $b \in G$ such that\/ $\ord(b) = e$.
\end{exr}

\begin{solution} Pick $b\in G$ with maximum order. By Lemma~\ref{order-properties}, $\ord(b)$ is the $\lcm$ of the orders of all elements in $G$. It follows that $a^{\ord(b)}=1$ for all $a\in G$, i.e., $e\le\ord(b)$. Since $b^e=1$, we also have $\ord(b)\mid e$, i.e., $\ord(b)\le e$.  \end{solution}

\begin{exr}
    Let\/ $G$ be abelian with exponent\/ $e$. Then\/ $G$ is cyclic if and only if\/ $|G| = e$.
\end{exr}

\begin{solution} This is a direct consequence of the previous exercise.  \end{solution}

\begin{exr}
    Let\/ $p$ be a prime, $C = \Z_{p^3} \times \Z_{p^3}$, $B = \Z_p \times \Z_p \times \Z_p$ and\/ $G = C \times B$. Then no subgroup of\/ $G$ has a complement isomorphic to\/ $\Z_{p^2}$ in\/~$G$.
\end{exr}

\begin{solution} Put $|\Omega_i(G)/\Omega_{i-1}(G)|=p^{n_i}$. According to Theorem~\ref{cyclic-factors-of-order-p^i}, we have
\begin{align*}
    n_1 &= 5\\
    n_1 - n_2 &= 3  &\implies n_2=2\\
    n_2 - n_3 &= 0  &\implies n_3=2\\
    n_3 - n_4 &= 2\\
    n_4 &= 0.
\end{align*}
Since these numbers (meaning the $n_i$) are intrinsic (i.e., depend exclusively on~$G$), so they are their differences, which imply that the number of factors of order $p^2$ must be zero because it must equal $n_2-n_3$.  \end{solution}

\begin{exr}
    Every abelian group of order\/ $546$ is cyclic.
\end{exr}

\begin{solution} Given that $546=2\cdot3\cdot7\cdot13$, every decomposition of the group in a direct product of cyclic factors would be cyclic because those factors would have coprime orders.  \end{solution}

\begin{exr}
    Give an example of a nonabelian group that satisfies the statement of\/ \textrm{\rm Theorem~\ref{subgroup-of-compatible-order}}.
\end{exr}

\begin{solution} The group $S_3$ is nonabelian, has order $3!=6$, a subgroup of order $2$ generated by the involution and another of order $3$ generated by the rotation.  \end{solution}

\begin{exr}\label{exercise-2.1.6}
    If $G$ is a finite abelian group, determine $\prod G$.
\end{exr}

\begin{solution} Given $x\in G\setminus\set1$ its contribution to the product cancels out with $x^{-1}$, unless $x$ happens to be an involution. Therefore, the product reduces to the identity or an involution (the product of all involutions except one).  \end{solution}

\begin{exr}
    Let\/ $G$ be a finite abelian group. Show that for every subgroup\/ $A \subseteq G$, there exists an endomorphism\/ $\phi$ of\/ $G$ such that\/ $\im\phi = A$.    
\end{exr}

\begin{solution} Another way to express the same is by saying that 

\begin{quote}
    \textit{every subgroup is isomorphic to a quotient of the group}.
\end{quote}
Let's begin with a 

\textbf{Lemma.} $A\subgroup G\implies
    \Omega_{i+1}(A)/\Omega_i(A)\subgroup\Omega_{i+1}(G)/\Omega_i(G)$.

{\small \begin{proof} The inclusion $A\to G$ induces inclusions $\Omega_j(A)\to\Omega_j(G)$. In particular we have a morphism
$$
    \Omega_{i+1}(A)/\Omega_i(A)\to\Omega_{i+1}(G)/\Omega_i(G),
$$
which is mono because
$$
    \Omega_i(A) = \Omega_{i+1}(A)\cap\Omega_i(G).
$$
 \end{proof}}

\medskip

Without loss of generality we may assume that $G$ is a $p$-group [cf.~Corollary~\ref{product-of-all-Gp}]. According to Corollary~\ref{product-of-cyclic}, we can further assume that
$$
    G=(\Z_{p^{e_1}})^{r_1}\oplus\cdots
        \oplus(\Z_{p^{e_i}})^{r_i}\oplus\cdots\oplus(\Z_{p^{e_n}})^{r_n},
$$
where the $e_1>\cdots>e_n\ge1$ and $r_i>0$ for all $i$. By the same corollary, there exists an isomorphism
$$
    A\cong(\Z_{p^{d_1}})^{s_1}\oplus\cdots\oplus(\Z_{p^{d_m}})^{s_m},
$$
with $d_1>\cdots>d_m\ge1$ and $s_j>0$ for all $j$.

By the lemma above and Theorem~\ref{cyclic-factors-of-order-p^i}, we deduce that there must be~$s_1$ factors of $G$ with order $\ge p^{d_1}$. In particular, $d_1\le e_1$ (otherwise, there would be none). More precisely, if $i$ is the maximum index satisfying $d_1\le e_i$, then we must have $r_1+\cdots+r_i\ge s_1$. Thus, we can sum the quotient projections
\begin{align*}
    \varphi_j\colon\Z_{p^{e_j}}&\to\Z_{p^{d_1}}
\end{align*}
to obtain the epimorphism
\begin{align*}
    \varphi=\bigoplus_{j=1}^i\varphi_j\colon
        (\Z_{p^{e_1}})^{r'_1}\oplus\cdots
            \oplus(\Z_{p^{e_i}})^{r'_i}
        &\to(\Z_{p^{d_1}})^{s_1},
\end{align*}
where $0\le r'_i\le r_i$ are integers that satisfy $r'_1+\cdots+r'_i=s_1$.

Now, among the remaining factors of $G$, there must be $s_2$ of them with order at least $p^{d_2}$. Thus, we can apply the same argument and repeat it until we exhaust the $m$ summands of $A$. In the end, we will be able to map $s_1+\cdots+s_m$ factors of $G$ onto the decomposition of $A$. The final epimorphism would then consist of this one, plus the mapping of any remaining factors to~$0$.  \end{solution}

\begin{exr}
    If\/ $\Aut(G)$ is abelian, then\/ $G$ is cyclic.
\end{exr}

\begin{solution} Without loss of generality we may assume that $G$ is a $p$-group.

Let $p$ be a prime. Consider the automorphisms
\begin{align*}
    \phi\colon\Z_{p^e}\oplus\Z_{p^e}
        &\to\Z_{p^e}\oplus\Z_{p^e}
    &\psi\colon\Z_{p^e}\oplus\Z_{p^e}
        &\to\Z_{p^e}\oplus\Z_{p^e}\\
    (x,y)&\mapsto(y,x)&(x,y)&\mapsto(x,x+y).
\end{align*}
We have
$$
    \psi\circ\phi(x,y)=(y,x+y)\quad\text{and}\quad
    \phi\circ\psi(x,y)=(x+y,x).
$$
It follows that $\psi\circ\phi\ne\phi\circ\psi$. As a result, the decomposition
$$
    G=(\Z_{p^{e_1}})^{r_1}\oplus\cdots\oplus(\Z_{p^{e_n}})^{r_n},
$$
with $e_1>\cdots>e_n\ge1$, must satisfy $r_1=\cdots=r_n=1$.

Suppose that $n>1$ and define
\begin{align*}
    \phi\colon\Z_{p^{e_1}}\oplus\Z_{p^{e_2}}
        &\to\Z_{p^{e_1}}\oplus\Z_{p^{e_2}}
    &\psi\colon\Z_{p^{e_1}}\oplus\Z_{p^{e_2}}
        &\to\Z_{p^{e_1}}\oplus\Z_{p^{e_2}}\\
        (x,y)&\mapsto(-x,y)&(x,y)&\mapsto(x,x+y).
\end{align*}
Then,
$$
    \psi\circ\phi(x,y)=(-x,-x+y)\quad\text{and}\quad
    \phi\circ\psi(x,y)=(-x,x+y)
$$
Since $x=-x$ doesn't hold for every $x\in\Z_{p^{e_1}}$ because $e_1>1$, we deduce that $\psi\circ\phi\ne\phi\circ\psi$. Therefore, $n=1$ and $G$ is cyclic.  \end{solution}

\begin{exr}
    With the help of\/ \textrm{\rm Exercise~\ref{exercise-2.1.6}}, show Wilson's Theorem:
    $$
        (p - 1)! \equiv -1 \pmod p,
    $$
    where\/ $p$ is a prime.
\end{exr}

\begin{solution} Apply the exercise to $G=\Z_p^*$. Since $x^2=1$ on has two solutions, namely $x=1$ and $x=p-1$, we deduce that
$$
    p-1=\prod\Z_p^*=(p-1)!
$$
in $\Z_p$.  \end{solution}

\begin{exr}
    Let\/ $a, p \in \N$, $p$ a prime and\/ $a\perp p$. Then\/ $a^{p-1} \equiv 1 \pmod{p}$.
\end{exr}

\begin{solution} Given that $a\perp p$, we can see $a$ as an element of $\Z_p^*$. In particular, $\ord(a)\mid|\Z_p^*|=p-1$ and so $a^{p-1}=1$ in $\Z_p$.  \end{solution}

\section{Brodkey Theorem}

\begin{thm}\label{brodkey-general}
    Let\/ $G$ be a finite group and\/ $p$ a prime. Choose\/ $P$ and\/ $Q$ in $\Syl_p(G)$ such that\/ $P\cap Q$ is minimal in the set of intersections of two Sylow\/ $p$-subgroups of\/ $G$. Then\/ $O_p(G)$ is the unique largest subgroup of\/ $P\cap Q$ that is normal in both $P$ and $Q$.
\end{thm}

\begin{proof} Let $H=P\cap Q$. Since $O_p(G)$ is normal in $G$, it is also normal in $P$ and $Q$. Let~$K$ be a subgroup of $H$ with this same property, i.e., $K\normal P$ and $K\normal Q$. We have to show that $K\subseteq O_p(G)$ or, in other words, $K\subseteq S$ for all $S\in\Syl_p(G)$.

Put $L=N_G(K)$. Then $P,\, Q\subseteq L$ and so $P,\,Q\in\Syl_p(L)$. Since $S\cap L$ is a $p$-subgroup, there exists $y\in L$ such that $S\cap L\subseteq P^y$. In addition, $Q^y\subseteq L$. Thus,
$$
    S\cap Q^y = S\cap L\cap Q^y \subseteq P^y\cap Q^y = H^y,
$$
and equality is attained by the minimality of $|H|=|H^y|$. In particular,
$$
    K=K^y\subseteq H^y = S\cap Q^y \subseteq S.
$$
 \end{proof}

\begin{cor}\label{brodkey-thm}{\rm[Brodkey Theorem]}
    If every Sylow $p$-subgroups of the group\/~$G$ is abelian, then there exist $P,\,Q\in\Syl_p(G)$ such that $P\cap Q = O_p(G)$.
\end{cor}

\begin{proof} Choose $P,\,Q\in\Syl_p(G)$ such that $H=P\cap Q$ has minimum order. Since $P$ and $Q$ are abelian, $H\normal P$ and $H\normal Q$. By the theorem, $H=O_p(G)$, as desired.  \end{proof}

\begin{cor}
    Let\/ $P\in\Syl_p(G)$, where\/ $G$ is a finite group, and assume that\/ $P$ is abelian. Then $|G:O_p(G)|\le|G:P|^2$.
\end{cor}

\begin{proof} Let $Q_1,\,Q_2\in\Syl_p(G)$ be such that $Q_1\cap Q_2=O_p(G)$. We have
$$
    |G| \ge |Q_1Q_2|=\frac{|Q_1||Q_2|}{|O_p(G)|}=\frac{|P|^2}{|O_p(G)|},
$$
and the desired inequality follows.  \end{proof}

\begin{cor}
    Let\/ $P\in\Syl_p(G)$ where\/ $G$ is finite and\/ $P$ abelian, and assume that $|P|^2>|G|$. Then $O_p(G)>1$ and\/ $G$ is not simple unless $|G|=p$.
\end{cor}

\begin{proof} The inequality $O_p(G)>1$ is a direct consequence of the previous corollary. Thus, $G$ is not simple unless $G=O_p(G)$, i.e., $G=P$, which is abelian, hence not simple unless $|G|=p$.  \end{proof}

\subsection{Problems F}

\begin{probl}\label{problem-1.F.1}
    Prove that, if\/ $N\normal G$, $H\normal G$ and\/ $N\cap H=\gen1$, then\/ $N$ and\/ $H$ centralize each other.

    \textrm{\rm Hint. Consider elements of the form $[x,y]$, where $x\in N$ and $y\in H$.}
\end{probl}

\begin{solution} It is enough to show that $N$ centralizes $H$, i.e., $N\subseteq C_G(H)$. According to Theorem \ref{direct-product}, the product induces a monomorphism
$$
    1\to N\times H\to G
$$
that maps $N\times\gen1$ onto $N$ and $\gen1\times H$ onto $H$. And given that $N\times\gen1$ commutes with $\gen1\times H$ and the monomorphism maps $C_{N\times H}(\gen1\times H)$ onto $C_{NH}(H)\subseteq C_G(H)$, we arrive at the desired conclusion.  \end{solution}

\begin{probl}
    Let\/ $G$ be a group and\/ $p$ a prime. Show that if\/ $O_{p}(G)=\gen1$, then there exist $P,\,Q \in\Syl_{p}(G)$ such that $Z(P)\cap Z(Q)=\gen1$.
\end{probl}

\begin{proof} Take $P$ and $Q$ so that $|P\cap Q|$ is minimum. By Theorem~\ref{brodkey-general}, no nontrivial subgroup is normal in both $P$ and $Q$. In particular, $Z(P)\cap Z(Q)$ is trivial.  \end{proof}

\begin{probl}\label{problem-1.F.3}
    Let\/ $G=NP$, where\/ $N\normal G$, $P\in\Syl_p(G)$ and\/ $N\cap P=\gen1$, and assume that the conjugation action of\/ $P$ on\/ $N$ is faithful. Show that\/ $P$ acts faithfully on at least one orbit of this action.

    \textrm{\rm Hint. In fact, if $x\in N$ is chosen so that $|P\cap P^x|$ is as small as possible, then $P$ acts faithfully on the $P$-orbit containing $x$.}
\end{probl}

\begin{solution} First observe that $P$ can act on $N$ because $N$ is normal. Second, the property that this action is faithful translates into
\begin{equation*}
    P\cap C_G(N)=\gen1\tag{*}
\end{equation*}
Now, following the hint, chose $x\in N$ so that $|P\cap P^x|$ is minimal. Note that we may assume that $|P\cap P^x|$ realizes the minimum among the intersections $Q\cap R$ with $Q,\,R\in\Syl_p(G)$. To see this write $Q=P^a$ and $R=P^b$ for some $a,\,b\in G$. Hence, $|Q\cap R|= |P^a\cap P^b|=|P\cap P^c|$ for $c=a^{-1}b$. Then, write $c=xy$ with $x\in N$ and $y\in P$ and obtain $P^c=P^x$. In consequence, we can apply Theorem~\ref{brodkey-general} and conclude that $O_p(G)$ is the largest subgroup of $P\cap P^x$ that is normal in both $P$ and $P^x$.

The orbit ${\cal O}_x$ of the $P$-conjugation on $N$ is given by
$$
    {\cal O}_x = \set{x^y\mid y\in P}.
$$
To verify that $P$ acts faithfully on ${\cal O}_x$, assume that $z\in P$ satisfies $(x^y)^z=x^y$ for all $y\in P$. We have to prove that this implies $z=1$. 

Equivalently, we have to show that $z\in N$ or $z\in C_G(N)$. Our hypothesis means that $z\in C_G(x^y)$ for all $y\in P$, i.e., $z\in P\cap L$, where
$$
    L = \bigcap_{y\in P}C_G(x^y).
$$
Note that $P\cap L\normal P$: if $b\in P$, given $\zeta\in P\cap L$ and $y\in P$, we have
$$
    \zeta^bx^y = \zeta^bx^{b^{-1}by}=(\zeta x^{b^{-1}y})^b
        =(x^{b^{-1}y}\zeta)^b = x^y\zeta^b.
$$
Take $\zeta\in P\cap L$ and $b\in P$. Then $\zeta=\zeta^x\in P^x$ because $\zeta\leftrightarrow x$ and
$$
    \zeta^{b^x}=\zeta^{xbx^{-1}}=xbx^{-1}\zeta xb^{-1}x^{-1}
        = x\zeta^bx^{-1}=\zeta^b,
$$
where the last equality holds because $\zeta^b\in L$. Consequently, $P\cap L\normal P^x$.

The last two facts imply that $P\cap L\subseteq O_p(G)$. Since every element of $\Syl_p(L)$ is the intersection of $L$ with an element of $\Syl_p(G)$, it follows that
$$
    P\cap O_p(L) \supseteq P\cap L\cap O_p(G) = P\cap L \supseteq P\cap O_p(L),
$$
i.e., $P\cap L=P\cap O_p(L)=L\cap O_p(G)$.

Therefore, $z\in O_p(G)$. But $N\cap O_p(G)\subseteq N\cap P=\gen1$ and both groups are normal, which implies, by Problem~\ref{problem-1.F.1}, that $z\in O_p(G)\subseteq C_G(N)$.  \end{solution}

\section{Chermak-Delgado Theorem}

\begin{defn}
    Let\/ $G$ be a finite group. Given a subgroup $H$, its \textsl{Chermak-Delgado measure} is the integer
    $$
        m_G(H) = |H||C_G(H)|.
    $$
\end{defn}

\begin{defn}
    A collection of subgroups of a group $G$ is a \textsl{lattice} if it is closed under intersections and \textsl{joins}, where the \textsl{join} of two subgroups $H$ and $K$ is $\gen{H\cup K}$, also written $\gen{H,K}$.
\end{defn}

\begin{lem}\label{m_G-inequality-1}
    With the preceding notations, $m_G(H)\le m_G(C_G(H))$ and equality holds if, and only if, $H=C_G(C_G(H))$.
\end{lem}

\begin{proof} Given that $H\subseteq C_G(C_G(H))$, we have
$$
    m_G(C_G(H))=|C_G(H)||C_G(C_G(H))|\ge |C_G(H)||H|=m_G(H),
$$
with equality is attained if, and only if, $H=C_G(C_G(H))$.  \end{proof} 

\begin{lem}\label{m_G-inequality-2}
    Let\/ $H$ and\/ $K$ be subgroups of a finite group $G$. Then
    $$
        m_G\gen{H,K} \ge \frac{m_G(H)m_G(K)}{m_G(H\cap K)},
    $$
    where equality is attained if, and only if, $C_G(H\cap K) = C_G(H)C_G(K)$ and\/ $\gen{H,K} = HK$.
\end{lem}

\begin{proof} With obvious notations, first observe that $C_{\gen{H,K}}=C_H\cap C_K$ and second that $C_H\cup C_K\subseteq C_{H\cap K}$, which implies, $C_HC_K\subseteq C_{H\cap K}$. Therefore,
\begin{align*}
    m_G(H\cap K) &= |H\cap K||C_{H\cap K}|\\
        &\ge\frac{|H||K|}{|HK|}|C_HC_K|\\
        &\ge \frac{|H||K|}{|\gen{H,K}|}\frac{|C_H||C_K|}{|C_{\gen{H,K}}|}\\
        &= \frac{m_G(H)m_G(K)}{m_G\gen{H,K}},
\end{align*}
proving the first statement. The second is also clear.  \end{proof}

\begin{thm}\label{stronger-chermak-delgado}
    Given a finite group\/ $G$, let\/ ${\cal L} = {\cal L}(G)$ be the collection of subgroups of\/ $G$ for which the Chermak-Delgado measure is as large as possible. Then:
    \begin{enumerate}[\rm a)]
    \item $\cal L$ is a lattice.
    \item If\/ $H,K\in\cal L$, then $\gen{H,K} = HK$.
    \item If\/ $H\in\cal L$, then $C_G(H)\in\cal L$ and $C_G(C_G(H)) = H$.
    \end{enumerate}
\end{thm}

\begin{proof} Let $m$ denote the maximum value of the Chermak-Delgado measure on~$G$.

Take $H,\,K\in\cal L$. Then, by Lemma \ref{m_G-inequality-2}, we have
$$
    m^2 = m_G(H)m_G(K) \le m_G\gen{H,K}m_G(H\cap K) \le m^2,
$$
where the second inequality is a direct consequence of the definition of $m$. Thus, $m_G\gen{H,K}=m$ and $m_G(H\cap K)=m$, which shows that $\cal L$ is, in fact, a lattice (part a).

In addition, the same Lemma implies that $\gen{H,K}=HK$ (part b)

Part c) follows from Lemma \ref{m_G-inequality-1}.  \end{proof}

\begin{cor}\label{m_G-is-max}
   Every finite group\/ $G$ contains a unique subgroup\/ $M$ that is minimal among all subgroups with the property that\/ $m_G(M)$ is the maximum of the Chermak-Delgado measures of the subgroups of $G$. Moreover, $M$ is abelian and contains the center\/ $Z(G)$ of\/ $G$.
\end{cor}

\begin{proof} With the notations of the theorem, let $M=\bigcap\cal L$. Since $\cal L$ is a lattice, $M\in\cal L$ and is minimal in this collection. It is abelian because $M$, being included in all members of $\cal L$, it also included in $C_G(M)$, which belongs in $\cal L$ according to the theorem. Finally, $Z(G)\subseteq C_G(C_G(M))=M$.  \end{proof}

\begin{defn}
    The subgroup\/ $M$ of the corollary is called the \textsl{Chermak-Delgado} subgroup of\/ $G$ and denoted\/ $\CD(G)$.
\end{defn}

\begin{cor}\label{chermak-delgado} {\rm[Chermak-Delgado]}
    Let\/ $G$ be a finite group. Then\/ $G$ has a characteristic abelian subgroup\/ $M$ such that\/ $|G:M| \le |G:A|^2$ for every abelian subgroup $A\subseteq G$.
\end{cor}

\begin{proof} Let $M=\CD(G)$ be the Chermak-Delgado subgroup of $G$. Then $M$ is characteristic because automorphisms preserve the Chermak-Delgado measure. By the previous corollary, $M$ is abelian. Moreover, if $A\subseteq G$ is abelian, then $A\subseteq C_G(A)$ and so $m_G(M)\ge m_G(A)=|A||C_G(A)|\ge |A|^2$, by Corollary~\ref{m_G-is-max}. Hence,
$$
    |G:M| = \frac{|G|}{|M|}=\frac{|G||C_G(M)|}{m_G(M)}
        \le\frac{|G||G|}{|A|^2} = |G:A|^2.
$$
 \end{proof}

\begin{cor}\label{not-nonabelian-simple}
    Let\/ $H$ be a subgroup of a finite group\/ $G$, and assume that\/ $m_G(H)>|G|$. Then\/ $G$ is not a nonabelian simple group.
\end{cor}

\begin{proof} Let $M=\CD(G)$ and suppose that $G$ is nonabelian and simple. First observe that $m_G(H)>|G|\ge1$ implies $G\ne\gen1$. Since $M$ is abelian, $M$ is proper and given that it is characteristic, is normal, hence $G$ is not simple.  \end{proof}

\subsection{Problems G}

\begin{probl}\label{problem-1.G.1}
    Let\/ $A\subseteq G$, where\/ $A$ is abelian, and assume that there doesn't exist a characteristic abelian subgroup\/ $N$ of\/ $G$ such that\/ $|G:N| < |G:A|^2$. Show that\/ $A=C_G(A)$ is a member of the maximum-measure lattice ${\cal L}(G)$ and that\/ $|G:Z(G)| = |G:A|^2$.
\end{probl}

\begin{solution} Let $M=\CD(G)$. The hypothesis implies that $|G:M|=|G:A|^2$ and the proof of Corollary~\ref{chermak-delgado} shows that equality is attained if $|A|=|C_G(A)|$, $m_G(M)=m_G(A)$ and $|G|=|C_G(M)|$. Since $A$ is abelian, the first equality shows that $A=C_G(A)$, the second that $A\in{\cal L}(G)$ and the third that $M=Z(G)$. In particular, $|G:Z(G)|=|G:M|=|G:A|^2$.  \end{solution}

\begin{probl}\label{problem-1.G.2}
    Let\/ ${\cal L}(G)$ be the maximum-measure lattice and suppose that\/ $H \in{\cal L}(G)$ and that\/ $H\varsubsetneq G$. Show that there exists a normal subgroup\/ $M$ of\/ $G$ such that\/ $H \subseteq M \varsubsetneq G$.

    \textrm{\rm Hint. Observe that all conjugates of $H$ in $G$ lie in ${\cal L}(G)$ and thus the product of any two of them is a subgroup. If $H$ is not contained in a proper normal subgroup of $G$, show that it is possible to write $G=HK$, where $K\varsubsetneq G$ and $K$ contains a conjugate of $H$. Deduce a contradiction from this.}
\end{probl}

\begin{solution} Put $M=\CD(G)$, $m=m_G(M)$ and ${\cal L}={\cal L}(G)$. Given $x\in G$, $C_G(H^x)=C_G(H)^x$ and therefore $m_G(H^x)=m$, i.e., $H^x\in\cal L$. In particular, by Theorem~\ref{stronger-chermak-delgado}, $H^xH^y = \gen{H^x,H^y}$ is a subgroup of $G$ for any pair of elements $x,\,y\in G$.

Suppose that $H$ is not included in any proper normal subgroup of $G$.

Given a subset $X\subseteq G$, define $H^X$ as 
$$
    H^X = \prod_{x\in X}H^x.
$$
Then $H^G=G$ because $H^G$ is normal and includes $H$. Therefore, $G=HH^T$ for $T=\set{c\in G\mid H^c\ne H}$. Let $S$ be a set with minimum order satisfying $S\subseteq T$ and $HH^S=G$. Note that $S\ne\emptyset$ and $H^a\ne H^b$ if $a\ne b$, $a,b\in S$. 

We claim that $H^S\ne G$. Otherwise, pick $s_0\in S$ and observe that
$$
    G = G^{s_0^{-1}} = H^{s_0^{-1}S} = \prod_{s\in S}H^{s_0^{-1}s}
        =H\prod_{s\in S\setminus\set{s_0}}H^{s_0^{-1}s}.
$$
Since ---in addition--- $s_0^{-1}S\setminus\set1\subseteq T$, we arrive at a contradiction because $S$ had minimum order with that property. Thus, we have found $K$, namely $H^S$, satisfying $G=HK$, $K\varsubsetneq G$ and $K\supseteq H^x$ for, at least, one $x\in G$. This is a contradiction because, by Problem~\ref{problem-1.A.4}, we can write $G=H^xK=K$.  \end{solution}

\begin{probl}
    Let\/ $G$ be simple and suppose that\/ $H \subseteq G$ and\/ $m_G(H)=|G|$. Show that $H=\gen1$ or $H=G$.

    \textrm{\rm Hint. If $G$ is nonabelian, show that $H \in{\cal L}(G)$}.
\end{probl}

\begin{solution} If $G$ is nonabelian then $Z(G)=\gen1$ because $G$ is simple and $Z(G)$ is characteristic, hence normal. By Corollary~\ref{not-nonabelian-simple}, $m_G(\CD(G)) \le |G|=m_G(H)$, which means that $H\in{\cal L}(G)$. If $H\ne G$, the previous problem implies the existence of a proper normal group $M$ including $H$, impossible because $G$ is simple, except in the case where $H=\gen1$.  \end{solution}

\begin{probl}
    Let\/ $A \subseteq G$, where\/ $A$ is abelian and\/ $G$ is nonabelian. Show that there exists a normal abelian subgroup\/ $N$ of\/ $G$ such that\/ $|G:N|<|G:A|^2$.
\end{probl}

\begin{solution} Suppose, toward a contradiction, that there is no such a normal abelian subgroup. In particular, there is no characteristic normal subgroup satisfying the inequality. Hence, we can apply Problem~\ref{problem-1.G.1}, to deduce that $A\in{\cal L}(G)$. In consequence, we can invoke Problem~\ref{problem-1.G.2} to obtain $N\normal G$ such that $A\subseteq N\varsubsetneq G$.

There are two possibilities. In the first one there exists $B\ch N$ abelian such that $|N:B|<|N:A|^2$. In this case,
\begin{align*}
    |G:B| &= |G:N||N:B|\\
        &< |G:N||N:A|^2\\
        &< (|G:N||N:A|)^2\\
        &= |G:A|^2  &&;\ A\subseteq N,
\end{align*}
with $B\normal G$ because $B\ch N\normal G$, and we are done.

Otherwise, if no characteristic abelian subgroup $B$ of $N$ satisfies the inequality $|N:B|<|N:A|^2$, from Problem~\ref{problem-1.G.1} applied to $G=N$, we deduce that $|N:Z(N)|=|N:A|^2$. Therefore,
\begin{align*}
    |G:Z(N)|&=|G:N||N:Z(N)|\\
        &= |G:N||N:A|^2 \\
        &< (|G:N||N:A|)^2   &&;\ N\varsubsetneq G\\
        &= |G:A|^2,  &&;\ A\subseteq N
\end{align*}
where $Z(N)\normal G$ because $Z(N)\ch N\normal G$.  \end{solution}
