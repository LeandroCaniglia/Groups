\chapter{Vector Spaces}

\section{Bases}\label{chap:bases}

\begin{thm}
    Let\/ $\kappa$ be a field. Every\/ $\kappa$-vector space\/ $\mathbb V$ has a basis.
\end{thm}

\begin{proof}
    Consider the family $\mathcal L$ of subsets of $\mathbb V$ whose members are linearly independent collections of vectors. Since $\emptyset\in\mathcal L$, $\mathcal L\ne\emptyset$.

    Let $\mathcal F$ be a filtrant subfamily of $\mathcal L$ (with respect to the inclusion). We claim that $\bigcup\mathcal F\in\mathcal L$. To see this suppose that
    $$
        a_1v_1+\cdots+a_nv_n=0,
    $$
    where $a_1,\dots,a_n\in\kappa$ and $v_i\in B_i$ for some $B_i\in\mathcal F$. If $B\in\mathcal F$ satisfies $B_i\subseteq B$ for all $1\le i\le n$, then all the $v_i\in B$. Since the elements of $B$ are linearly independent, we deduce that $a_i=0$ for $1\le i\le n$.

    By Zorn's lemma, it follows that $\mathcal L$ has a maximal member, say $B$. The subspace generated by $B$ is $\mathbb V$, otherwise any $v\in\mathbb V$ not generated by $B$ would produce a member of $\mathcal L$, namely $B\cup\set v$, which contradicts the maximality of~$B$.
\end{proof}

\begin{thm}
    Let\/ $\kappa$ be a field and $\mathbb V$ a $\kappa$-vector space. Then two bases of $\mathbb V$ have the same cardinal.
\end{thm}

\begin{proof}
    The result is well-known in the finite dimensional case, so we may assume that~$\mathbb V$ has infinite dimension.

    Let $B=(v_i)_{i\in I}$ and $B'=(w_j)_{j\in J}$ be two bases of $\mathbb V$. We can define the map
    \begin{align*}
        \phi\colon I&\to\bigcup_{n=1}^\infty J^n\\
        i&\mapsto(j_{i_1},\dots,j_{i_n}),
    \end{align*}
    where $v_i$ is generated by $\set{w_{j_{i_1}},\dots,w_{j_{i_n}}}$ and no subset of it. Since $J$ is infinite, $|J^n|=|J|$ and so the union has the cardinal $\aleph_0|J|=|J|$. Since $\phi$ is injective, we get $|I|\le|J|$.
\end{proof}