\chapter{Subnormality}
\section{Minimal Normal Subgroups}

\begin{defn}\label{socle-defn}
    The \textsl{socle} of a finite group\/ $G$ is the subgroup $\Soc(G)$ generated by all minimal normal subgroups of\/ $G$, i.e.,
    $$
        \Soc(G) = \prod\Nm(G),
    $$
    where\/ $\Nm(G)$ denotes the family of all minimal normal subgroups. Here, of course, \textsl{minimal} implies nontrivial. In particular, $\Soc\gen1=\prod\emptyset=\gen1$.
\end{defn}

\begin{rems}\label{soc-intersection}${}$
    \begin{enumerate}[\rm i)]
        \item If $N$ is normal and nontrivial, then $N$ includes some minimal normal group and therefore $N\cap\Soc(G)\ne\gen1$.
        \item $\Soc(G)$ is characteristic\/\textrm{\rm: automorphisms preserve inclusions and normality.}
        \item $G$ simple $\implies\Soc(G) = G$.
        \item $\Soc(G)$ can be expressed as the direct product of a subfamily of\/ $\Nm(G)$\textrm{\rm: a running direct product of minimal normal subgroups only increases with the next minimal normal subgroup when such a subgroup has trivial intersection with said product [cf.~Theorem~\ref{product-of-minimal-normal}].}
    \end{enumerate}
\end{rems}

\begin{ntn}
    We will write $M\normal_m G$ to indicate that $M$ is minimal normal in~$G$.
\end{ntn}


\begin{prop}
    Let\/ $M$ be a minimal normal subgroup of\/ $G$.
    \begin{enumerate}[\rm a)]
        \item If\/ $N\normal G$ then\/ $M \subseteq N$ or\/ $M \cap N = \gen1$. In the latter case, $[M, N] = \gen1$.
        \item If\/ $M$ is abelian then\/ $M \subseteq H$ or\/ $M \cap H = \gen1$ for any subgroup\/ $H$ of\/ $G$ with\/ $G = MH$.
        \item If\/ $\phi\colon G\to H$ is an epimorphism then\/ $\phi(M) = \gen1$ or\/ $\phi(M)\normal_mH$.
    \end{enumerate}
\end{prop}

\begin{proof}${}$
\begin{enumerate}[\rm a)]
    \item Since $M\cap N\normal_mG$, the first conclusion is a direct consequence of the definition. The second simply says that $M\leftrightarrow N$ when $M\cap N=\gen1$.

    \item By Remark~\ref{abelian-normal}, $M\cap H\normal MH=G$. The conclusion follows because $M\cap H\subseteq M$.

    \item By Proposition~\ref{image-of-normalizer} we get $\phi(M)\normal H$. Suppose that $\phi(M)\ne\gen1$. Take $L\normal H$ satisfying $L\subseteq\phi(M)$. Then $\phi^{-1}(L)\normal G$ by Remark~\ref{biunivocal-normal-quotient}. Since $\phi^{-1}(L)\cap M\normal G$ is a subgroup of $M$, $\phi^{-1}(L)=\gen1$ or $M\subseteq\phi^{-1}(L)$. In the first case, $L=\phi(\phi^{-1}(L))=\gen1$. In the second, $\phi(M)\subseteq\phi(\phi^{-1}(L))=L$.
\end{enumerate}
\end{proof}

\begin{thm}\label{product-of-minimal-normal}
    Let\/ $\mathcal N\subseteq\Nm(G)$ be a finite set of minimal normal subgroups of\/ $G$ and let
    $$
        X = \prod\mathcal N.
    $$
    \begin{enumerate}[\rm a)]
    \item If\/ $N$ is a normal subgroup of\/ $G$, then there exist\/ $M_1, \dots, M_n \in\mathcal N$ such that
    $$
        NX = N \times M_1 \times \cdots \times M_n.
    $$
    \item There exist\/ $M_1, \ldots, M_r \in\mathcal N$ such that
    $$
        X = M_1 \times \cdots \times M_r.
    $$
    \end{enumerate}
\end{thm}

\needspace{2\baselineskip}
\begin{proof}${}$
\begin{enumerate}[\rm a)]
    \item Take a maximal subset $\cal L$ of $\mathcal N$ such that $Q=N\prod\cal L$ is a direct product. If $Q\ne NX$ there must exist $M\in\mathcal N$ such that $M\not\subseteq Q$. Since $M\normal_mG$ and $Q\normal G$ it follows that $Q\cap M=\gen1$. According to Theorem~\ref{direct-product} this implies that $QM$ is a direct product, contradicting the maximality of~$\cal L$.

    \item This is nothing but a) applied to $N=\gen1$.
\end{enumerate}
\end{proof}


\begin{cor}\label{minimal-normal-of-minimal-normal}
    Let\/ $N$ be minimal normal in\/ $G$ and\/ $M$ minimal normal in\/~$N$. Assume that the set\/ $\set{M^x \mid x \in G}$ is finite. Then\/ $M$ is simple and there exist\/ $x_1, \dots, x_n$ in\/ $G$ such that
    \begin{equation}\label{eq1.7}
        N = M^{x_1} \times \cdots \times M^{x_n}.
    \end{equation}
\end{cor}

\begin{proof} The product $\prod_{x\in G}M^x$ is a subgroup of $N$ because $M^x\normal N^x=N$ for $x\in G$. And given that it is normal in $G$, it must equal $N$. By the theorem, equation $(\ref{eq1.7})$ is attained for some $x_1,\dots,x_n\in G$. In particular, $M^{x_i}\leftrightarrow M^{x_j}$ for $i\ne j$. Therefore, given $L\normal M$, we get $L^{x_1}\normal M^{x_1}$ with $L^{x_1}\leftrightarrow M^{x_i}$ for $i\ne 1$. Thus, $L^{x_1}\normal N$. It follows that $L\normal N$, which implies that $L$ is trivial or $M$ because $L\subseteq M$ and $M\normal_mN$.  \end{proof}

\begin{cor}\label{abelian-minimal-normal-is-elementary}
    Let\/ $A$ be an abelian minimal normal subgroup of the finite group\/ $G$. Then there exists a prime\/ $p$ such that\/ $A$ is a direct product of subgroups that are isomorphic to\/ $\Z_p$.
\end{cor}

\begin{proof} In the case where $A$ is abelian, $M\normal_mA\iff M\cong\Z_p$ for some prime $p\mid|A|$. Thus, the previous corollary implies that $A$ is a direct product of copies of~$\Z_p$.  \end{proof}

\begin{defn}
    Let\/ $G$ be a group. A subgroup\/ $H$ is said to be \textsl{subnormal} in\/ $G$ when there exists a finite sequence
    $$
        H = H_0 \normal H_1 \normal \cdots \normal H_r = G.
    $$
    In such a case we write $H\snormal G$.
\end{defn}

\begin{thm}\label{semisimple-normal}
    Let\/ $G = G_1 \times \cdots \times G_n$ and\/ $N$ a normal subgroup of\/ $G$.
    \begin{enumerate}[\rm a)]
    \item If\/ $N$ is perfect, then\/ $N = (N \cap G_1) \times \cdots \times (N \cap G_n)$.
    \item If\/ $G_1, \dots, G_n$ are nonabelian simple groups, then\/ $N$ is perfect and there exists a subset\/ $J \subseteq \nset n$ such that
    $$
        N = \prod_{j \in J} G_j\quad\text{\rm and}\quad G_k \cap N = 1\text{\rm\ for }k \notin J.
    $$
    \item If\/ $G_1, \dots, G_n$ are nonabelian simple groups and\/ $H\snormal G$, then\/ $H\normal G$, and part\/~{\rm b)} applies on\/ $H$ as well.
    \end{enumerate}
\end{thm}

\needspace{2\baselineskip}
\begin{proof}${}$
\begin{enumerate}[\rm a)]
    \item By Proposition~\ref{commutator-props} we have
    $$
        N=N'=[N,N]\subseteq[N,G] = \prod_{i=1}^n[N,G_i]\subgroup\prod_{i=1}^nN\cap G_i
            \subseteq N.
    $$
    \item By part a) it suffices to show that $N$ is perfect. We will prove this by induction on $n$.

    The case $n=1$ is trivial because $N=G_1$ is perfect. Let's assume that $n>1$.
    
    If $N=G$ there is nothing to prove. Assume $N\ne G$. Then there exists $k$ such that $G_k\not\subseteq N$. Since $N\cap G_k\normal G_k$, by simplicity $N\cap G_k=\gen1$. Put $\bar N=NG_k/G_k$, $\bar G_i=G_iG_k/G_k$ and $\bar G=G/G_k$.

    Note that $\bar G_i\cong G_i$ for $i\ne k$, and therefore $\bar G_i$ is simple and nonabelian. Moreover,
    $$
        \bar G = \prod_{i\ne k}\bar G_i
    $$
    because, for $i\ne j$ both different from $k$, we have
    \begin{align*}
        \bar x\in \bar G_i\cap \bar G_j &\iff x\in G_iG_k\cap G_jG_k\\
            &\iff x\in G_k   &&\text{; Thm.~\ref{direct-product}}\\
            &\iff \bar x= 1.
    \end{align*}
    Since $\bar N\normal\bar G$, the inductive hypothesis implies that $\bar N$ is perfect. Therefore,
    $$
        N\cong NG_k/G_k = (NG_k/G_k)' = N'G_k/G_k \cong N',
    $$
    where both isomorphisms hold because $N'\cap G_k\subseteq N\cap G_k=\gen1$. Then $|N|=|N'|$, and the inclusion $N'\subseteq N$ implies $N=N'$.

    \item It is enough to show that $H$ is decomposable as a direct product of a subfamily of the $G_i$. Take a sequence
    $$
        H=H_0\normal H_1\normal\cdots\normal H_r=G.
    $$
    If $r\le 1$ then $H\normal G$ and we are done. If $r>1$ then $H_{r-1}\normal G$. Therefore, according to part~b), $H_{r-1}$ is decomposable as the direct product of a subfamily of the $G_i$. By induction on $r$, we deduce that $H$ is the direct product of a subsubfamily of the $G_i$.
\end{enumerate}
\end{proof}

\begin{cor}
    Let\/ $N$ be a nonabelian minimal normal subgroup of the finite group\/ $G$.  Then
    \begin{enumerate}[\rm a)]
        \item The elements of\/ $\Nm(N)$ are nonabelian simple groups, which are conjugate in\/ $G$.
        \item $N = \bigtimes\Nm(N)$.
        \item For every\/ $M\normal N$ the collection
        $$
            \Nm(N;M)=\set{L\in\Nm(N)\mid L\subseteq M}
        $$
        satisfies
        $$
            \Nm(N;M) = \Nm(M).
        $$
        In particular,
        $$
            M = \bigtimes\Nm(N;M)
        $$
    \end{enumerate}
\end{cor}

\needspace{2\baselineskip}
\begin{proof}${}$
\begin{enumerate}[\rm a)]
    \item Take $M\in\Nm(N)$. Corollary~\ref{minimal-normal-of-minimal-normal} implies that $M$ is simple and that equation~$(\ref{eq1.7})$ holds. In particular $M$ is nonabelian. If $L$ is another element in $\Nm(N)$, from part~b) of Theorem~\ref{semisimple-normal} (applied to $G=N$ and $N=L$) we deduce that $L$ is a direct product of some of the factors $M^{x_i}$ occurring in~$(\ref{eq1.7})$. But given that $L\normal_mN$, such product can only have one factor, i.e., $L$ and $M$ are conjugate in $G$.

    \item Firstly note that the product $\prod\Nm(N)$ is direct. In addition, according to Corollary~\ref{minimal-normal-of-minimal-normal}, it equals $N$.

    \item By definition $\Nm(N;M)\subseteq\Nm(M)$. To verify the other inclusion take $L\in\Nm(M)$. Then $L\normal_m M\normal N$. It follows that $L\snormal N$. Thus, according to part~c) of Theorem~\ref{semisimple-normal}, $L\normal N$ and so $L\in\Nm(N;M)$. The last equality now follows from part~b) of the same theorem for $G=N$ and $N=M$.
\end{enumerate}
\end{proof}

\subsection{Exercises - Kurzweil \& Stellmacher - \S 1.7}

\begin{exr}
     Let\/ $G$ be a finite group and\/ $M$ a maximal subgroup of\/ $G$. All minimal normal subgroups\/ $N$ of\/ $G$ that satisfy\/ $N\cap M=\gen1$ are isomorphic.
\end{exr}

\begin{solution} {[See also \href{https://math.stackexchange.com/a/2694761/269050}{this MSE answer}]} Let's say that a subgroup $X$ of $G$ is \textsl{good\/} if $X\normal_mG$ and $X\cap M=\gen1$. If $X$ is good then $G=XM$ and $M\cong G/X$. Let $N$ and $L$ be two good subgroups of $G$. To show that $N\cong L$ we may assume that $N\ne L$. It follows that $N\cap L=\gen1$ and, consequently, $N\leftrightarrow L$. Let `$\sim$' be the relation on $N\times L$ defined as
\begin{equation}\label{eq1.7.1}
    a \sim b\iff ab\in M.
\end{equation}
Given $a\in N$ we can write $a=zb$ with $b\in L$ and $z\in M$. Thus, $ab^{-1}=z\in M$, i.e., $a\sim b^{-1}$. In particular, $\dom(\sim)=N$.

If $a\sim b_1$ and $a\sim b_2$, by definition, both $ab_1$ and $ab_2$ belong to $M$, which implies
$$
    b_1^{-1}b_2 = b_1^{-1}a^{-1}ab_2 = (ab_1)^{-1}(ab_2)\in M.
$$
Since $b_1b_2^{-1}$ is clearly in $L$, we get $b_1^{-1}b_2\in M\cap L=\gen1$, i.e., $b_1=b_2$. Thus, the relation is actually a function from $N$ to $L$.

To verify that $\sim$ is a morphism of groups take $a_1\sim b_1$ and $a_2\sim b_2$. Then
$$
    (a_1a_2)(b_1b_2) = (a_1b_1)(a_2b_2)\in M,
$$
i.e., $a_1a_2\sim b_1b_2$.

The morphism is mono because $a\sim 1$ means $a\in M$, which implies $a=1$. It is epi because given $b\in L$ we can write $b^{-1}=za$ for appropriate $z\in M$ and $a\in N$ which implies $ab=z^{-1}\in M$, i.e., $a\sim b$.

In sum, $(\ref{eq1.7.1})$ defines an isomorphism from $N$ onto $L$.  \end{solution}

\begin{exr}
    Let\/ $G$ be a finite group and\/ $M$ a maximal subgroup of\/ $G$. If\/ $M$ is nonabelian and simple, then there exist at most two minimal normal subgroups in\/~$G$.
\end{exr}

\begin{solution} Let $N\normal_mG$, $N\ne M$. Since $N\cap M\normal M$, we get $N\cap M=\gen1$. If $L\normal_mG$, $L\ne N$, then $NL/N\normal G/N$. But $G/N\cong M$ is simple and so $G=NL$, where the product is direct because $N\cap L=\gen1$. In addition, $N\cong NL/L\cong M$ is simple and the same goes for $L$. Should $H$ be minimal normal, Theorem~\ref{semisimple-normal} would imply $H=N$ or $H=L$. 

{\small \textbf{Another approach:} [\href{https://math.stackexchange.com/a/1188536/269050}{From a comment by Derek Holt}] Since $N\cap L=\gen1$, $N\leftrightarrow L$ and so $L\subseteq C_G(N)$. But equality is attained because $C_G(N)\cap N\normal N$ and $N\cong M$ is simple and nonabelian, which implies that $C_G(N)\cap N=\gen1$. Therefore, $L=C_G(N)$ is determined by~$N$.}

 \end{solution}

\begin{exr}
    In the conditions of the preceding exercise, give an example where\/ $G$ possesses two minimal normal subgroups.
\end{exr}


\begin{exr}
    Let\/ $G$ be a finite group and\/ $M$ a maximal subgroup of\/ $G$. Suppose that\/ $G$ contains two minimal normal subgroups, neither of which is contained in\/ $M$. Then every minimal normal subgroup of\/ $M$ is contained in\/ $\Soc(G)$, the product of all minimal normal subgroups of\/ $G$.

    \textrm{\rm Hint [l.c.]: Let $X\not\subseteq M$, $X\normal_m G$. Then $H\normal_m M\implies HX/X\normal_mG/X$. Also show that $[H,X]\normal G$.}
\end{exr}

\begin{solution} {[Brought from \href{https://math.stackexchange.com/a/1220274/269050}{MSE}]} Take $H\normal_mM$ and suppose that $H\not\subseteq\Soc(G)$. One first conclusion we can draw is that $H\cap\Soc(G)=\gen1$. In particular, 
\begin{equation}\label{eq1.7.4}
    H\nnormal G,
\end{equation}
otherwise, $H$ would contain a minimal normal subgroup of $G$.

Let's say that a subgroup $X$ of $G$ is \textsl{good} if $X\normal_mG$ and $X\not\subseteq M$. The hint's proof is divided into three claims. Bellow we will refer to two different good subgroups~$X$ and~$Y$, whose existence is a hypothesis.

\needspace{2\baselineskip}
\textbf{Claim 1:} $HX\normal G$

{\small Since $G=XM$, to prove the claim it is enough to verify that both $X$ and $M$ normalize $HX$. But this is clear because, for every $x\in X$, $(HX)^x\subseteq HX$ and, on the other hand, $M$ normalizes both $H$ and $X$.}

\medskip

\needspace{2\baselineskip}
\textbf{Claim 2:} $HX/X\normal_mG/X$

{\small Take $K\normal G$ be such that $X\subseteq K\subseteq HX$. To prove the claim we have to show that $K=X$ or $HX$. There are two possibilities $K\cap H=H$ or $\gen1$. The first means that $H\subseteq K$ and we are done. Thus, we may assume that $K\cap H=\gen1$. Given $b\in K$, write $b=ax$ for some $a\in H$ and $x\in X\subseteq K$. Then $a=bx^{-1}\in H\cap K=\gen1$, which implies that $b=x\in X$. Since $b$ was arbitrarily chosen and $X\subseteq K$, we deduce that $K=X$.}

\medskip

\textbf{Claim 3:} $[H,Y]\normal G$

{\small Since $Y\leftrightarrow X$ and $[H,Y]\subseteq Y$, we obtain $[H,Y]\leftrightarrow X$. In particular, $X$ normalizes $[H,Y]$. And since $M$ also normalizes $[H,Y]$, from the equation $G=XM$ we conclude that $[H,Y] \normal G$.
}

\medskip

\textbf{Claim 4:} $[H,Y]=Y$ and $[H,Y]X/X= HX/X$.

{\small Take $a\in H$, $y\in Y$. Write $y=xz$ with $z\in M$. We have
$$
    [a,y]=aya^{-1}y^{-1}=axza^{-1}z^{-1}x^{-1}
        = ax(a^{-1})^zx^{-1}\in HX.
$$
Therefore $[H,Y]X\subseteq HX$, which implies $[H,Y]X/X\subseteq HX/X$. 

Should $H\leftrightarrow Y$, both $Y$ and $M$ would normalize $H$ and so $H\normal G$, in contradiction with~($\ref{eq1.7.4}$). It follows that $[H,Y]$ is a nontrivial subgroup of $Y$.

Now observe that $[H,Y]\subseteq Y$. Since $\gen1\ne[H,Y]\normal G$ we get $[H,Y]=Y$ and $[H,Y]X/X=HX/X$ (by Claim~2).}

\medskip

\textbf{Conclusion.} From Claim 4 we deduce
$$
    HX/X=[H,Y]X/X = YX/X,
$$
i.e., $HX=YX$. It consequence,
$$
    H \subseteq HX = YX\in\Soc(G),
$$
which contradicts our initial assumption.  \end{solution}

\begin{exr}
    Let's say that a group is \textsl{good} whenever every minimal normal subgroup is contained in the center. Let\/ $G$ be good.
    \begin{enumerate}[\rm a)]
    \item If\/ $N$ and\/ $M$ are good normal subgroups of\/ $G$, then\/ $NM$ is good.
    \item Every normal subgroup of\/ $G$ is good.
    \end{enumerate}
\end{exr}

\begin{solution}
\begin{enumerate}[\rm a)]
    \item Fix any $H\normal_mNM$.
    
    \textbf{Claim 1: } $H\subseteq N\implies H\subseteq Z(N)$.
    
    {\small Assume that $H\subseteq N$ and pick $A\normal_mN$, $A\subseteq H$. By hypothesis $A\subseteq Z(N)$. Moreover, given $y\in M$, we have $A^y\normal N^y=N$. It follows that $A^M\normal N$. But $M$ normalizes $A^M$ and so $A^M\normal NM$. Since $A^M\subseteq H^M=H\normal_m NM$, we get $A^M=H$. Hence, to show that $H\subseteq Z(N)$ it is enough to verify that $A^y\subseteq Z(N)$. But this is clear because $Z(N)\ch N\normal G$ implies that $A^y\subseteq Z(N)^y=Z(N)$.}

    \medskip

    \textbf{Claim 2:} $H\subseteq N\implies H\subseteq Z(NM)$

    {\small Assume that $H\subseteq N$. If ---in addition--- $H\subseteq M$, Claim~1 applied to $M$ implies that $H\subseteq Z(M)$. It follows that $H\subseteq Z(NM)$. Otherwise, if $H\not\subseteq M$, given that both $H$ and $M$ are normal in $NM$, we deduce that $H\cap M=\gen1$ and so $H\leftrightarrow M$. Therefore, in this case we also have $H\subseteq Z(NM)$.}

    \medskip

    \textbf{Conclusion:} Claims 1 and 2 reduce the question to the case $H\cap N=\gen1$. By the symmetry between $N$ and $M$ we may also assume that $H\cap M=\gen1$. But then $H\leftrightarrow N$ and $H\leftrightarrow M$, which implies $H\subseteq Z(NM)$.

    \item {[Brought from \href{https://math.stackexchange.com/a/2703032/269050}{MSE}]} Let $N$ be a normal subgroup of $G$. Take $H\normal_mN$. Since $H^x\normal_mN^x=N$ for $x\in G$, we get $H^G\subgroup N$ with $H^G\normal G$. Take $A\normal_mG$, $A\subseteq H^G$. By hypothesis, $A\subseteq Z(G)$. By Theorem~\ref{product-of-minimal-normal}, we can write $H^G$ as a direct product
    $$
        H^G = H^{x_1}\cdots H^{x_n}.
    $$
    Pick $a\in A\setminus\set1$ and write it as
    $$
        a=c_1^{x_1}\cdots c_{n}^{x_n}
            \in H^{x_1}\times\cdots\times H^{x_n}.
    $$
    Since $a\ne1$, we may assume that $c_1^{x_1}\ne1$. Take $y\in N$, we have
    $$
        a = a^y = (c_1^{x_1})^y\cdots (c_n^{x_n})^y
    $$
    and so $(c_i^{x_i})^y=c_i$, i.e., $c_i^{x_i}\leftrightarrow y$ for all $i$. In particular, $c_1^{x_1}\leftrightarrow y$. Since $y$ was arbitrarily chosen, $1\ne c_1^{x_1}\in H^{x_1}\cap Z(N)$. Recalling that $H^{x_1}\normal_mN$, we get $H^{x_1}\subseteq Z(N)$ and therefore $H\subseteq Z(N)$ because $Z(N)\ch N\normal G$.
\end{enumerate}
\end{solution}


\section{Subnormal Subgroups}
\begin{defn}
    Let's recall that a subgroup\/ $H$ of a group\/ $G$ is subnormal when there exists a finite sequence
    $$
        H = H_0 \normal H_1 \normal \cdots \normal H_r = G.
    $$
    In such a case the smallest value of\/ $r$ for which such a sequence exists is the \textsl{subnormality depth} of\/ $H$, denoted by $\sd_G(H)$.
\end{defn}

\begin{rem}
    Note that when $H\snormal G$ the sequence can be chosen to be strictly increasing. Unlike normality, subnormality is a transitive relation.
\end{rem}

\begin{rem}
    Let\/ $N\subseteq H$ be subgroups of the group\/ $G$. If $N\normal G$ and the quotient $H/N$ is subnormal in $G/N$, then $H\snormal G$.
\end{rem}

\begin{lem}\label{nilpotent-lemma}
    Let\/ $G$ be finite. Then\/ $G$ is nilpotent if, and only if, every subgroup of\/ $G$ is subnormal.
\end{lem}

\begin{proof}${}$

\begin{description}
    \item[\rm\textit{if\/} part)] Let $H$ be a subgroup. Since $H\snormal G$, there is a strictly increasing sequence from $H$ to $G$, say $H\normal H_1\normal\cdots\normal H_r=G$. In particular, $H\varsubsetneq H_1\subseteq N_G(H)$. The conclusion now follows from Theorem~\ref{nilpotent-equivalences}.

    \item[\rm\textit{only if\/})] The same theorem implies that $H\varsubsetneq N_G(H)$. Put $H_1=N_G(H)$. If we repeat the same construction defining $H_{i+1}=N_G(H_i)$, we will eventually reach $G$.
\end{description}
\end{proof}

\begin{xmpl}
    The \textsl{Klein subgroup} of\/ $S_4$ can be represented by the three $2$-cycles, plus the identity 
    $$
        K = \set{(),(1 2)(3 4), (1 3)(2 4), (1 4)(2 3)}.
    $$
    \begin{enumerate}[\rm i)]
        \item The group is abelian{\rm: $K$ is the identity plus all elements in $S_4$ with order $2$ and sign $1$. Therefore, for $a$, $b$ and $c$ in $K$, $ab=c\ne()\implies ba=c$.}
        
        \item $K$ is normal{\rm: Indeed, conjugation preserves order and sign.}
        
        \item If\/ $H\subseteq K$, $|H|=2$, then $|N_{S_4}(H)|=8${\rm: $H\varsubsetneq N_{S_4}(H)$ because\/ $(ab)\in H$ implies $(cd)(ab)(cd)=(ab)$, which shows the existence of\/ $1$ additional element in\/ $N_{S_4}(H)$. Thus, $|N_{S_4}(H)|\ge3$, and so\/ $N_{S_4}(H)\ge4$. Since $(acd)(ab)(dca)=(bd)$, there is at least $1$ element of order $3$ that doesn't belong to $N_{S_4}(H)$. Given that $H\normal K$ because $K$ is abelian, we have $K\subseteq N_{S_4}(H)$. Moreover, the inclusion is strict because $(cd)$ is not in $K$. It follows that $|N_{S_4}(H)|=8$.}
    
        \item $H\snormal S_4${\rm: Since\/ $K$ is abelian, $H\normal K$ and by ii) $K\normal S_4$.}
    
        \item $N_{S_4}(H)\nnormal S_4${\rm: $N_{S_4}(H)\in\Syl_2(S_4)$ and $n_2(S_4)=|S_4:N_{S_4}(H)|=3$.}
    \end{enumerate}
    As a result, $H\snormal S_4$ but we cannot reach $S_4$ from $H$ taking stabilizers.
\end{xmpl}

\begin{thm}\label{nilpotent-and-subnormal}
    Let\/ $G$ be a finite group and\/ $H$ a subgroup. Then $H\subseteq F(G)$ if, and only if, $H$ is nilpotent and subnormal in\/ $G$.
\end{thm}

\needspace{1\baselineskip}
\begin{proof}${}$

\begin{description}
    \item[\rm{\it if\/} part)] By induction on $|G:H|$, the case $|G:H|=1$ is trivial because it happens when $H=G$, i.e., when $G$ is nilpotent, i.e., when $G=F(G)$ [cf.~Corollary~\ref{normal-nilpotent-fitting}]. Let $H_{r-1}$ be the next-to-last group in a strictly growing normal chain from $H$ to $G$. Since $|H_{r-1}:H|< |G:H|$, we can apply the inductive hypothesis and conclude that $H\subseteq F(H_{r-1})$. But $F(H_{r-1})\ch H_{r-1}\normal G$. Therefore, we can invoke Lemma~\ref{normal-transitivity} to get $F(H_{r-1})\normal G$. Then, $F(H_{r-1})\subseteq F(G)$ by the same corollary. Ergo, $H\subseteq F(G)$.

    \item[\rm{\it only if\/})] First observe that $H$ is nilpotent because $H\subseteq F(G)$ and any subgroup of a nilpotent group is nilpotent by Corollary~\ref{nilpotent-subgroups-and-quotients}. By Lemma~\ref{nilpotent-lemma}, $H$ is subnormal in $F(G)$. Since $F(G)\normal G$, the conclusion follows by subnormal transitivity.
\end{description}
\end{proof}

\begin{lem}\label{subnormal-restriction}
     If\/ $H \snormal G$ and $K \subgroup G$, then $H\cap K \snormal K$.
\end{lem}

\begin{proof} Take a normal sequence $H=H_0\normal H_1\normal\cdots\normal H_r=G$. By Proposition~\ref{prod-quotient}, $H_{i-1}\cap K\normal H_i\cap K$ for $i=1\dots r$. Therefore, the intersected chain is a normal chain from $H\cap K$ to $K$.  \end{proof}

\begin{prop}
    Let\/ $G$ be a group, and\/ $H$ and\/ $K$ subnormal in\/ $G$. Then $H\cap K$ is subnormal in\/~$G$.
\end{prop}

\begin{proof} This is a direct consequence of the lemma and the transitivity of subnormality because $H\cap K\snormal K\snormal G$.  \end{proof}


\begin{thm}\label{minimal-normal-normalizes-subnormal}
    Let\/ $H\snormal G$, where\/ $G$ is a finite group, and let $M\normal_mG$. Then $M \subseteq N_G(H)$. In other words, $\Soc(G)\subseteq N_G(H)$.
\end{thm}

\begin{proof} It proceeds by induction on $|G:H|$. If $|G:H|=1$, then $H=G$ is normal and there is nothing to prove. Take a strictly growing normal chain from $H$ to $G$ and let $H_{r-1}$ be its next-to-last subgroup. Let's analyze two cases:
\begin{description}
    \item[\small${[}M\cap H_{r-1}=\gen1{]}$] Since both $M$ and $H_{r-1}$ are normal, $MH_{r-1}=M\times H_{r-1}$. In particular [cf.~Theorem~\ref{direct-product}] $M\leftrightarrow H_{r-1}$ and so
    $$
        M \subseteq C_G(H_{r-1})\subseteq C_G(H) \subseteq N_G(H).
    $$
    
    \item[\small${[}M\cap H_{r-1}\ne\gen1{]}$] Firstly, $M\cap H_{r-1}$, being normal, must equal $M$ because $M$ is minimal with that property. Thus, $M\subseteq H_{r-1}$ and so $M\normal H_{r-1}$. 
    
    Secondly, the inductive hypothesis implies that $\Soc(H_{r-1})\subseteq N_{H_{r-1}}(H)$. Since $N_{H_{r-1}}(H)\subseteq N_G(H)$, it follows that $\Soc(H_{r-1})\subseteq N_G(H)$.
    
    Finally, $\gen1\ne M\normal H_{r-1}$. Hence, $M\cap\Soc(H_{r-1})\ne\gen1$ by Remark~\ref{soc-intersection}. Since $\Soc(H_{r-1})\ch H_{r-1}\normal G$, by Lemma~\ref{normal-transitivity}, we obtain $\Soc(H_{r-1})\normal G$. Then $M\cap\Soc(H_{r-1})\normal G$. But $M\cap\Soc(H_{r-1})\subseteq M$ and therefore, $M\cap\Soc(H_{r-1})=M$. Therefore, $M \subseteq \Soc(H_{r-1})\subseteq N_G(H)$.
\end{description}
\end{proof}

\begin{thm}\label{subnormal-lattice}
    If\/ $G$ is a finite group, the collection of its subnormal subgroups is a lattice.
\end{thm}

\begin{proof} It suffices to show that if $H,\,K \snormal G$ then\/ $\gen{H,K} \snormal G$. The proof works by induction on $|G|$. The case $|G|=1$ is trivial, so we proceed with the case $|G|>1$. Let $M$ be a minimal normal subgroup of $G$. Consider the quotient $\bar G=G/M$ and the images $\bar H$ and $\bar K$ on it. The inductive hypothesis implies that $\gen{\bar H,\bar K}\snormal\bar G$. It follows that $\gen{H,K}M\snormal G$. By Theorem~\ref{minimal-normal-normalizes-subnormal}, we have $M\subseteq N_G\gen{H,K}$. Then $\gen{H,K}\normal \gen{H,K}M\snormal G$.  \end{proof}

\begin{thm}\label{zipper-lemma} {\rm[Zipper Lemma]}
    Suppose that\/ $H$ is a subgroup of a finite group\/ $G$ and assume that\/ $H\snormal K$ for every proper subgroup\/ $K$ of\/ $G$ that contains\/ $H$. If\/ $H$ is not subnormal in\/ $G$, then there is a unique maximal subgroup of\/ $G$ that contains\/ $H$.
\end{thm}

\begin{proof} Let's work by induction on $|G:H|$. If $|G:H|=1$, then $H=G$ and there is nothing to prove. Consider the case $|G:H|>1$. We may suppose that $H$ is not subnormal in $G$. In particular $N_G(H)\varsubsetneq G$ and there exists $M$ maximal such that $N_G(H)\subseteq M$. Thus, to complete the proof, it is enough to show that no other maximal includes $H$. Suppose otherwise and let $L$ be such other maximal. By hypothesis $H\snormal L$.

If $H\normal L$, then $L\subseteq N_G(H)\subseteq M$, i.e., $L=M$ and we are done.

We are left with the case $H\not\normal L$. It follows that any normal chain from $H$ to $L$ with the shortest length, say
$$
    H = H_0\normal H_1\normal\cdots\normal H_{r-1}\normal H_r = L,
$$
has $r\ge 2$. Note also that $H\not\normal H_2$, otherwise we could remove $H_1$ and get a shorter chain. Pick $x\in H_2$ so that $H^x\ne H$ and put $J=\gen{H,H^x}$. Since $H^x\subseteq H_1^x = H_1$, we have
$$
    J\subseteq H_1\subseteq N_G(H) \subseteq M.
$$
Let's say (within the scope of this proof) that a subgroup of $G$ is \textsl{good} when it satisfies the hypothesis of the theorem. Of course, $H$ is good. And given that every conjugation is an automorphism, $H^x$ is good too. We claim that $J$ is also good.

To verify the claim, let $K$ be a proper subgroup containing $J$. Then $H,\,H^x\subseteq K$ and therefore, $H\snormal K$ and $H^x\snormal K$ (both subgroups are good). By Theorem~\ref{subnormal-lattice}, $J\snormal K$ as needed.

Note also that $J$ is not subnormal in $G$, otherwise $H\normal J\snormal G$ and $H$ would be subnormal in $G$, which isn't.

But $|G:J|<|G:H|$ because we've chosen $x$ so that $H^x\ne H$. Therefore, we can use the inductive hypothesis and invoke the existence of a unique maximal above~$J$. However, $J$ is included in both $L$ and $M$, which allows us to conclude that $L=M$.  \end{proof}

\begin{rem}
    The proof of the following theorem uses that $HH^x=H^xH$ for all $x\in G$ implies that $H^G$ (notation defined in\/ \textrm{\rm Problem~\ref{problem-1.G.2}}) is a subgroup.
    
    \textrm{\rm To verify this in advance, observe that
    $$
        H^xH^y = \Big(HH^{x^{-1}y}\Big)^x = \Big(H^{x^{-1}y}H\Big)^x = H^yH^x.
    $$
    In consequence,
    $$
        (HH^x)H^y = HH^yH^x = H^y(HH^x),
    $$
    which easily leads to the desired conclusion}.
\end{rem}

\begin{thm}\label{conjugate-commutativity}
   Let\/ $H \subseteq G$ be a subgroup of a finite group\/ $G$, and assume that\/ $HH^x = H^xH$ for all\/ $x \in G$. Then $H \snormal G$.
\end{thm}

\begin{proof} By induction on $|G|$. The case $|G|=1$ is trivial. If $|G|>1$ we may suppose that $H$ is not subnormal in $G$. The inductive hypothesis implies that $H$ satisfies the hypothesis of the Zipper Lemma. Therefore, there is a unique maximal $M$ such that $H\subseteq M$. Since $H\varsubsetneq G$, by Problem~\ref{problem-1.A.4}, $HH^x\varsubsetneq G$. It follows that $HH^x\subseteq M$ for all $x\in G$ because no other maximal subgroup includes $H$. In consequence, $H^G\subseteq M$. But $H^G\normal G$ and the inductive hypothesis implies that $H\snormal H^G$. Then $H\snormal G$, a contradiction.  \end{proof}

\begin{thm}
    Let\/ $A$ be an abelian subgroup of a finite group\/ $G$, and assume that for every subgroup\/ $H$ with\/ $A \subseteq H \subseteq G$, we have\/ $|H:A|^2 \le |H:Z(H)|$. Then\/ $A \subseteq F(G)$.
\end{thm}

\begin{proof} By induction on $|G|$. The case $|G|=1$ is trivial. If $|G|>1$, by the inductive hypothesis we may assume that $A\subseteq F(H)$ for all $A\subseteq H\varsubsetneq G$ as the inequality of the statement doesn't depend on $G$. By Theorem~\ref{nilpotent-and-subnormal} it follows that $A\snormal H$. Thus, the Zipper Lemma implies that $H\snormal G$ or there is a unique maximal containing $A$. The first possibility implies $A\snormal G$ and, by Theorem~\ref{nilpotent-and-subnormal}, $A\subseteq F(G)$.

It remains to consider the case of a unique maximal $M$ containing $A$. We claim that there exists $x\in G$ such that $\gen{A,A^x}=G$. Suppose otherwise and, for every $x\in G$, take a maximal proper subgroup $M_x$ such that $\gen{A,A^x}\subseteq M_x$. Since $M_x$ is a maximal subgroup that includes $A$, we must have $M_x=M$. If $M$ includes all the $A^x$, then $A^G\subseteq M$ and the inductive hypothesis would imply $A\subseteq F(A^G)$. Then, once again, $A\snormal A^G$. Since $A^G\normal G$, we would reach a contradiction. Hence, there must be some $x\in G$ for which $\gen{A,A^x}=G$.

We claim that $A\cap A^x\subseteq Z(G)$. To verify this, peek $a=b^x\in A\cap A^x$ and a generator $cd^x\in AA^x$. Then
$$
    a(cd^x) = cad^x = cb^xd^x=c(bd)^x=c(db)^x=(cd^x)b^x=(cd^x)a
$$
and the inclusion follows. According to Problem~\ref{problem-1.A.4}, $|G|>|AA^x|$ and so
$$
    |G| > |AA^x| = \frac{|A|^2}{|A\cap A^x|}\ge \frac{|A|^2}{|Z(G)|},
$$
which contradicts the hypothesis in the case $H=G$. Thus, there is no unique maximal containing $M$ and the thesis is met.  \end{proof}

\subsection{Problems A}

\begin{probl}
    Let $\pi$ be a set of prime numbers, and recall that\/ $O_\pi(G)$ is the unique largest normal $\pi$-subgroup of\/ $G$. Show that\/ $O_\pi(G)$ contains every subnormal $\pi$-subgroup of\/ $G$. Conclude that the subgroup generated by two subnormal $\pi$-subgroups of\/ $G$ is itself a $\pi$-subgroup.
\end{probl}


\begin{solution}  For the definition and characterization of $O_\pi(G)$ see Problem~\ref{problem-1.D.7}. Take a $\pi$-subgroup $H\snormal G$ and let's prove, by induction on $|G:H|$, that $H\subseteq O_\pi(G)$. We may assume that $H\nnormal G$. The case $|G:H|=1$ is empty because $H=G$, and there is nothing to prove.

Now suppose that $|G:H|>1$ and take a normal chain of length $r=\sd_G(H)$
$$
    H = H_0\normal H_1\normal\cdots\normal H_{r-1}\normal H_r=G.
$$
Since $r\ge2$, the inductive hypothesis implies that $H\subseteq O_\pi(H_{r-1})$. But $O_\pi(H_{r-1})\ch H_{r-1}\normal G$. So, $O_\pi(H_{r-1})\normal G$. Since it is also a $\pi$-group, it follows that $O_\pi(H_{r-1})\subseteq O_\pi(G)$, showing that $H\subseteq O_\pi(G)$, as desired.

The last sentence is now trivial because, given two subnormal $\pi$-subgroups $H$ and $K$ of $G$, since $O_\pi(G)$ contains them both, it also contains their join. In other words, the subnormal $\pi$-subgroups of $G$ form a lattice.  \end{solution}

\begin{probl}\label{problem-2.A.2}
    Again let $\pi$ be a set of prime numbers, and recall that $O^\pi(G)$ is the unique smallest normal subgroup of\/ $G$ whose quotient group is a $\pi$-group. If\/ $H\snormal G$ and $\spec|G:H|\subseteq\pi$, show that $H\supseteq O^\pi(G)$.
\end{probl}

\begin{solution} Let's start by establishing a general

\textbf{Lemma.} $N\subgroup G \Rightarrow O^\pi(N)\subseteq O^\pi(G)$ and $N\normal G \Rightarrow O^\pi(N)\normal G$.

\textit{Proof.} First observe that the inclusion $O^\pi(N)\subseteq O^\pi(G)$ is a direct consequence of Problem~\ref{problem-1.B.8} b). Second, $O^\pi(N)\ch N\normal G$.

\if{false}
    \textbf{Lemma 2.} Let $1\to K\to G_1\to G_2\to 1$ be a short exact sequence (s.e.s.). If $K$ and $G_2$ are $\pi$-groups, then $G_1$ is a $\pi$-group.
    
    \textit{Proof of\/ \rm Lemma 2.} $|G_1|=|K||G_2|\implies \spec|G_1|\subseteq\spec|K|\cup\spec|G_1|\subseteq\pi$.
\fi

\medskip

If $H\normal G$ there is nothing to prove, so we may assume that there is a normal chain
$$
    H = H_0\normal H_1\normal\cdots\normal H_{r-1}\normal H_r=G
$$
with $r\ge2$.

According to the lemma, $O^\pi(H_{i-1})\normal H_i$ for all $i=1,\dots,r$. 

The hypothesis of the problem implies that $H\supseteq O^\pi(H_1)$ because $H\normal H_1$ and $H_1/H$ is a $\pi$-group.

Let $j$ be the maximum index such that $H\supseteq O^\pi(H_1)\supseteq\cdots\supseteq O^\pi(H_j)$. We claim that $j=r$. Suppose otherwise, then $1\le j<r$. Consider the following
$$
    |H_{j+1}/O^\pi(H_j)|=|H_{j+1}/H_j||H_j/O^\pi(H_j)|.
$$
Since
$$
    |H_{j+1}/H_j| = |H_{j+1}:H|/|H_j:H|,
$$
we deduce that
$$
    \spec|H_{j+1}/H_j| \subseteq\spec|H_{j+1}:H|
        \subseteq\spec|G:H|\subseteq\pi.
$$
It follows that $H_{j+1}/O^\pi(H_j)$ is a $\pi$-group and so $O^\pi(H_j)\supseteq O^\pi(H_{j+1})$. Then $j$ wasn't the maximum and we are done.  \end{solution}


\begin{probl}\label{problem-2.A.3}
    Let\/ $H$, $K$ be subgroups of $G$ and suppose that $|G:H|\perp|K|$.
    \begin{enumerate}[\rm a)]
    \item If $H \snormal G$, then $K \subseteq H$.
    \item If $K \snormal G$, then $K \subseteq H$.
    \end{enumerate}
\end{probl}

\begin{solution} 

\begin{enumerate}[\rm a)]
    \item Take a normal chain of minimal length
    $$
        H=H_0\normal H_1\normal\cdots\normal H_{r-1}\normal H_r=G.
    $$
    If $r=1$, then $H\normal G$ and we have
    $$
        |G:HK| = \frac{|G:H||K\cap H|}{|K|},
    $$
    which, by hypothesis, implies that $|K|\mid|K\cap H|$, i.e., $K\cap H=K$.

    If $r\ge 2$, let $j$ be the minimum $i$ such that $K\subseteq H_i$. Clearly $j\le r$. Suppose $j>0$. Since $H_{j-1}\snormal G$ and $|G:H_{j-1}|$ divides $|G:H|$, we get $|G:H_{j-1}|\perp|K|$. By induction on $|G:H|$ this implies $K\subseteq H_{j-1}$, which contradicts the definition of $j$. Then $j=0$ and we are done.

    \item Since $|K|$ divides $|G|=|G : H||H|$ and $|K|\perp|G:H|$, we get $|K|\mid|H|$.

    Take a normal chain of minimal length
    $$
        K=K_0\normal K_1\normal\cdots\normal K_{r-1}\normal K_r=G.
    $$
    If $r=1$, then $K\normal G$ and we have
    $$
        |G:HK| = \frac{|G:H||K\cap H|}{|K|},
    $$
    which, by hypothesis implies that $|K|\mid|K\cap H|$, i.e., $K\cap H=K$. Suppose that $r\ge2$ and let $j$ be the minimum $i$ such that the quotient
    $$
        q_i=\frac{|G|}{|HK_i|}
    $$
    is an integer number. Since the condition holds for $i=r-1$ (and $i=r$), we know that $j<r$. If $j=0$ we are done, so let's suppose $j>0$. Given that $K_{j-1}\normal K_j$, and consequently $H\cap K_{j-1}\normal H\cap K_j$, the injection
    $$
        1\to H\cap K_j/H\cap K_{j-1}\to K_j/K_{j-1} 
    $$
    shows that the quotient
    $$
        m=\frac{|HK_j|}{|HK_{j-1}|}=\frac{|K_j:K_{j-1}|}{|H\cap K_j:H\cap K_{j-1}|}
    $$
    is an integer. But,
    $$
        q_{j-1}=\frac{|G|}{|HK_{j-1}|} = \frac{|G|}{|HK_j|}\frac{|HK_j|}{|HK_{j-1}|}
            =q_jm.
    $$
    Thus, $q_{j-1}=q_jm\in\Z$, which is a contradiction.
\end{enumerate}
\end{solution}

\begin{probl}
    Let\/ $K \subseteq G$, where\/ $G$ is finite and\/ $K$ simple, and suppose that\/ $KH=HK$ for all subnormal subgroups\/ $H$ of\/ $G$. Show that\/ $K \subseteq N_G(H)$ for all\/ $H \snormal G$.

    \textrm{\rm\textbf{Note.} The intersection of the normalizers of all subnormal subgroups of\/~$G$ is the \textsl{Wielandt subgroup\/} of $G$, denoted by $W(G)$.}
\end{probl}

\begin{solution} Take $H\snormal G$. Fix a normal chain of minimal length
$$
    H=H_0\normal H_1\normal\cdots\normal H_{r-1}\normal H_r=G.
$$
If $r=1$, then $H\normal G$ and there is nothing to prove because $N_G(H)=G$.

Let's consider the case $r\ge2$. By induction on $r$ we can write $K\subseteq N_G(H_1)$ because $H_1\snormal G$.

In consequence, $K\cap H_1\normal K$. Since $K$ is simple, there are two cases, $K\cap H_1=K$ and $K\cap H_1=\gen1$. The former implies $K\subseteq H_1\subseteq N_G(H)$ and we are done. Let's analyze the latter.

Take $a\in H$ and $x\in K$ and let's show that $a^x\in H$. Since $a^x \in KHK=HK$, there exist $b\in H$ and $y\in K$ such that $a^x=by$. Then $y=b^{-1}a^x\in K\cap H_1$ because $a^x\in H_1$. But $K\cap H_1=\gen1$ and so $a^x=b\in H$.  \end{solution}

\begin{probl}\label{problem-2.A.5}
    Let $\cal X$ be any collection of minimal normal subgroups of\/~$G$, and let\/ $N = \prod\cal X$.
    \begin{enumerate}[\rm a)]
    \item Show that\/ $N$ is the direct product of some of the members of\/ $\cal X$.
    \item Show that every minimal normal subgroup of\/ $N$ is simple.
    \item Show that\/ $N$ is a direct product of simple groups.
    \end{enumerate}
    \textrm{\rm Hint. For b), show that $\Soc(N) = N$.}

    \textrm{\rm\textbf{Note.} This problem shows that minimal normal subgroups and socles of finite groups are direct products of simple groups.}
\end{probl}


\begin{solution} However not specifically stated, we will assume that $G$ is finite.

\begin{enumerate}[\rm a)]
    \item After enumerating the elements of $\cal X$, we can select a subset $\cal Y$ of them inductively as follows: (1)~The first element of $\cal X$ belongs to $\cal Y$ and (2)~If $N_{i_1},\dots,N_{i_k}$ belong to $\cal Y$ and $i_k<|{\cal X}|$, define $i_{k+1}$ to be $j$, where $j$ is the first index after $i_k$ such that the $j$th element of $\cal X$, say $M$, satisfies $M\cap\prod{\cal Y}=\gen1$. Since such an intersection is normal and is included in~$M$, the only other possibility is to equal $M$, which means that $M$ can be discarded. If such a $j$ does exist, put $i_{k+1}=j$ and $N_j=M$. Otherwise, end the process. The conclusion follows from Theorem~\ref{direct-product}.

    \item Following the hint let's first observe that $\Soc(N)\subseteq N$ is trivial. To prove the hint, let's consider the following

    \textbf{Lemma.} Let $\normal_m$ denote minimal normality. 
    \begin{enumerate}[\rm i)]
        \item $M\normal_mG\implies\Soc(M)=M$.
        \item $M\times L\normal G\times H\iff M\normal G$ and $L\normal H$.
        \item $M\normal_mG\iff M\times\gen1\normal_mG\times H$.
        \item $\Soc(G\times H)\supseteq\Soc(G)\times\Soc(H)$.
        \item $\Soc(N)=N$.
    \end{enumerate}

    \begin{proof}${}$ 
    \begin{enumerate}[\rm i)]
        \item $\Soc(M)\ch M\normal_mG\implies\Soc(M)\normal G\implies\Soc(M)=M$.
        \item $(a,b)^{(x,y)} = (x,y)(a,b)(x,y)^{-1} = (x,y)(a,b)(x^{-1},y^{-1})=(a^x,b^y)$.
        \item It follows from ii) and the definitions.
        \item This is a direct consequence of iii).

        \item To prove that $N\subseteq\Soc(N)$ we proceed by induction on the number of elements of $\cal Y$ (notation defined in part~a). The case $|{\cal Y}|=1$ is covered by part i). Assume that $|{\cal Y}|>1$ and pick $M\in\cal Y$ to define ${\cal Y}'={\cal Y}\setminus\set M$. According to the inductive hypothesis and the lemma
        $$
            \Soc\Big(\prod{\cal Y}'\Big)\Soc(M)=\Big(\prod{\cal Y'}\Big)M.
        $$
        Since the RHS is a direct product because $\prod{\cal Y}'\cap M=\gen1$, the same holds for the LHS. Moreover, the RHS equals $\prod\cal Y$, which is nothing but $N$. Using part iv), we get
        $$
            \Soc(N)
                =\Soc\Big(\prod{\cal Y}'\times M\Big)
                \supseteq\Soc\Big(\prod{\cal Y}'\Big)\times\Soc(M) = N.
        $$
    \end{enumerate}
    \end{proof}

    Now that we have established the hint, let's consider our problem, namely $M\normal_m N\implies M$ simple. Since $N=\Soc(N)$, we may assume that $N=MH$, where $H$ is the product of other minimal normal subgroups of~$N$, all having trivial intersection with $M$. Given that the product is direct, we have $M\leftrightarrow H$. Thus, if $\gen1\ne L\normal M$, we have $L\normal MH=N$ because $L\leftrightarrow H$ too. The minimality of $M$ in $N$ can be invoked to conclude that $L=M$. Thus $M$ has no proper normal subgroups, i.e., is simple.
    
    \item This is a direct consequence of part b). Indeed, $N=\Soc(N)$ is a direct product of minimal normal groups, all of which are simple.
\end{enumerate}
\end{solution}

\begin{probl}\label{problem-2.A.6}
    In this situation of the previous problem, show that every nonabelian normal subgroup of\/ $G$ contained in\/ $N$ contains a member of\/ $\cal X$.
\end{probl}

\begin{solution} Let $H\normal G$ be a nonabelian subgroup of $N$. Take $M\in\cal X$. Then, $M\cap H\normal G$ and $M\cap H\subseteq M$. Since $M\normal_mG$, we have $M\subseteq H$ or $M\cap H=\gen1$. In the latter case, $M\subseteq C_G(H)$ [cf.~Problem~\ref{problem-1.F.1}]. Thus, if $H$ doesn't contain any member of $\cal X$, $N=\prod{\cal X}\subseteq C_G(H)$. Therefore, $H\subseteq N\subseteq C_G(H)$, which contradicts the fact that $H$ is nonabelian.  \end{solution}


\begin{probl}
    Let\/ $H\snormal G$, where\/ $H$ is nonabelian and simple. Show that\/ $H^G$ is a minimal normal subgroup of\/ $G$.

    \textrm{\rm\textbf{Hint.} Work by induction on $|G|$ to conclude that $H\subseteq \Soc(K)$ whenever $H\subseteq K$. Deduce that each conjugate of $H$ in $G$ is a minimal normal subgroup of $H^G$. Then apply the previous problem to the group $H^G$, where $\cal X$ is the set of all $G$-conjugates of $H$.}

    \textrm{\rm\textbf{See also:} \href{https://math.stackexchange.com/a/2368182/269050}{This MSE post}}.
\end{probl}

\begin{solution} Let's say that $L\subgroup G$ is \textsl{good} whenever $L\subseteq K\implies L\subseteq\Soc(K)$. 

Following the hint let's show that every nonabelian simple subnormal subgroup of $G$ is good. Let's proceed by induction on $|G|$. The case $|G|=1$ is trivial because $H=G=\Soc(G)$.

When $|G|>1$ let $H\subgroup K\subgroup G$. If $|K|<|G|$ the induction hypothesis implies that $H$ is good. Therefore, we may assume that $K=G$. Take a strictly increasing chain
$$
    H = H_0\normal H_1\normal\cdots\normal H_{r-1}\normal H_r=G.
$$
First note that $H\normal_mH_1$. To see this take $M\normal_mH_1$ such that $M\subseteq H$. Then $M\normal H$ and, using that $H$ is simple, $H=M$. Therefore, $H\normal\Soc(H_1)$. In particular, $H$ is good if $r=1$ and so we may assume that $r>1$.

Write $K=\Soc(H_{r-1})$. By induction, $H\subseteq K$. Suppose that $N\normal_m K$ is such that $H\not\subseteq N$. Since $H\cap N\normal H$, the simplicity of $H$ implies $H\cap N=\gen1$. By Theorem~\ref{minimal-normal-normalizes-subnormal} we know that $N\subseteq N_K(H)$. Take $x\in N$ and $y\in H$. Then $xyx^{-1}\in H$ and so $[x,y]=(xyx^{-1})y^{-1}\in H$. Moreover, given that $N\normal K$, we also get $[x,y]=x(yx^{-1}y^{-1})\in N$. Then $[x,y]\in H\cap N=\gen1$, i.e., $H\leftrightarrow N$. It follows that $H\subseteq N$ for at least one $N\normal_mK$, otherwise $H\leftrightarrow K$, which doesn't because $H$ is nonabelian. Then $H\normal N$, and since $N$ is simple by Problem~\ref{problem-2.A.5}, we get $H=N\normal_mK$.

It follows that $H^x\normal_mK^x=K$. Therefore, $H^G=\prod H^x\subseteq K$ is a product of minimal normal subgroups of $K$. Since $H^G\normal G$, we can pick $N\normal_mG$ such that $N\subseteq H^G$.

We claim that $N$ is nonabelian. Suppose it is not, i.e., suppose that $N$ is abelian. Since $H$ is simple and nonabelian, $N\cap H=\gen1$. Take $z\in N$ and $y\in H$,
$$
    [z,y]=z\big(yz^{-1}y^{-1}\big)\in N\quad\textrm{and}\quad
        [z,y]=\big(zyz^{-1}\big)y^{-1}\in H
$$
because $H\normal K$ and $N\subseteq H^G\subseteq K$. Thus, $[z,y]\in N\cap H=\gen1$. It follows that $N\leftrightarrow H$. By normality $N\leftrightarrow H^x$ for $x\in G$. Therefore, $N\leftrightarrow H^G$, i.e., $N\subseteq Z(H^G)$. But $H^G$ is a direct product of simple nonabelian groups which, for this very same reason, have trivial centers. Thus $Z(H^G)=\gen1$ [cf.~Proposition~\ref{product-center-and-maximal}], a contradiction.

Now we can use the previous problem to conclude that $N$ must include some conjugate $H^x$. Given that $N$ is normal, we deduce that $H\subseteq N\subseteq\Soc(G)$. The proof by induction is now complete.

\medskip

Note also that, in the course of the proof, we showed that
$$
    N\normal_mG,\; N\subseteq H^G\implies H\subseteq N.
$$
By normality, $H^x\subseteq N$ for $x\in G$, i.e., $H^G\subseteq N\subseteq H^G$.  \end{solution}

\begin{probl}
    Let\/ $H$ and\/ $K$ be different nonabelian subnormal simple subgroups of\/ $G$. Show that\/ $H$ and\/ $K$ commute elementwise.
\end{probl}

\begin{solution} According to the previous problem, $H^G$ is minimal normal. From Theorem~\ref{minimal-normal-normalizes-subnormal} we know that $H^G\subseteq N_G(K)$. Therefore,
$$
    H\subseteq H^G\subseteq N_G(K)
$$
and so $H\cap K\normal H$. It follows that $H\cap K=\gen1$ or $H\subseteq K$. By symmetry, the latter would imply $H=K$, which isn't. Take $y\in H$ and $z\in K$, then
$$
    [y,z]=yzy^{-1}z^{-1}=z^yz^{-1}\in K.
$$
Symmetrically, $[y,z]\in H$. Therefore, $[y,z]\in H\cap K=\gen1$, i.e., $y\leftrightarrow z$.  \end{solution}


\begin{probl}
    Let\/ $H\snormal G$ and assume that\/ $H = O^\pi(H)$, where\/ $\pi$ is a set of primes. Show that\/ $O_\pi(G)$ normalizes\/ $H$.

    \textrm{\rm\textbf{Hint.} It is no loss to assume that $G = HO_\pi(G)$. Show in this case that $H = O^\pi(G)$.}
\end{probl}

\begin{solution} Since the definition of $O^\pi(H)$ only depends on $H$, and not on $G$, we can replace $H$ with any subgroup $G_*$ of $G$ containing it without compromising the fact that $H=O^\pi(H)$ or $H\snormal G_*$. Put $G_*=HO_\pi(G)$. Then $O_\pi(G)\normal G_*$. Since it is also a $\pi$-subgroup of $G_*$, we have $O_\pi(G)\subseteq O_\pi(G_*)$. Thus, if we prove that $O_\pi(G_*)\subseteq N_{G_*}(H)$, we would get $O_\pi(G)\subseteq N_{G_*}(H)\subseteq N_G(H)$. Therefore, there is no loss of generality in assuming that $G=HO_\pi(G)$. Moreover, in this case, $O_\pi(G)\subseteq N_G(H)\iff H\normal G$, which would follow if we show that $H=O^\pi(G)$ because that would imply $H\normal G$.

From
$$
    |G| = \frac{|O_\pi(G)||H|}{|H\cap O_\pi(G)|}
$$
we deduce that $|G:H|\mid|O_\pi(G)|$. Therefore, $\spec|G:H|\subseteq\pi$ and, according to Problem~\ref{problem-2.A.2}, $H\supseteq O^\pi(G)$. On the other hand, the lemma included in that problem implies that $H=O^\pi(H)\subseteq O^\pi(G)$. It follows that $H=O^\pi(G)$. In particular, $H$ is normal in $G$ and the conclusion follows.  \end{solution}

\begin{probl}
    We say that subgroups\ $H, K\subseteq G$ are \textsl{strongly conjugate} if they are conjugate in the group\/ $\gen{H,K}$. Show that\/ $H\snormal G$ if, and only if, the only subgroup of\/ $G$ that is strongly conjugate to\/ $H$ is\/ $H$ itself.
\end{probl}

\newcommand{\sconj}{\op{\hat\approx}}
\begin{solution} Let's write $H\sconj K$ to denote that $H$ and $K$ are strongly conjugate. Note that $\sconj$ is a reflexive and symmetric relation. Along this solution we will say that $H$ is \textsl{good} when $H\sconj K\implies H=K$.

\begin{description}
    \item[\rm\textit{if\/} part:] Assume that $H$ is good and let's work by induction on $|G|$ to prove that $H$ is subnormal. Suppose toward a contradiction that $H$ isn't subnormal in $G$ and let's apply the Zipper Lemma [cf.~Theorem~\ref{zipper-lemma}]. Let $M$ be the only maximal subgroup including $H$. Since $H$ isn't normal, $N_G(H)\subseteq M$.
    
    If $HH^x=H^xH$ for all $x\in G$, then $H\snormal G$ by Theorem~\ref{conjugate-commutativity}. Take $\omega\in G$ satisfying $HH^\omega\ne H^\omega H$. Since $H\ne H^\omega$, we must have $\omega\notin\gen{H,H^\omega}$, otherwise it would be $H\sconj H^\omega$. In particular, $J=\gen{H,H^\omega}$ is proper. It follows that $J\subseteq M$. Since $\gen{H,H^x}\subseteq M$ when such a group includes $x$, we deduce that $H^x\subseteq M$ for all $x\in G$. Therefore, $H^G\subseteq M$ and the inductive hypothesis implies $H\snormal H^G$. This concludes the proof because $H^G\normal G$.
    
    
    \item[\rm\textit{only if\/}:] Take a strictly increasing normal series
    $$
        H = H_0\normal H_1\normal\cdots\normal H_{r-1}\normal G
    $$
    and proceed by induction on $r$. The case $r\le1$ is trivial because in that case $H\normal G$ and every normal subgroup is clearly good. Let's now consider the case $r>1$. By the inductive hypothesis, $H$ is good in $H_{r-1}$. To see that $H$ is good in $G$, suppose that $K\subgroup G$ satisfies $H\sconj K$. This means that there exists $x\in\gen{H,K}$ such that $K=H^x$. In particular, $K\subseteq H_{r-1}^x$. Since $H_{r-1}\normal G$, it follows that $K\subseteq H_{r-1}$. The goodness of $H$ in $H_{r-1}$ implies that $H=K$, which proves that $H$ is good in $G$.
\end{description}
\end{solution}

\section{Baer Theorem}

\begin{thm}\label{baer-thm}{\rm[Baer]}
    Let\/ $H$ be a subgroup of a finite group\/ $G$. Then $H$ is included in the fitting group $F(G)$ if, and only if, $\gen{H,H^x}$ is nilpotent for all\/~$x\in G$.
\end{thm}

\begin{proof}${}$

\begin{description}
    \item[\rm{\it if\/} part:] If $\gen{H,H^x}$ is nilpotent, then $H$ is nilpotent [cf.~Corollary~\ref{nilpotent-subgroups-and-quotients}]. Therefore, according to Theorem~\ref{nilpotent-and-subnormal} to prove that $H\subgroup F(G)$ is is enough to show that $H\snormal G$. We will proceed by induction on $|G|$. Assume $|G|>1$ and suppose that $H$ is not subnormal in $G$. The induction hypothesis implies that $H\snormal K$ for all $H\subseteq K\varsubsetneq G$. In consequence, $H^x\snormal K$ for all $x\in K$ and so $\gen{H,H^x}\snormal K$ too. Therefore, we can apply the Zipper lemma Theorem~\ref{zipper-lemma} to conclude that $H$ is included in an only maximal group $M$.

    If $\gen{H,H^x}=G$ for some $x$, then $G$ would nilpotent and Lemma~\ref{nilpotent-lemma} would imply that $H\snormal G$, which is not. It follows that $\gen{H,H^x}\subseteq M$ for $x\in G$. Then $H^G\subseteq M$. In particular, $H^G\subseteq M$ is proper and the induction hypothesis implies that $H\snormal H^G\normal G$.

    \item[\rm{\it only if\/}:] If $H\subseteq F(G)$, then $H^x\subseteq F(G)^x=F(G)$. Then $\gen{H,H^x}\subseteq F(G)$ is nilpotent because $F(G)$ is [cf.~Corollaries~\ref{normal-nilpotent-fitting} \& \ref{nilpotent-subgroups-and-quotients}].
\end{description}
\end{proof}


\begin{defns}${}$\label{def:dihedral-group}

    An \textsl{involution} in a group\/ $G$ is an element of order\/ $2$.

    We say that the involution $t$ \textsl{inverts} an element $x\in G$, or that $x$ \textsl{is inverted by} $t$, when $x^t=x^{-1}$.

    A group\/ $D$ is \textsl{dihedral} if it contains a nontrivial cyclic subgroup\/ $C$ of index\/~$2$ such that every element of\/ $D \setminus C$ is an involution.
\end{defns}

\textbf{Note.} In the definition of a dihedral group, the elements of $C$ play the role of rotations, while involutions can be seen as reflections.

\begin{prop}
    In a group of even order the number of involutions is odd.
\end{prop}

\begin{proof} Let $I$ denote the set of involutions in the finite group $G$ of even order. Then, every element $x$ in $G\setminus I$ satisfies $x=1$ or $x\ne x^{-1}$ and we can partition $G\setminus I$ in sets of the form $\set{x,x^{-1}}$, all of size~$2$ except for the one where $x=1$. Thus, $|G\setminus I|$ is odd, which implies that $|\,I\,|$ is odd too.  \end{proof}

\begin{cor}
    In a group of even order there is always an involution whose conjugacy class has odd order.
\end{cor}

\begin{proof} Since the union of all conjugacy classes of involutions equals the set of all involutions, which has odd size, there must be at least one conjugacy class with an odd number of elements.  \end{proof}

\begin{rem}
    If\/ $t$ is an involution in a group\/ $G$, then\/ $t$ inverts\/ $x\in G$ iff\/ $(tx)^2=1$ because\/ $(tx)^2=x^tx$.
\end{rem}

\begin{rem}\label{rotation-involution}
    If\/ $D$ is a dihedral group with rotation subgroup\/ $C=\gen c$ of even order\/~$n$, then\/ $v=c^{n/2}$ is \textsl{the} rotation of order\/~$2$ in\/~$D$, i.e., the only involution in\/~$C$. In particular, if\/ $t\in D\setminus C$, then\/ $v\leftrightarrow t$ because\/ $v=v^{-1}=v^t$. In the case\/ $n=2$ this means that\/ $D$ is abelian, and in fact isomorphic to\/ $\Z_2\oplus\Z_2$.
\end{rem}

\begin{rem}\label{dihedral-elements}
    Let\/ $D$ be a dihedral group with\/ $2n$ elements and rotation subgroup\/ $C=\gen c$. Then for every\/ $t\in D\setminus C$, equality\/ $D=C\cup tC$ translates into
    $$
        D = C\cup\set{tc^i\mid 0\le i< n}.
    $$
\end{rem}

\needspace{2\baselineskip}
\begin{lem}\label{dihedral-lemma}
    Let $D$ be a group.
    \begin{enumerate}[\rm a)]
        \item Suppose that\/ $C = \gen c$ is a nontrivial cyclic subgroup of index\/ $2$ in\/ $D$, and\/ $t \in D\setminus C$ is an involution. Then every element of\/ $D\setminus C$ is an involution if, and only if, $t$ inverts $c$. In this case, $D$ is dihedral, and every $s\in D\setminus C$ inverts $x$ for all $x \in C$. Also, $D$ is generated by the distinct involutions $tc$ and~$t$.
        
        \item Suppose that\/ $D$ is generated by distinct involutions\/ $s$ and\/ $t$. Then the cyclic subgroup\/ $C = \gen{st}$ is nontrivial, and it fails to contain\/ $s$ and\/ $t$. Also, $|D:C| = 2$, and\/ $t$ inverts the generator\/ $st$ of\/ $C$. In particular, we are in the above situation, and\/ $D$ is dihedral.
    \end{enumerate}
\end{lem}

\begin{proof}${}$

\begin{enumerate}[\rm a)]
    \item Firstly note that $tC=D\setminus C$ because $D=C\cup tC$ with $tC\cap C=\emptyset$. Then, if every element of $D\setminus C$ is an involution, $t$ inverts $c$ because $(tc)^2=1$. Conversely, if $t$ inverts $c$, then every element $tc^n$ of $tC$ is also an involution because
    $$
        (tc^n)^2=tc^ntc^n=(c^t)^nc^n=(c^{-1})^nc^n=1.
    $$
    Therefore, under this condition, $D$ is dihedral by definition. Moreover, given $x\in C$ and $s\in D\setminus C$, it must be $sx\in D\setminus C$ and so, $x^sx=(sx)^2=1$, i.e., $s$ inverts $x$. The last property follows from the equation $D=C\cup tC$.

    \item Since $s\ne t$, we have $st\ne1$ and so $C=\gen{st}\ne\gen1$. Suppose $s=(st)^n$. Then $n>0$ and $s=st(st)^{n-1}$, hence $t=(st)^{n-1}$ and the same reasoning would imply $s=(st)^{n-2}$ etc.

    Since $D=C\cup sC$ with $C\cap sC=\emptyset$ as just shown, $|D:C|=2$. Then $t$ inverts $st$ because
    $$
        (st)^t=tstt=ts=(st)^{-1}.
    $$
    The last statement is a direct consequence of part a).
\end{enumerate}
\end{proof}

\begin{thm}\label{dihedral-odd-cycle}
    Let\/ $t$ be an involution in a finite group\/ $G$, and assume that\/ $t\notin O_2(G)$. Then there exists an element\/ $x\in G$ inverted by\/ $t$ such that $\ord(x)$ is an odd prime.
\end{thm}

\begin{proof} Since $\ord(t)=2$ and $O_2(G)$ is the only $2$-subgroup of $F(G)$, the hypothesis implies that $\gen t\not\subseteq F(G)$. By Baer's Theorem~\ref{baer-thm}, there exists $y\in G$ such that $\gen{t,t^y}$ is not nilpotent. In particular, $D=\gen{t,t^y}$ is not a $2$-group [Corollary~\ref{p-groups-are-nilpotent}] and $t\ne t^y$.

%As it's easy to see, $y\notin G$ and so $D\varsubsetneq G$. 
From the previous lemma we see that $D$ is dihedral of order $|D:C||C|=2|C|$, where $C=\gen{tt^y}$. Since $D$ is not a $2$-group, we can pick $x\in C$ of odd prime order. Using the lemma once again, we know that $t\in D\setminus C$ inverts $x$.  \end{proof}

\begin{rem}\label{rem:translated-group}
    If\/ $G$ is a group, given\/ $x\in G$ we can define the \textsl{translation} of\/ $G$ by\/ $x$ introducing the operation
    $$
        a\cdot_x b= ax^{-1}b,
    $$
    for\/ $a,b\in G$. With this operation\/ $G$ is also a group:
    \begin{enumerate}[-]
        \item \textit{associativity:} $a\cdot_x(b\cdot_x c)=ax^{-1}(bx^{-1}c)=(ax^{-1}b)x^{-1}c=(a\cdot_x b)\cdot_x c$.
        \item \textit{identity:} $a\cdot_x x= ax^{-1}x=a$.
        \item \textit{inverse:} $a\cdot_x(xa^{-1}x) = ax^{-1}xa^{-1}x=x$.
    \end{enumerate}
    With this notation, if\/ $D$ is a dihedral group generated by two involutions $t$ and~$s$, then both\/ $t$ and\/ $s$ are involutions in\/ $(D,\cdot_x)$ for all $x\in D$.
\end{rem}

\subsection{ChatGPT-3.5}

\begin{probl}
    Prove that the dihedral group of order $2n$ has $n$ reflections and $n$ rotations.
\end{probl}

\begin{solution} By definition, a dihedral group $D$ has a cyclic subgroup $C$ of index~$2$ and every element in $D\setminus C$ is an involution. Elements in $C$ are called rotations and involutions reflections. By definition of index the number of different left cosets of $C$ equals $2$. Thus, we can write $D=C\cup tC$ for a reflection $t$ with $C\cap tC=\emptyset$. Therefore, $|D|=2n$ with $n=|C|$. It particular $|D\setminus C|=n$ as well.  \end{solution} 

\begin{probl}
    Show that the center of\/ $D_{2n}$ is trivial when\/ $n$ is odd, and is generated by the rotation of order~$2$ when\/ $n>2$ is even and $D_{2n}=\Z_2\oplus\Z_2$ (additive notation) for\/ $n=2$.
\end{probl}

\begin{solution} Let $C=\gen c$ be the subgroup of rotations. According to the previous problem, $|C|=n$.

Take $z\in Z(D_{2n})$. Then $z\leftrightarrow t$ for every $t\in D_{2n}\setminus C$. If $z\in C$, then $z=z^t=z^{-1}$, which means that $z=1$ or $z$ is an involution. If $z\ne1$, then $2=\ord(z)\mid n$, i.e., $n$ is even. Then $Z(D_{2n})$ is trivial whenever $n$ is odd.

If $n$ is even, then $v=c^{n/2}\in C$ is \textsl{the\/} rotation of order~$2$, the only involution in~$C$. Since $v^t=v^{-1}=v$, for every $t\in D_{2n}\setminus C$, it follows that $v\leftrightarrow D_{2n}\setminus C$. Since $v\leftrightarrow c$, we deduce that $v\in Z(D_{2n})$. Take $z\in Z(D_{2n})$. If $z\in D_{2n}\setminus C$, then $c=c^z=c^{-1}$, i.e., $c^2=1$, which implies $c=v$, $n=2$ and $D_4=\Z_2\oplus\Z_2$ [cf.~Remark~\ref{rotation-involution}]. If $z\in C$, pick $t\in D_{2n}\setminus C$. Then $z=z^t=z^{-1}$, i.e., $z=1$ or $\ord(z)=2$. Since $v$ is the only involution in~$C$, it follows that $z\in\gen{v}$, i.e., $Z(D_{2n})=\gen v$.  \end{solution}

\begin{probl}
    Prove that $D_{2n}$ is nonabelian for $n \geq 3$.    
\end{probl}

\begin{solution} Suppose that $D_{2n}$ is abelian. According to the previous problem, $n$ must be even. In particular, $n\ge2$. If $n=2$, then the rotation $v$ of order $2$ generates the subgroup $C$ of rotations. Since it also generates the center, we arrive at a contradiction because $D_{2n}\setminus C$ is known to have $n=2$ elements.  \end{solution}

\begin{probl}
    Show that the order of the product of any two elements in $D_{2n}$ is at most $n$.
\end{probl}

\begin{solution} In fact, the order of any element is at most $n$. If the element is in $C$, its order divides $|C|=n$ and therefore it is at most $n$. Otherwise the element is an involution and its order is $2$, which is bounded by $n$ (there is no dihedral group with $2$ elements).  \end{solution}

\begin{probl}\label{chat-2.B.5}
    Let $C$ the subgroup of rotations of $D_{2n}$ and $T=D\setminus C$. Prove that
    $$
        CC=C,\quad CT=T,\quad TC=T,\quad TT=C.
    $$
\end{probl}

\begin{solution}
\begin{description}
    \item[\rm$CC=C$:] Trivial because $C$ is a group.
    \item[\rm$CT=T$:] This is a direct consequence of Remark~\ref{dihedral-elements}.
    \item[\rm$TC=T$:] The identity $tc^j=c^{n-j}t$ implies $CT=TC$.
    \item[\rm$TT=C$:] It follows from Remark~\ref{dihedral-elements} and the identity $tc^j=c^{n-j}t$.
\end{description}
\end{solution}

\begin{probl}\label{chat-2.B.6}
    Determine the conjugacy classes in $D_{2n}$ and their sizes.
\end{probl}

\begin{solution} Put $D=D_{2n}$ and let $C=\gen c$ be the subgroup of rotations. Firstly take $b\in C$ and $x\in D$. If $x\in C$, then $c\leftrightarrow x$ and so $c^x=c$. Otherwise, $b^x=b^{-1}$. It follows that $[\,b\,]=\set{b,b^{-1}}$. Therefore, we are left with the task of determine the conjugacy classes of all involutions.

Given that $(c^j)^t=c^{-j}$, we have $c^jt=tc^{-j}$. Then, 
$$
    (tc^i)^{c^j} = c^jtc^{i-j}=tc^{i-2j}.
$$
On the other hand,
$$
    (tc^i)^{tc^j} = c^{-j}ttc^ic^{-j}t = c^{i-2j}t=tc^{2j-i}.
$$
 The equations above show that
$$
    [tc^i]=\set{tc^{\pm(2j-i)}\mid j\in\Z}.
$$
We can specialize this last equation for $i=0$ to deduce that
\begin{equation}\label{eq3}
    [\,t\,]=\set{tc^{2j}\mid j\in\Z} = \set{tc^{2j}\mid 2j\in\Z_n},
\end{equation}
where the elements of the RHS set are pairwise different. So, the question reduces to counting the number of elements in this set. There are two cases: $n$ odd and $n$ even.
\begin{description}
    \item[{\it $n$ odd\/}:] In this case $C$ includes no involution. In particular, there are $n$ of them. And since $2\perp n$, it follows that $2$ is invertible in $\Z_n$, which means that
    \begin{align*}
        \Z&\to\Z_n\\
        j&\mapsto 2j
    \end{align*}
    is surjective. Since $tc^{2j}=tc^{2h}\iff 2j\equiv2h\pmod n$, we get
    $$
        |[\,t\,]|=|\set{2j\in\Z_n\mid j\in\Z}|=|\set{\Z_n}|=n.
    $$
    \item[{\it $n$ even\/}:] In this case we have the rotation of order~$2$ plus the involutions in $D\setminus C$, which amount to a total of $n+1$ involutions. Given that $2\mid n$, the number of elements in $\Z_n$ which are multiples of $2$ is $n/2$. Therefore, there are two conjugacy classes, namely $[\,t\,]$ and $[tc]$.
\end{description}
\end{solution}

\begin{probl}
    Show that $D_{2n}$ is isomorphic to the symmetry group of a regular $n$-gon.
\end{probl}

\begin{solution} See Spaces - \S\,Inner Product.\end{solution}

\begin{probl}
    Find the subgroups of $D_{2n}$ and determine their orders.
\end{probl}

\begin{solution} First we have $C$, the cyclic subgroup of rotations, which has order $n$. Then we have all its subgroups $\gen{c^{n/d}}$ for $d\mid n$, of order $d$ [cf.~Corollary~\ref{cyclic-subgroups}]. Note that all these subgroups are normal because there is only one subgroup of $C$ of any possible order. Then, we have the subgroups generated by involutions, which have order~$2$. Finally, according to Problem~\ref{chat-2.B.5}, we have the subgroups of the form $\gen{s,t}$ for $s\ne t\in D\setminus C$, which are dihedral groups of order $d=\ord(st)$, where $d\mid n$ [cf.~Lemma~\ref{dihedral-lemma}].  \end{solution}


\subsection{Problems B}

\begin{probl}
    Let\/ $H\subgroup G$ and assume that for each element\/ $x\in G$, either\/ $\gen{H, H^x}$ is nilpotent or\/ $HH^x=H^xH$. Show that\/ $H\snormal G$.
\end{probl}

\begin{solution} Suppose that $H$ is not subnormal in $G$. In particular, $H\ne G$ and, by Lemma~\ref{nilpotent-lemma}, $G$ is not nilpotent. Given $x\in G$, there are two cases: (1)~$\gen{H,H^x}$ is nilpotent and (2)~$HH^x$ is subgroup.

By induction on $|G|$ we may assume that $H\snormal K$ for every proper subgroup $K$ of~$G$. Therefore, we can apply Zipper Theorem~\ref{zipper-lemma} to conclude that $H\subseteq M$ for a single maximal subgroup $M\varsubsetneq G$. 

In case (1) it is $\gen{H,H^x}\ne G$ because the subgroup is nilpotent and $G$ isn't. In case (2) we have $\gen{H,H^x}=HH^x\ne G$ by Problem~\ref{problem-1.A.4}. Therefore, in both cases $\gen{H,H^x}\subseteq M$, which implies $H^G\subseteq M$. Then the induction hypothesis allows us to conclude that $H\snormal H^G$ and, of course, $H^G\normal G$.  \end{solution}

\begin{probl}\label{problem-2.B.2}
    Let\/ $D$ be the dihedral group of order\/ $2n$.
    \begin{enumerate}[\rm a)]
    \item If\/ $n$ is odd, show that\/ $D$ contains exactly\/ $n$ involutions, and that they all lie in a single conjugacy class.
    \item If\/ $n$ is even, show that\/ $D$ contains exactly\/ $n + 1$ involutions, and these lie in exactly three conjugacy classes, with sizes\/ $1$, $n/2$, and\/ $n/2$, respectively.
    \end{enumerate}
\end{probl}

\begin{solution} This is included in Problem~\ref{chat-2.B.6}.
\end{solution}

\begin{probl}
    Let\/ $s$ and\/ $t$ be involutions in a group\/ $G$. If\/ $s$ and\/ $t$ are not conjugate in\/ $G$, show that there exists an involution\/ $z\in G$, different from\/ $s$ and\/ $t$, and commuting with both of them.
\end{probl}

\begin{solution} Consider the dihedral group $D=\gen{s,t}$ [cf.~Lemma~\ref{dihedral-lemma}~b)]. Put $|D|=2n$. By Problem~\ref{problem-2.B.2} we know that $n$ must be even and that there are $n+1$ involutions in three conjugacy classes. From the previous problem we also know that one of the conjugacy classes has $1$ element and the other two $n/2$. Let $v$ be the involution satisfying $[v]=\set v$. Then $v\in Z(D)$.

If $v\in\set{s,t}$, then $\gen{s,t}$ is abelian with $\ord(st)=2$, and so $\gen{s,t}=\Z_2\oplus\Z_2$ (additive notation). The conclusion becomes evident.

In the case where $v\notin\set{s,t}$ we simply have $v\leftrightarrow s$ and $v\leftrightarrow t$.  \end{solution}

\begin{probl}
    Suppose that\/ $G$ has more than one Sylow\/ $2$-subgroup and that every two distinct Sylow\/ $2$-subgroups of\/ $G$ intersect trivially. Show that\/ $G$ contains exactly one conjugacy class of involutions.
\end{probl}

\begin{solution} Let's first consider the case where $G$ is dihedral. Since $C\normal G$, given $P,\;Q\in\Syl_2(G)$ we have $C\cap P,\; C\cap Q\in\Syl_2(C)$ (in any group the intersection of Sylow with normal is Sylow). But $C$ is abelian and therefore $C\cap P=C\cap Q$, which is trivial (if and) only if $|C|$ is odd.

In the general case take two different involutions $s$ and $t$. Let $P$ and $Q$ be the only Sylow $2$-subgroups of $G$ such that $s\in P$ and $t\in Q$. Since we can replace $s$ with $s^a$ for appropriate $a\in G$, we are allowed to assume that $P\ne Q$. Hence, $P\cap Q=\gen1$. 

Consider the dihedral group $D=\gen{s,t}$ of order $2n$ [cf.~Lemma~\ref{dihedral-lemma}]. The uniqueness of $P$ and $Q$ implies that $D\cap P$ and $D\cap Q$ are two different Sylow $2$-groups of $D$ with trivial intersection. As we saw above, this can only happen if $n$ is odd. By Problem~\ref{problem-2.B.2}, $s$ and $t$ are $D$- hence $G$-conjugates.  \end{solution}

\begin{probl}
    Let\/ $B\subgroup G$ with\/ $|G:B|=2$. Show that the following are equivalent:
    \begin{enumerate}[\rm i)]
    \item $G\setminus B$ contains an involution\/ $t$ such that\/ $b^t=b^{-1}$ for all elements $b\in B$.
    \item $G\setminus B$ consists entirely of involutions.
    \item Every element\/ $t$ of\/ $G\setminus B$ is an involution such that\/ $b^t=b^{-1}$ for all elements\/ $b\in B$.
    \end{enumerate}
    Show that if these conditions hold, then\/ $B$ must be abelian.

    \textrm{\rm{\bf Note.} In this situation, $G$ is said to be \textsl{generalized dihedral}.}
\end{probl}

\begin{solution}
\begin{enumerate}[\rm i)]
    \item $\Rightarrow$ ii) Take $s\in G\setminus B$. Since $t\notin B$, $st\notin sB$. Thus,
    $$
        ts=(st)^t=(st)^{-1}=ts^{-1}.
    $$

    \item $\Rightarrow$ iii) Take $b\in B$ and $t\in G\setminus B$. Since $tb\notin B$, it is an involution. Therefore,
    $$
        1 = (tb)^2 = tbtb = b^tb.
    $$

    \item $\Rightarrow$ i) Trivial because $|G:B|=2\implies G\setminus B\ne\emptyset$.
\end{enumerate}

\needspace{3\baselineskip}
To see the last statement take $x,y\in B$. Pick an involution $t$. Then
$$
    x^{-1}y^{-1}=x^ty^t = (xy)^t = (xy)^{-1}=y^{-1}x^{-1}.
$$
 \end{solution}

\begin{probl}
    Show that\/ $G$ has a normal Sylow\/ $p$-subgroup if, and only if, every subgroup of the form\/ $\gen{x,y}$, where\/ $x$ and\/ $y$ are conjugate elements of\/ $G$ having\/ $p$-power order, has a normal Sylow\/ $p$-subgroup.
\end{probl}

\begin{solution}
\begin{description}
    \item{\rm{\it if\/} part:} Let $P\in\Syl_p(G)$. Take $y\in P$ and $x\in G$ and let's see that $y^x\in P$. Put $H=\gen{y,y^x}$. By hypothesis, there is only one $p$-subgroup $Q$ of $H$. And since $\ord(y^x)=\ord(y)$ is a power of $p$, we deduce that $y^x\in Q$, i.e., $Q=H$. In particular, $H$ is nilpotent. Since $x$ was arbitrarily chosen, we can apply Baer's Theorem~\ref{baer-thm} to $\gen y$ and deduce that $\gen y\subseteq F(G)$. It follows that $y\in O_p(G)$, the only $p$-subgroup of $F(G)$. Since $O_p(G)$ is normal in $G$, $y^x\in O_p(G)\subseteq P$.
    
    \item{\rm{\it only if\/}:} Trivial: every subgroup of $G$ has a normal Sylow $p$-subgroup.
\end{description}
\end{solution}

\section{Local Subgroups}

\begin{defn}
    Let $p$ be a prime. A subgroup $H$ of a group $G$ is \textsl{$p$-local} if $H=N_G(P)$ for some nontrivial $p$-subgroup $P$. A subgroup $H$ is \textsl{local} if it is $p$-local for some prime $p$.
\end{defn}

\textbf{Digression.} Let\/ $G$ be a group of order\/ $60$. Assume that $G$ includes a subgroup\/ $H$ of index\/ $5$. Consider the set\/ $\lco GH$ of left-cosets of\/ $H$ and the action of\/ $G$ on it given by\/ $(x,yH)\mapsto xyH$. We can define $\sigma\colon G\to\Sym(\lco GH)$ as $\sigma_x(yH)=xyH$. Note that since $|G:H|=5$, $\Sym(\lco GH)\cong S_5$. If ---in addition--- $G$ is simple, $\sigma$ is mono. Thus, $\im(\sigma)\subseteq\Alt(\lco GH)$ because $\sigma^{-1}(\Alt(\lco GH))\normal G$ and $\im(\sigma)$ includes at least one $3$-cycle, namely, the image of any element of order $3$ in $G$. It follows that, after coastriction, we may assume that $\sigma\colon G\to\Alt(\lco GH)\cong A_5$ is a monomorphism, hence an isomorphism as both groups have order~$60$.

In conclusion, every simple group $G$ of order $60$ with a subgroup of index $5$ is isomorphic to $A_5$. As we will see next, we can remove the index $5$ hypothesis as any simple group of order $60$ has a subgroup with said index.

Put $n_2=n_2(G)$. Then $n_2\in\set{\cancel 3, 5, 15}$, where $3$ is ruled out because it would imply $|G:N_G(P)|=3$ for every $P\in\Syl_2(G)$, which is impossible by the $n!$~theorem~\ref{n!-theorem}. If $n_2=5$, we can take $H=N_2(P)$ for any $P\in\Syl_2(G)$. If $n_2=15$, we can invoke Theorem~\ref{n_p(G)=1} to conclude that
$$
    15\equiv1\pmod{|Q:Q\cap R|}
$$
for certain $Q,R\in\Syl_2(G)$. In particular, $Q\cap R=2$. Write $H=N_G(Q\cap R)$. By Problem~\ref{problem-1.A.1}, $Q\cap R\normal Q$ and so $Q\subseteq H$. For the very same reason $R\subseteq H$ and so $|H|=4m$ with $1<m<15$, $m\mid 3\cdot5$, i.e., $|H|\in\set{4\cdot3,\; 4\cdot5}$. Thus, $|G:H|\in\set{3,5}$. Again, the $n!$-theorem rules out $3$ and so $|G:H|=5$, as desired.  \qed

\begin{lem}\label{p-local-preimages}
    Let\/ $N\normal G$, and write\/ $\bar G = G/N$, using the standard ``bar convention'', where\/ $G \to\bar G$ is the canonical projection onto the quotient. Then for all primes\/ $p$, every\/ $p$-local subgroup of\/ $\bar G$ has the form $\bar L$, where $L$ is some $p$-local subgroup of $G$.
\end{lem}

\begin{proof} Put $|\bar G|=p^em$ with $e>0$ and $p\perp m$. A $p$-local subgroup $\tilde L$ of $\bar G$ has the form $\tilde L=N_{\bar G}(\bar Q)$ for some $N\subgroup Q\subgroup G$, with $|\bar Q|=p^e$. As it is easy to see, 
$$
    \widebar{N_G(Q)}=N_{\bar G}(\bar Q)=\tilde L.
$$
Therefore, we may put $L=N_G(Q)$ and get $\tilde L=\bar L$. Pick $P\in\Syl_p(Q)$ and let's show that $\bar L=\widebar{N_G(P)}$.

We claim that $Q=NP$. Indeed, $|Q:N|=p^e$ and $p\perp|Q:P|$, so the assertion is a direct consequence of Problem~\ref{problem-1.A.3}.

By the Frattini Argument~(\ref{frattini-argument}), since $Q\normal L$ and $P\in\Syl_p(Q)$, we have $L=QN_L(P)$. Then
$$
    L = QN_L(P)=NPN_L(P)=NN_L(P)=N(N_G(P)\cap L)=NN_G(P)\cap L,
$$
where the last equality follows from Proposition~\ref{dedekind}. Thus,
$L\subseteq NN_G(P)$ and so $\bar L\subseteq\widebar{N_G(P)}$.

On the other hand,
$$
    x\in N_G(P)\implies x\in N_G(NP)=N_G(Q)=L\implies \bar x\in\bar L,
$$
i.e., $\widebar{N_G(P)}\subseteq\bar L$. In conclusion, $\tilde L= \widebar{N_G(P)}$. Moreover, $P$ is nontrivial because $p^e\mid|Q|$ and so $p^e\mid|P|$.  \end{proof}

\begin{rem}
    If a group\/ $G$ has a normal Sylow\/ $2$-subgroup, then every subgroup of\/ $G$ also has a normal Sylow\/ $2$-subgroup.

    \textrm{\small\rm Indeed. Let $N\normal G$ be the only Sylow $2$-subgroup of $G$. Take $H\subgroup G$. If $2\perp|H|$, then $\gen1$ is a normal $2$-subgroup of $H$. Otherwise, if $P\in\Syl_2(H)$, $P=H\cap N\normal H$ [cf.~Problem~\ref{problem-1.C.2}].}

    For a stronger converse statement we have the following
\end{rem}

\begin{thm}
    Suppose that for every odd prime\/ $p$, every\/ $p$-local subgroup of a finite group\/ $G$ has a normal Sylow\/ $2$-subgroup. Then\/ $G$ has a normal Sylow\/ $2$-subgroup.
\end{thm}

\begin{proof} Let's first show that $|G|$ is odd by supposing, toward a contradiction, that it's even. Let's start by considering the case $O_2(G)=\gen1$. We can invoke Theorem~\ref{dihedral-odd-cycle} to obtain an element $x\in G$ inverted by $t$ and with $\ord(x)=p$, an odd prime. Then $D=\gen{t,tx}$ is dihedral with rotation group $C=\gen x$. Since $N_G(C)$ is $p$-local, the hypothesis allows us to establish that $N_G(C)$ has a normal Sylow $2$-group $N$. From the equation $x^t=x^{-1}$, we deduce that $t\in N_G(C)$. Therefore, $t\in N$. Because their orders are coprime, $C\leftrightarrow N$ [cf.~Problem~\ref{problem-1.F.1}]. In particular, $x\leftrightarrow t$, i.e., $x=x^t=x^{-1}$, a contradiction. Since the contradiction arouse from assuming that $|G|$ was even, we conclude that it isn't.

Now put $\bar G=G/O_2(G)$. Then $O_2(\bar G)=\gen1$. Note that $\bar G$ satisfies the hypothesis stated for $G$. Indeed; according to Lemma~\ref{p-local-preimages}, every $p$-local subgroup of $\bar G$ is the quotient of a $p$-local subgroup of $G$. Moreover, the normal Sylow $2$-group of such a $p$-local subgroup, when seen in the quotient, is a Sylow $2$-group of the same index, which is also normal. By the first part, $|\bar G|$ is odd. Then $O_2(G)$ is a Sylow $2$-subgroup of $G$, which is also normal.  \end{proof}

\begin{prop}\label{p-local-quotient}
    Let\/ $G$ be a finite group, and let\/ $N\normal G$. Write\/ $\bar G = G/N$. If\/ $P$ is a nontrivial\/ $p$-subgroup of\/ $G$ and $p\perp|N|$, then\/ $\bar P$ is nontrivial, and\/ $\widebar{N_G(P)} = N_{\bar G}(\bar P)$. In particular, if\/ $L$ is\/ $p$-local in\/ $G$, then\/ $\bar L$ is\/ $p$-local in\/ $\bar G$.
\end{prop}

\begin{proof} Given that $p\perp|N|$, $P\cap N=\gen1$ and so $|PN|=|P||N|$. As a first conclusion $\bar P=PN/N=P/P\cap N=P$ is nontrivial [cf.~Proposition~\ref{prod-quotient}]. The second is that $P\in\Syl_p(PN)$.

Now consider the local group $L=N_G(P)$. From
$$
    x\in L\iff P^x=P \implies \bar P^{\bar x}=\bar P
$$
we deduce that $\bar L\subseteq N_{\bar G}(\bar P)$.

To verify the other inclusion let $H$ be the (only) subgroup of $G$ that satisfies $N\subseteq H$ and $\bar H=N_{\bar G}(\bar P)$. Since $\bar P\normal\bar H$, we get $PN\normal H$ [cf.~Proposition~\ref{quotient-preserves-normal}]. We can now invoke Frattini's Argument Theorem~\ref{frattini-argument} and get $H=N_H(P)PN$. Thus,
$$
    H=N_H(P)PN=N_H(P)N\subseteq N_G(P)N=LN,
$$
which implies $N_{\bar G}(\bar P)=\bar H=\bar L$, as desired.  \end{proof}

\paragraph{Solvable Groups.} Here we introduce the notion of solvable group and establish some of their basic properties. We will make use of this concept in \S~\contour{black}{\textmd{\ref{schur-zassenhaus}~\nameref{schur-zassenhaus}}}.

\medskip

\begin{defn}\label{solvable-defn}
    A (not necessarily finite) group\/ $G$ is said to be \textsl{solvable} if there exist normal subgroups $N_i$ such that
    $$
        \gen1=N_0\subseteq N_1\subseteq N_2\subseteq\cdots\subseteq N_r=G
    $$
    where\/ $N_i/N_{i-1}$ is abelian for\/ $1\le i\le r$.

    In particular, all abelian groups are solvable.
\end{defn}


\begin{prop}\label{solvable-subgroups-and-quotients}
    Let\/ $G$ be a solvable group. Then every subgroup of\/ $G$ and every homomorphic image of\/ $G$ is solvable.
\end{prop}

\begin{proof} Let $(N_i)_{0\le i\le r}$ be like in the definition of solvable group. If $H$ is a subgroup of $G$, then $H\cap N_i\normal H$ and $H\cap N_i/H\cap N_{i-1}$ is abelian because
$$
    1\to H\cap N_i/H\cap N_{i-1} \to N_i/N_{i-1}
$$
is exact. Moreover, if $N\normal G$, then $N_iN\normal G$ and so $N_iN/N\normal G/N$. In addition,
$$
    (N_iN/N)/(N_{i-1}N/N)=N_iN/N_{i-1}N
$$
is abelian because the composition
$$
    N_i\to N_iN\to N_iN/N_{i-1}N
$$
induces an epimorphism
$$
    N_i/N_{i-1}\to N_iN/N_{i-1}N.
$$
 \end{proof}

\begin{prop}\label{solvable-ses}
    If\/ $N\normal G$ and\/ $G/N$ are solvable, then\/ $G$ is solvable.
\end{prop}

\begin{proof} The normal series
$$
    \gen1 = N_0\subseteq\cdots N_r=N\quad\text{and}\quad
    \gen1= N_{r+1}/N\subseteq\cdots N_{r+s}/N = G/N,
$$
where $N_i/N_{i-1}$ and $(N_{r+j}/N)/(N_{r+j-1}/N)$ are abelian, produce a normal series
$$
     \gen1 = N_0\subseteq\cdots N_r= N_{r+1}\subseteq\cdots N_{r+s}=G
$$
where all the quotients are abelian.  \end{proof}

\begin{ntn}
    Recall that given a group\/ $G$ its derived group $G'$ is the characteristic subgroup of\/ $G$ generated by the commutators\/ $[x,y]$ of elements\/ $x,y\in G$. If we iterate this construction we will get\/ $G''$, $G'''$, and more generally\/ $G^{(r)}$, for generality or for orders superior to, say,~$3$. The convention extends to\/ $r=0$ with\/ $G^{(0)}=G$.
\end{ntn}

\begin{prop}\label{solvable-equals-finite-derivatives}
    A group\/ $G$ is solvable if, and only if, there exists\/ $r$ such that\/ $G^{(r)}=\gen1$.
\end{prop}

\begin{proof} If $G$ is solvable, there exists a sequence of normal groups
$$
    \gen1=N_0\subseteq N_1\subseteq\cdots\subseteq N_r=G
$$
where $N_i/N_{i-1}$ is abelian for all $i$. Equivalently, $N_i'\subseteq N_{i-1}$ for $1\le i\le r$. And since $H\mapsto H'$ preserves inclusions, it follows that $G^{(r-j)}\subseteq N_j$ for $0\le j\le r$. In particular, $G^{(r)}\subseteq N_0=\gen1$.

Conversely, if $G^{(s)}=\gen1$ for some integer $s$, by decreasing indunction on $i$ the subgroups $N_i=G^{(s-i)}$ satisfy $N_{i-1}=N_i'\ch N_i\normal G$. Thus, they are all normal and form the desired sequence.  \end{proof}

\begin{cor}\label{solvable-equivalence}
    A group\/ $G$ is solvable if, and only if, there exists a sequence
    $$
        \gen 1=G_0\normal G_1\normal\cdots\normal G_r=G,
    $$
    where each of the subgroups is normal in the following as indicated, and\/ $G_i/G_{i-1}$ is abelian for\/ $1\le i\le r$.
\end{cor}

\begin{proof} The condition is clearly necessary. It is sufficient because it implies that $G_i'\subseteq G_{i-1}$ for all $i$, which recursively implies $G^{(r)}=\gen1$.  \end{proof}

\begin{cor}\label{nilpotent-implies-solvable}
    nilpotent $\implies$ solvable.
\end{cor}

\begin{proof} Since subgroups of nilpotent groups are nilpotent (\ref{nilpotent-subgroups-and-quotients}), it is enough to show that the derived group $G'$ of a nilpotent group $G$ is proper, and then apply Corollary~\ref{solvable-equivalence}.

Take a maximal subgroup $M\subgroup G$. By Theorem~\ref{nilpotent-equivalences}, $M$ is normal. It follows that the quotient $G/M$ has no proper subgroups and so it must be isomorphic to $\Z_p$ for some prime $p$. In particular, $G/M$ is abelian and so $G'\subseteq M$.  \end{proof}

\begin{cor}\label{cor:p-groups-are-solvable}
    Every\/ $p$-group is solvable.
\end{cor}

\begin{proof}
    Let $G$ be a $p$-group. By Corollary~\ref{p-groups-have-center}, $Z(G)\ne\gen1$. Then $Z(G)$ is solvable by the proposition and $G/Z(G)$ is solvable by induction on $|G|$. The result follows from Proposition~\ref{solvable-ses}.
\end{proof}

\begin{cor}\label{cor:solvable-iff-p-group-chain}
    A group\/ $G$ is solvable if, and only if, it has a chain of normal subgroups\/ $\gen1=N_0\subgroup N_1\subgroup\cdots\subgroup N_r=G$ such that\/ $N_i/N_{i-1}$ is a\/ $p$-group.
\end{cor}

\begin{proof}
    The \textit{if\/} part is a direct consequence of Corollary~\ref{cor:p-groups-are-solvable} and Proposition~\ref{solvable-ses}.

    For the \textit{only if\/} part, let's say that a group is \textsl{good\/} if it has a chain of normal groups whose consecutive quotients are $p$-groups. The argument used in the proof of Proposition~\ref{solvable-ses} can be used to show that if $H\normal G$ and $G/H$ are both good, then $G$ is good. Corollary~\ref{solvable-equivalence} and induction on the length of the sequence allow us to reduce ourselves to the case where both $H\normal G$ and $G/H$ are abelian. Since every (finite) abelian group is a product of $p$-groups [cf.~Theorem~\ref{product-of-all-Gp}], both $H$ and $G/H$ are good. Hence, $G$ is good.
\end{proof}

\begin{defn}
    A finite group\/ $G$ is an \textsl{$N$-group} when every local subgroup is solvable.
\end{defn}

\subsection{Problems C}

\begin{probl}${}$
    \begin{enumerate}[\rm a)]
    \item Every proper homomorphic image of an\/ $N$-group is solvable. Here ``proper'' means that the morphism has a nontrivial kernel.

    \item Let\/ $G$ be a nonsolvable\/ $N$-group. Then\/ $G$ has a unique minimal normal subgroup\/ $M$.
    \end{enumerate}
    \textrm{\rm\textbf{Hint.} The Frattini argument is relevant. Also, recall that subgroups and homomorphic images of solvable groups are solvable, and that if $K\normal L$ with $K$ and $L/K$ both solvable, then $L$ is solvable.}

    \textrm{\rm\textbf{Note.} A major step leading toward the classification of finite simple groups was J.~Thompson's classification of nonsolvable $N$-groups. A nonabelian finite simple group is minimal simple if every proper subgroup is solvable, and since minimal simple groups are clearly $N$-groups, Thompson's work provided a classification of all minimal simple groups.}
\end{probl}

\begin{solution} {[See also \href{https://math.stackexchange.com/q/3765771}{this MSE post}]}

\begin{enumerate}[\rm a)]
    \item Let $G$ be an $N$-group and $\varphi\colon G\to\bar G$ an epimorphism with $N=\ker(\varphi)$ nontrivial. Pick $p\mid|N|$ and $P\in\Syl_p(G)$. Since $P\cap N\in\Syl_p(N)$, the Frattini Argument~(\ref{frattini-argument}) implies that $G=N_G(P\cap N)N$. Put $L=N_G(P\cap N)$ and $\bar L=\varphi(L)$. The hypothesis implies that $L$ is solvable. Since $\bar G=\bar L$ the conclusion is a direct consequence of Proposition~\ref{solvable-subgroups-and-quotients}.

    \item Let $M\normal_m G$. By part~a) $\bar G=G/M$ is solvable. According to Proposition~\ref{solvable-equals-finite-derivatives}, there exists an integer $r$ such that $\bar G^{(r)}=\gen1$. But $\widebar{G^{(r)}}=\bar G^{(r)}$ and so $G^{(r)}\subseteq M$. Finally, the fact that $G^{(r)}\ne\gen1$ (same proposition) and is normal (as shown in the proof of said proposition) implies that $M=G^{(r)}$.
\end{enumerate}
\end{solution}

\section{Zenkov Theorem}

Brodkey Theorem~\ref{brodkey-thm} proves that in a group $G$ with abelian Sylow $p$-groups, there are two of them whose intersection is $O_p(G)$. A generalization of this result is the following

\begin{thm}\label{zenkov}
    {\rm[Zenkov]} Let\/ $A$ and\/ $B$ be abelian subgroups of a finite group\/ $G$, and let\/ $M$ be a minimal member of the set\/ $\set{A \cap B^x \mid x \in G}$. Then\/ $M\subseteq F(G)$.
\end{thm}

\begin{proof} After replacing $B$ with an appropriate conjugate of $B$, we may assume that the minimal member $M$ is $A\cap B$. The proof works by induction on $|G|$.

Suppose that $G=\gen{A,B^x}$ for some $x\in G$. Then $A\cap B^x\subseteq Z(G)$ and so
$$
    A\cap B^x=(A\cap B^x)^{x^{-1}}=A^{x^{-1}}\!\!\cap B\subseteq B.
$$
It follows that $M=A\cap B = A\cap B^x\subseteq Z(G)\subseteq F(G)$.

Let's now consider the case where $\gen{A,B^x}\varsubsetneq G$ for all $x\in G$. By Proposition~\ref{nilpotent-subgroups-and-quotients}, $M$ is nilpotent (as well as $A$ and $B$). Thus, we can invoke Theorem~\ref{nilpotent-equivalences} to conclude that $M$ is generated by its Sylow subgroups. Therefore, by Baer's Theorem~\ref{baer-thm}, to show that $M\subseteq F(G)$ it is enough to show that $\gen{P,P^x}$ is nilpotent for every prime $p$, every $P\in\Syl_p(M)$ and every $x\in G$.

Take $x\in G$ and put $H=\gen{A,B^x}$ and $C=B\cap H$. Given $y\in H$
$$
    A\cap C^y = A\cap B^y\cap H= A\cap B^y.
$$
It follows that $M=A\cap C$ is minimal in $\set{A\cap C^y\mid y\in H}$. Then, the inductive hypothesis applied to $H$ implies $M\subseteq F(H)$. In particular, if $P\in\Syl_p(M)$, we have $P\subseteq F(H)$. Since $O_p(H)$ is the only Sylow $p$-subgroup of $F(H)$, it follows that $P\subseteq O_p(H)$. Moreover, $P^x\subseteq B^x\subseteq H$, which implies that $O_p(H)P^x$ is a $p$-group (the product of a normal $p$-group with a $p$-group is a $p$-group). It follows that $\gen{P,P^x}\subseteq O_p(H)P^x$ is also a $p$-group, hence nilpotent as desired.  \end{proof}

\begin{cor}\label{abelian-intersects-fitting}
    Let\/ $A\subgroup G$, where\/ $A$ is abelian and\/ $G$ is a nontrivial finite group, and assume that\/ $|A|\ge|G:A|$. Then\/ $A\cap F(G)\ne\gen1$.
\end{cor}

\begin{proof} Given $x\in G$, $|AA^x|=|A|^2/|A\cap A^x|$. By Problem~\ref{problem-1.A.4} there are two possibilities, namely $A=G$ or $|AA^x|<|G|$. The former renders the conclusion trivial. The latter implies
$$
    |G| > |A|^2/|A\cap A^x|,
$$
i.e., $|A\cap A^x|>|A|^2/|G|\ge |A||G:A|/|G|=1$. Thus, the theorem applied to $A=B$ ensures that $\gen1\varsubsetneq A\cap A^y\subseteq F(G)$ for some $y\in G$, which implies $A\cap F(G)\ne\gen1$.  \end{proof}

\begin{rem}
    By\/ {\rm Theorem~\ref{nilpotent-and-subnormal}} if\/ $A$ is an abelian subgroup of a finite nontrivial group\/ $G$ satisfying\/ $|A|\ge|G:A|$, then\/ $A$ includes a subnormal subgroup of\/ $G$. If ---moreover--- $A$ is cyclic and proper, we obtain the following
\end{rem}

\begin{thm}\label{lucchini-thm}
    {\rm[Lucchini]} Let\/ $A$ be a cyclic proper subgroup of a finite group\/ $G$, and let\/ $K=\Core_G(A)$. Then\/ $|A:K|<|G:A|$, and in particular, if\/ $|A|\ge|G:A|$, then\/ $K\ne\gen1$.
\end{thm}

\begin{proof} The proof works by induction on $|G|$.

\begin{description}
    
    \item[\rm1.~\textit{Reduction to $K=\gen1$\/}:] First note that $A/K$ is a proper cyclic subgroup of $G/K$ and that $\Core_{G/K}(A/K)=\gen1$. If $K\ne\gen1$, we can apply the inductive hypothesis to $G/K$ and obtain $|A/K|<|G/K:A/K|=|G:A|$, as desired. Therefore, to complete the proof, it is enough to assume that $K=\gen1$ and show that $|A|<|G:A|$. In what follows we will suppose, toward a contradiction, that $|A|\ge|G:A|$.

    \item[\rm2.~\textit{$\exists\,M\normal_mG,\; M\subseteq Z(F(G))$\/}:] By Corollary~\ref{abelian-intersects-fitting}, we get $A\cap F(G)\ne\gen1$. In particular $F(G)\ne1$ and therefore we can pick $M\normal_m G$, $M\subseteq F(G)$. Then, by Proposition~\ref{nontrivial-fitting-center}, $M\cap Z(F(G))\ne\gen1$. Thus, $M\subseteq Z(F(G))$ because of the minimality of $M$. In particular, $M$ is abelian.

    \item[\rm3.~\textit{$M$ is elementary abelian\/}:] Pick $\gen1\ne P\in\Syl_p(M)$. Then $P\ch M\normal G$ and so $M=P$ (Step~2) is a $p$-group. By Problem~\ref{problem-1.D.8}, $M/\Phi(M)$ is elementary abelian and given that $\Phi(M)\ch M\normal G$, we deduce that $\Phi(M)=\gen1$ and $M$ is elementary abelian (i.e., $y^p=1$ for all $y\in M$).

    \item[\rm4.~\textit{$AM\varsubsetneq G$\/}:] Since $M\subseteq Z(F(G))$ (Step~2), $M$ normalizes $A\cap F(G)$. And since $A$ also normalizes it, we get $A\cap F(G)\normal AM$. In particular, $AM\varsubsetneq G$; otherwise, $A$ would contain the nontrivial normal subgroup $A\cap F(G)$, which is impossible because $\Core_G(A)=1$ (Step~1).

    \item[\rm5.~\textit{Define $M\subseteq L\normal G$\/}:] Write $\bar G=G/M$ and $\bar A=AM/M$. Then $\Core_{\bar G}(\bar A)=\bar L$, with $M\subseteq L$, $\bar L=L/M$. Note that $L\normal G$ and $AL=AM$ because $\bar A\bar L=\bar A$.

    \item[\rm6.~\textit{$B=A\cap L$ satisfies $|B|>|L:B|$\/}:] Firstly observe that
    $$
        |AM:A|=|AL:A|=|L:A\cap L|=|L:B|
    $$
    or $|AM:L|=|A:B|$. Secondly, by Step~4 we have $\bar A\varsubsetneq\bar G$ and we can invoke the inductive hypothesis to establish $|\bar A:\bar L|<|\bar G:\bar A|$, which translates into $|AM:L|<|G:AM|$. Therefore,
    $$
        |L:B|=|AM:A|=\frac{|G:A|}{|G:AM|}<\frac{|G:A|}{|AM:L|}
            =\frac{|G:A|}{|A:B|}\le\frac{|A|}{|A:B|}=|B|.
    $$

    \item[\rm7.~\textit{$L$ is nonabelian\/}:] Suppose that $L$ is abelian. Then the mapping $\phi\colon L\to L$, $x\mapsto x^p$, is an morphism. Moreover, $M\subseteq\ker\phi$ (Step~3). By Dedeking's Proposition~\ref{dedekind}, $MB=M(A\cap L)=MA\cap L=AL\cap L= L$. Using Step~3,
    $$
        \phi(L)=\phi(MB)=\phi(B)\subseteq B\subseteq A.
    $$
    But $\phi(L)\normal G$ because $L\normal G$ and therefore $\phi(L)\subseteq K=\gen1$ (Step~1). In particular, $\phi(B)=\gen1$ and given that $B\subseteq A$ is cyclic, we deduce that $|B|\le p$. By Step~6, $|L:B|<|B|\le p$. Since $L/B=MB/B$, which makes sense because we are supposing $L$ abelian, we see that $L/B$ is a $p$-group (Step~3). It follows that $L=B$. Thus, $L\subseteq A$ and $L\normal G$ (Step~5), we get $L\subseteq K=\gen1$. In particular, $M=\gen1$ (Step~5), which is not the case.

    \item[\rm8.~\textit{$Z(L)$ is cyclic\/}:] Given that $L/M=\bar L=\Core_{\bar G}(\bar A)\subseteq\bar A$ is cyclic and $L$ is nonabelian (Step~7), we deduce that $M\not\subseteq Z(L)$, i.e., $M\cap Z(L)\varsubsetneq M$. Since $Z(L)\ch L\normal G$, the minimality of $M$ implies that $M\cap Z(L)=\gen1$. Then
    $$
        Z(L)=Z(L)/M\cap Z(L)\subseteq L/M
    $$
    is cyclic.
    
    \item[\rm9.~\textit{$\gen1\ne B\cap F(L)\subseteq Z(L)$\/}:] By Corollary~\ref{abelian-intersects-fitting} applied to $B\subgroup L$ (see Step~6), we obtain $B\cap F(L)\ne\gen1$. Using that $M\subseteq Z(F(G))$ (Step~2) and that $F(L)\subseteq F(G)$, we obtain $M\subseteq C_G(F(L))$. Then $MB\subseteq C_G(B\cap F(L))$. Since $MB=L$ (see Step~7), we get $L\subseteq C_G(B\cap F(L))$. In other words, $\gen1\ne B\cap F(L))\subseteq Z(L)$.

    \item[\rm10.~\textit{Contradiction\/}:] Since $Z(L)$ is cyclic (Step~8), $B\cap F(L)\ch Z(L)$ (every subgroup of a cyclic group its characterized by its order). Therefore, from $Z(L)\ch L\normal G$, we deduce that $B\cap F(L)\normal G$. Then $B\cap F(L)$ is a nontrivial (Step~9) normal subgroup of $A$, which is impossible because $\Core_G(A)=K=1$ (Step~1).
\end{description}
The contradiction arouse from supposing that $|A|\ge|G:A|$. Then $|A|<|G:A|$ and the proof is complete.  \end{proof}

\subsection{Problems D}

\begin{probl} Let\/ $G=NA$, where\/ $N\normal G$, $C_A(N)=\gen1$ and\/ $A$ is abelian.
    \begin{enumerate}[\rm a)]
    \item If\/ $F(N)=\gen1$, show that\/ $|A|<|N|$.
    \item If\/ $|N|$ and\/ $|A|$ are coprime, show that\/ $|A|<|N|$.
    \end{enumerate}
\end{probl} 

\begin{solution} Suppose that $|A|\ge|N|$. Then $|A|^2\ge|N||A|\ge |NA|=|G|$ and we can use Corollary~\ref{abelian-intersects-fitting} to establish that $A\cap F(G)\ne\gen1$.

\begin{enumerate}[\rm a)]
    \item By Corollary~\ref{fitting-intersection-normal} $F(G)\cap N=F(N)$. If $F(N)=\gen1$, since both subgroups are normal, we obtain that $F(G)\leftrightarrow N$ [Problem~\ref{problem-1.F.1}]. Then, $F(G)\subseteq C_G(N)$. Therefore,
    $$
        \gen1\ne A\cap F(G)\subseteq A\cap C_G(N)=C_A(N)=\gen1,
    $$
    a contradiction.
        
    \item Let's take the opportunity to see some preliminaries
    
    \begin{quote}\small
        Since $|A\cap N|$ divides both $|A|$ and $|N|$, it is $A\cap N=\gen1$. Therefore, $|G|=|A||N|$.
        
        Take $p\in\spec|G|$ and write $|G|=p^em$ with $e>0$ and $p\perp m$.
        
        If $p\mid|A|$ then $p^e\mid |A|$ and so $\Syl_p(A)\subseteq\Syl_p(G)$. On the other hand, if $P\in\Syl_p(G)$, there exists $x\in G$ for which $P^x\in\Syl_p(A)$. It follows that $P\in\Syl_p(A^y)$ for some $y\in N$. Using that $\Syl_p(A)=\set{O_p(A)}$, we get
        $$
            O_p(G)=\bigcap_{y\in N}O_p(A)^y.
        $$
        Otherwise, if $p\mid|N|$ then $p^e\mid|N|$ and so $\Syl_p(N)\subseteq\Syl_p(G)$. On the other hand, if $P\in\Syl_p(G)$, then $P^x\in\Syl_p(N)$ for some $x\in G$, and therefore $P\in\Syl_p(N)$ because $N\normal G$. As a result, $O_p(G)=O_p(N)$.
    
        In sum,
        $$
            F(G) = F(N)\prod_{p\in\spec|A|}O_p(G).
        $$
        \end{quote}
    
    The solution to the problem proceeds as follows. Pick $a\in A\cap F(G)$, $a\ne1$. We may further assume that $\ord(a)=p$ for some $p\in\spec|A|$. Given that $O_p(G)$ is the only Sylow $p$-group of $F(G)$, we have $a\in O_p(G)$. But $O_p(G)\cap N=\gen1$ because $p\perp|N|$. Therefore, $O_p(G)\leftrightarrow N$, i.e., $O_p(G)\subseteq C_G(N)$. Hence, $a\in A\cap C_G(N)=\gen1$, a contradiction.
\end{enumerate}
\end{solution}

\begin{probl}
    Let\/ $G=NA$, where\/ $N\normal G$, $C_A(N)=\gen1$, and\/ $A\cap N=\gen1$. If\/ $N$ is nontrivial and\/ $A$ is cyclic, show that\/ $|A|<|N|$.
\end{probl}

\begin{solution} Suppose that $|A|\ge|N|$. By Lucchini Theorem~\ref{lucchini-thm} we know that
$$
    \bigcap_{y\in N}A^y=\Core_G(A)\ne\gen1.
$$
But $\Core_G(A)\cap N\subseteq A\cap N=\gen1$ and so $\Core_G(A)\leftrightarrow N$ [Problem~\ref{problem-1.F.1}], i.e., $\Core_A(G)\subseteq C_G(N)$. Since $\Core_G(A)\subseteq A$, we arrive at a contradiction, namely $\gen1\ne\Core_G(A)\subseteq C_A(N)=\gen1$.  \end{solution}
